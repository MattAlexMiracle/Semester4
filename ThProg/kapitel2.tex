\documentclass{article}
\usepackage{listings}

\usepackage[utf8]{inputenc}
\usepackage{amssymb}
\usepackage{lipsum}
\usepackage{amsmath}
\usepackage{fancyhdr}
\usepackage{geometry}
\usepackage{scrextend}
\usepackage[english,german]{babel}
\usepackage{titling}
\setlength{\droptitle}{-3cm}
\usepackage{tikz}
\usepackage{algorithm,algpseudocode}
\usepackage[doublespacing]{setspace}
\usetikzlibrary{datavisualization}
\usetikzlibrary{datavisualization.formats.functions}
\usepackage{polynom}
\usepackage{amsmath}
\usepackage{gauss}
\usepackage{tkz-euclide}
\usetikzlibrary{datavisualization}
\usetikzlibrary{datavisualization.formats.functions}
\title{Übungsblatt 5}
\author{
Alexander Mattick Kennung: qi69dube\\
Kapitel 2
}
\usepackage{import}
\date{\today}
\geometry{a4paper, margin=2cm}
\usepackage{stackengine}
\parskip 1em
\newcommand\stackequal[2]{%
  \mathrel{\stackunder[2pt]{\stackon[4pt]{=}{$\scriptscriptstyle#1$}}{%
  $\scriptscriptstyle#2$}}
 }
\makeatletter
\renewcommand*\env@matrix[1][*\c@MaxMatrixCols c]{%
  \hskip -\arraycolsep
  \let\@ifnextchar\new@ifnextchar
  \array{#1}}
\makeatother
\lstset{
  language=haskell,
}
\lstnewenvironment{code}{\lstset{language=Haskell,basicstyle=\small}}{}
\usepackage{minted}
\usepackage{enumitem}
\setlist{nosep}
\usepackage{titlesec}

\titlespacing*{\subsection}{0pt}{2pt}{3pt}
\titlespacing*{\section}{0pt}{0pt}{5pt}
\titlespacing*{\subsubsection}{0pt}{1pt}{2pt}



\begin{document}
	\maketitle
	\section{Syntax und operationale Semantik}
	\subsubsection{Binäre Relation}
	Teilmenge des Kreuzprodukts zweier (ungleicher) Mengen $R\subseteq X\times Y$ oder infix $xRy$, wie $\leq\subseteq \mathbb{N}\times\mathbb{N}= \{(n,m)\in\mathbb{N}\times\mathbb{N}|\forall k\in\mathbb{N} (m=n+k)\}$.
	\begin{itemize}
		\item reflexiv $\forall x (xRx)$
		\item symmetrisch $\forall x,y( xRy\implies yRx)$
		\item transitiv $\forall x,y,z (xRy\land yRz \implies xRz)$
		\item Präordnung, wenn R reflexiv und transitiv (eine ordnung braucht auch antisymmetrie)
		\item Äquivalenzrelation, wenn R eine Präordnung und symmetrisch ist.
	\end{itemize}
	Gleichheit ist die einzige Äquivalenz und totale Ordnung\\
	Gleichheit mod k ist eine Äq (reflexiv, man kann immer als vielfaches 0 wählen, symmetrisch und transitiv)
	$\mathbb{Z}/_{\equiv k} = \{[n]_{\equiv k}|n\in mathbb{Z}\} = \mathbb{Z}_k$ mit $|\mathbb{Z}_k|=k$ und $[n]_{\equiv k}=\{m|n{\equiv_k}m\}$\\
	Beliebte Relationen\\
	$=$ (weil das aber merkwürdig ist für z.B. equivalenz von Relationen) gibt es auch $id=\Delta=\{(x,x)|x\in X\}$ kleinste Äquivalenztrelation auf X. (jede Äq muss reflexiv sein, also $\Delta$ beinhalten)\\
	Zu $R\subseteq Y\times Z$ und $S\subseteq X\times Y$ kann die komposition:\\
	$R\circ S = \{(x,z)|\exists y (xSy\land yRz)\}$ Achtung applikativer Syntax, also rechts zuerst (S dann R).\\
	Die funktionskomposition kann darauf reduziert werden: $f: X\to Y = Gr f = \{(x,f(x))|x\in X\}$ (nur halt apllikativ: also graph links als input).\\
	Die n-fache Verkettung wird als $R^n$ bezeichnet, wobei $R^0=id$.\\
	Umkehrrelation oder Inverse einer Relation ist wohldefiniert:\\
	$R^- =\{(y,x) | xRy\}\subseteq Y\times X$.\\
	$\leq^-=\geq$, $\leq\circ \leq = \leq $,\\
	$< \subseteq \mathbb{N}\times\mathbb{N}< \circ < = \{(n,m)\in\mathbb{N}\times\mathbb{N}|n+2\leq m\}$  das plus zwei entsteht dadurch,dass bei jedem $<$ mindestens 1 unterschied sein muss.\\
	Bei $<\in\mathbb{Q}\times\mathbb{Q}$ ist jedoch $<\circ < = <$, da zwischen jede Rationale zahl immer eine weiter passt $\forall x,y( x\leq y \implies \exists z(x<z<y))$.\\
	\textbf{Def}. Sei $P\subseteq\{refl, symm, trans\}$.\\
	Der \textbf{P-abschluss} von $R\subseteq X\times X$ ist die kleineste Relation von S mit $R\subseteq S$ und S hat die Eigenschaft P.\\
	Eindeutigkeit, weil geordnete Menge von Relationen.	Existenz z.B. $P=\{trans\}: S= \bigcap \{Q\subseteq X\times X | R\subseteq Q, Q transitiv\}$. Also: man wählt alle Relationen die R beinhalten und die Eigenschaft haben und nimmt dann den Durchschnitt. Der Durchschnitt hat auch immer die Eigenschaft, weil sie nur über $\forall$ definiert sind (und keine disjunktionen auf der rechten seite der implikation verwenden, und FOL sind).\\
	Daraus folgt:
	\textbf{Lemma 2.14 }(Erzeugte equivalenz)
	\begin{itemize}
		\item R ist reflexiv $\iff id\subseteq R$
		\item R ist symm $\iff R^-\subseteq R \iff R^-=R$
		\item R ist tranisitiv $\iff R\circ R\cup R$
	\end{itemize}
	daraus folgt:
	Explizit berechenbare Eigenschaften:\\
	Reflexiver abschluss von R: man muss alle selbstrelationen hinzufügen, also $R\cup \Delta$.\\
	Symmetrischer Abschluss: $R\cup R^-$ weil $(R\cup R^-)^- = R^-\cup R^{-^-} = R^-\cup R$.\\
	Transitiver Abschluss: man braucht nicht nur $R\circ R$ sondern auch die weiteren $R\circ R\circ R$, also $\bigcup\limits^\infty_{n=1} R^n = \{(x,y)| \exists n\geq 1 ((x,y)\in R^n)\}$,
	das heißt, $xR^+y\iff \exists n,x_0,\dots, x_{n+1}(x=x_0Rx_yR\dots Rx_nRx_{n+1}=y)$.\\
	Dies nennt man $R^+$ ähnlich wie bei regulären ausdrücken: es muss mindestens einmal die Relation angewandt werden! (genau genommen sind Reg. Ausdrücke und Relationen isomorph).
	Transitiv-Reflexiver Abschluss (erzeugte Präordnung).
	$R^+\cup \Delta = R^*=\bigcup\limits^\infty_{n=0}R^n$ (auch wie bei regulären ausdrücken, entweder 0 oder n mal).\\
	\textbf{Lemma 2.15 (Erzeugte Äquivalenz)}\\
	1) S symmetrisch $\implies S^+, S^*$ symmetrisch\\
	2) $(R\cup R^-)^*$ ist symmetrisch\\
	3) $(R\cup R^-)^*$ ist die von R erzeugte Äquivalenz.\\
	Beweis: 3) folgt aus 2): erzeugte Äquivalenz S ist\\
	-symmetrisch, also ist $R\cup R^-\subseteq S$\\
	-transitiv und reflexiver also $(R\cup R^-)^*\subseteq S$.\\
	nach (2) ist $(R\cup r^-)^*$ Äquivalenz, also $S\subseteq (R\cup R^-)^*$.\\
	(2) folgt aus (1), da $R\cup R^-$ symmetrisch.\\
	(1) Sei z.B. $xS^*y$, d.h. es existierten $x=x_0Sx_1\dots Sx_n=y$.\\
	zZ.: $yS^*x: y=x_nSx_{n-1}\dots x_1Sx_0=x$ (das inverse von S ist auch in S weil S symmetrisch!)\\
	


\end{document}