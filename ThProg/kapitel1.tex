\documentclass{article}
\usepackage{listings}

\usepackage[utf8]{inputenc}
\usepackage{amssymb}
\usepackage{lipsum}
\usepackage{amsmath}
\usepackage{fancyhdr}
\usepackage{geometry}
\usepackage{scrextend}
\usepackage[english,german]{babel}
\usepackage{titling}
\setlength{\droptitle}{-3cm}
\usepackage{tikz}
\usepackage{algorithm,algpseudocode}
\usepackage[doublespacing]{setspace}
\usetikzlibrary{datavisualization}
\usetikzlibrary{datavisualization.formats.functions}
\usepackage{polynom}
\usepackage{amsmath}
\usepackage{gauss}
\usepackage{tkz-euclide}
\usetikzlibrary{datavisualization}
\usetikzlibrary{datavisualization.formats.functions}
\title{Übungsblatt 5}
\author{
Alexander Mattick Kennung: qi69dube\\
Kapitel 1
}
\usepackage{import}
\date{\today}
\geometry{a4paper, margin=2cm}
\usepackage{stackengine}
\parskip 1em
\newcommand\stackequal[2]{%
  \mathrel{\stackunder[2pt]{\stackon[4pt]{=}{$\scriptscriptstyle#1$}}{%
  $\scriptscriptstyle#2$}}
 }
\makeatletter
\renewcommand*\env@matrix[1][*\c@MaxMatrixCols c]{%
  \hskip -\arraycolsep
  \let\@ifnextchar\new@ifnextchar
  \array{#1}}
\makeatother
\lstset{
  language=haskell,
}
\lstnewenvironment{code}{\lstset{language=Haskell,basicstyle=\small}}{}
\usepackage{minted}
\usepackage{enumitem}
\setlist{nosep}
\usepackage{titlesec}

\titlespacing*{\subsection}{0pt}{2pt}{3pt}
\titlespacing*{\section}{0pt}{0pt}{5pt}
\titlespacing*{\subsubsection}{0pt}{1pt}{2pt}



\begin{document}
	\maketitle
	\section{Überblick}
	\begin{itemize}[noitemsep,topsep=0pt]
	\item Ein programm, das ein Ergebnis liefert soll termineren. Ein programm, das kein Ergebnis liefert (e.g. OS), soll immer antwortfähig bleiben( reaktives System)
	\item Liefert das Programm das korrekte Ergebnis? (bzw bei reaktiven Systemen) verhält es sich richtig
	\end{itemize}
	$\implies$ wir betrachten Semantik.\\
	Planung:
	\begin{itemize}[noitemsep,topsep=0pt]
		\item Termersetzun (technisch alle funk. Programmiersprache FOL termersetzung)
		\item $\lambda$-Kalkül (HOL-Termersetzung $\equiv$ Lisp, $\lambda$ war lange zeit semantikfrei und Lisp ist es immernoch) 
		\item (Ko-)Datentyp, (Ko-)Induktion
		\item reguläre Ausdrücke und minimierung von Automaten
	\end{itemize}
	Literatur:
	\begin{itemize}[noitemsep,topsep=0pt]
		\item TES: Term Rewriting and all that (für die Leute dies genau wissen wollen), J.W. Klop term rewriting systems (kurz$\approx 80$ seiten, kostenlos), Giesl: Termersetzungsysteme (Polynomordnung/Terminierung)
		\item $\lambda$-Kalkül: Lambda Calculi with Types (einfach, vom Jesus des Lambda, kostenlos), Nipkow: Lambda-Kalkül (anders als in der Vorlesung)
		\item (Ko-)Induktion: A Tutorial on (Co-)Algebras and (Co-)Induktion (lesbar geschrieben, rel kurz), Automat and Coinduction-An exercise in Coalgebra
		\item Reg. Ausdrücke/Automaten: Hopcraft/Ullman/Motwani (bibel), Pitts: Lecture notes on Regular Languages and Finite Automata (kostenlos)
	\end{itemize}
	\section{Termersetzungsysteme}
	$\mathbb{N}\ni 0$, Implikation $\iff$ mit doppelstrich.\\
	- Umformung von termen gemäß \underline{gerichteter} Gleichungen (sukzessive, erschöpfend)
	Anwendungen:
	\begin{itemize}[noitemsep,topsep=0pt]
		\item (algebraische) Spezifikation (nicht mehr 100\% aktuell, andere sind beliebter)
		\item Programmverifikation
		\item automatisches Beweisen (Coq, Anabell)
		\item Computeralgebra (Gröbnerbasen/Buchbergeralgo\footnote{https://de.wikipedia.org/wiki/Buchberger-Algorithmus})
		\item Programmierung: Turingvollständig \& Grundlage der funktionalen Programmierung.
	\end{itemize}
	\begin{minted}{haskell}
	data Nat = Zero | Suc Nat
	plus::Nat->Nat->Nat
	plus Zero x = x
	plus (Suc y) x = Suc (plus y x)
	\end{minted}
	\underline{Auswertung}:\\
	 $plus(Suc(Suc Zero)) (Suc Zero) \to Suc(plus (Suc Zero) (Suc Zero)) \to 
		Suc (Suc(plus Zero (Suc Zero)) \to Suc(Suc(Suc(zero)))$\\
	\underline{``Beweisen''}:\\
	$(2+x) +y = 2+(x+y)\iff p (S(S(x))) y \to S( p( S(x) y))\to S( S ( p x y))$\\
	\underline{Optimieren}:\\
	plus (plus x y) z = plus x (plus y z) \underline{assoziative Gesetz}
	die Linke Seite braucht (nach obiger Suc-definition) 2x+y.\\
	Die rechte Seite braucht x+y\\
	(Weil nur das erste argument eine mehrfachauswertung der funktion bewirkt)\\
	geht ``plus x y = plus y x'' auch?\\
	NEIN, weil man hiermit in eine unendliche schleife geraten könnte! alternativ z.B. $y<x$ erzwingen, aber dann ist die frage: lohnt sich das? (hier nicht, weil der vergleich genau diese zeitdifferenz kostet).\\
	\underline{Verifikation}:\\
	$p ( S ( S Z)) x = p (S Z) (S x)$ ``Anforderung''\\
	$p (S ( S Z)) x \to S (p ( S Z ) x)\to S( S ( p Z x)) \to S ( S x)$\\
	$p (S Z) (S x)\to S ( p Z (S x)) \to S ( S x)$\\
	Beide werden in die gleiche Normalform reduziert ( NF bezeichnet hier einen Zustand, der nicht mehr reduziert werden kann!)\\
	Problem: Was ist wenn man nie eine NF erreicht? Wenn man ungleiche Normalformen erhält, könnte die Gleichung trotzdem gleich sein, wenn die Normalformbildung nicht deterministisch ist!\\

\end{document}