\documentclass{article}
\usepackage{listings}
\usepackage{mathrsfs}
\usepackage{cancel}
\usepackage[utf8]{inputenc}
\usepackage{amssymb}
\usepackage{lipsum}
\usepackage{framed}
\usepackage{fancyhdr}
\usepackage{geometry}
\usepackage{scrextend}
\usepackage[english,german]{babel}
\usepackage{titling}
\usepackage{bm}
\usepackage{verbatim}
\usepackage{fourier}
\setlength{\droptitle}{-3cm}
\usepackage{tikz}
\usepackage{algorithm,algpseudocode}
\usepackage[doublespacing]{setspace}
\usepackage{minted}
\usetikzlibrary{datavisualization}
\usetikzlibrary{datavisualization.formats.functions}
\usepackage{polynom}
\usepackage{amsmath,amsthm}
\usepackage{gauss}
\usepackage{euscript}
\usepackage{tkz-euclide}
\usepackage{stackengine}
\usepackage{bussproofs}
\usepackage{tikz-cd}

\usetikzlibrary{datavisualization}
\usetikzlibrary{datavisualization.formats.functions}
\title{Übungsblatt 5}
\author{
Alexander Mattick Kennung: qi69dube\\
Kapitel 1
}
\usepackage{import}
\date{\today}
\geometry{a4paper, margin=2cm}
\usepackage{stackengine}
\parskip 1em
\newcommand\stackequal[2]{%
  \mathrel{\stackunder[2pt]{\stackon[4pt]{=}{$\scriptscriptstyle#1$}}{%
  $\scriptscriptstyle#2$}}
 }
\makeatletter
\renewcommand*\env@matrix[1][*\c@MaxMatrixCols c]{%
  \hskip -\arraycolsep
  \let\@ifnextchar\new@ifnextchar
  \array{#1}}
\makeatother
\lstset{
  language=haskell,
}
\lstnewenvironment{code}{\lstset{language=Haskell,basicstyle=\small}}{}
\usepackage{enumitem}
\setlist[itemize]{noitemsep, topsep=0pt}
\usepackage{titlesec}
\newcommand{\nto}{\nrightarrow}
\newcommand{\smallAscr}{\scriptscriptstyle\mathcal{A}}
%\newcommand{\nsqsubseteq}{\xout{\sqsubseteq}}
\title{Vorlesung 2}
\titlespacing*{\subsection}{0pt}{2pt}{3pt}
\titlespacing*{\section}{0pt}{0pt}{5pt}
\titlespacing*{\subsubsection}{0pt}{1pt}{2pt}
\newtheorem{satz}{Satz}
\newtheorem{korrolar}{Korrolar}[section]
\newtheorem{lemma}{Lemma}[section]

\theoremstyle{definition}
\newtheorem{beweis}{Beweis}[section]
\newtheorem{beispiel}{Beispiel}[section]
\newtheorem{definition}{Definition}[section]


\begin{document}
	\section{Übung 4}
	\subsection{}
	\begin{minted}{haskell}
	reverse::List a->List a
	reverse Nil = Nil
	reverse (Cons x xs) = snoc (reverse xs) x
	\end{minted}
	\[\forall xs\ ys.reverse (xs\oplus ys)= (reverse\ ys)\oplus(reverse\ xs)\]
	Sei ys::List a beliebig aber fest.(so schwachsinnige diese schreibweise auch ist)\\
	I.A. xs =Nil\\
	wir erhalten¸\\
	reverse (Nil$\oplus$ ys) = reverse ys\\
	und\\
	(reverse ys)$\oplus$(reverse Nil) = (reverse ys)$\oplus$Nil =  reverse ys\\
	per Übung 2.3.a\\
	Der induktionsanfang gilt also für beliebige ys.\\
	I.V.: xs=Cons z zs, für zs gilt\\
	reverse (zs$\oplus$ ys)= (reverse ys)$\oplus$(reverse zs)\\
	linke Seite Umformen:\\
	reverse ((Cons z zs)$\oplus$ ys) $\stackrel{def\ \oplus}{=}$ reverse (Cons z (zs$\oplus$ys)) $\stackrel{def\ reverse}{=}$ snoc (reverse (zs$\oplus$ys)) z $\stackrel{def\ IV}{=}$\\
	snoc ((reverse ys)$\oplus$(reverse zs)) z\\
	von der anderen Seite\\
	(reverse ys)$\oplus$(reverse (Cons z zs)) $\stackrel{def\ reverse}{=}$ (reverse ys)$\oplus$(snoc (reverse zs) z) $\stackrel{(1)}{=}$ snoc ((reverse ys)$\oplus$(reverse zs)) z\\
	(1) Anwendung von 2.3.b von rechts nach links.\\
	Somit gilt auch die I.V. also gilt die Aussage.\\
	\subsection{}
	\subsubsection{}
	\[\forall xs\ ys.\ reverse'\ xs \ ys =(reverse'\ xs\ Nil)\oplus ys\]
	(idee: der rechte parameter bleibt unverändert, bis auf hinzufügen der Liste im ``stackformat'')\\
	I.A. die Aussage gilt für xs=Nil\\
	reverse' Nil  ys= ys $\stackrel{def\ \oplus}{=}$Nil $\oplus $ys= reverse (Nil Nil) $\oplus $ys\\
	per definition von reverse'.\\
	I.V. xs = Cons z zs\\
	Die Aussage gilt für zs: reverse' zs bel=(reverse' zs Nil)$\oplus$bel\\
	reverse' (Cons z zs)  ys$\stackrel{def\ reverse'}{ = }$ reverse' zs (Cons z ys) $\stackrel{IV}{=}$ (reverse' zs Nil) $\oplus$ (Cons z ys)\\
	andere Seite\\
	(reverse' (Cons z zs) Nil)$\oplus$ ys $\stackrel{def\ reverse'}{=}$ (reverse'  zs (Cons z Nil))$\oplus$ ys $\stackrel{IV}{=}$ ((reverse' zs Nil) $\oplus$ (Cons z Nil))$\oplus$ ys $\stackrel{(1)}{=} $\\
	(reverse' zs Nil) $\oplus$( (Cons z Nil)$\oplus$ ys) $\stackrel{def\ \oplus}{=}$ (reverse' zs Nil) $\oplus$( Cons z (Nil$\oplus$ ys)) $\stackrel{def\ \oplus}{=}$ (reverse' zs Nil) $\oplus$( Cons z ys)\\
	wobei in (1) die Assoziativität der konkatenation benutzt wurde (Beweis aus Vorlesung cc und $\oplus$ sind $\alpha$-Äquivalent)\\
	\subsubsection{}
	I.A.: sei xs =Nil\\
	reverse Nil =Nil $\stackrel{def\ reverse'}{=}$ (reverse' Nil Nil)\\
	gilt.\\
	I.V. xs = Cons z zs\\
	Die Aussage gilt für zs: reverse zs = reverse' zs Nil\\
	reverse (Cons z zs) $\stackrel{def\ reverse}{=}$ snoc (reverse zs) z $\stackrel{IV}{=}$ snoc (reverse' zs Nil) z\\
	reverse' (Cons z zs) Nil $\stackrel{def\ reverse'}{=}$ reverse' zs (Cons z Nil) $\stackrel{Lemma\ a)}{=}$ (reverse' zs Nil) $\oplus$ (Cons z Nil) $\stackrel{3.3}{=}$\\
	snoc (reverse' zs Nil) z\\
	Somit gilt die Aussage.\\
	\section{}
	\subsection{}
	\[\forall f\ g\ xs.(map\ f.map\ g)\ xs = map\ (f.g)\ xs\]
	Sei f,g beliebig, aber fest.\\
	I.A.: xs=Nil\\
	(map f.map g) Nil = $\lambda x. (map\ f)((map\ g)(x))$ Nil = (map f)((map g)(Nil)) = (map f)(Nil) = Nil\\
	Anwenden der definition von (.) und dann 2 mal von map \\
	map (f.g) Nil = Nil.\\
	definition von map.\\
	I.V. xs = Cons z zs\\
	Die Aussage gilt für zs: (map f.map g) zs = map (f.g) zs\\
	(map f.map g) (Cons z zs) $\stackrel{def\ (.)}{=}$ $\lambda x. (map\ f)((map\ g)(x))$ (Cons z zs) $\stackrel{applikation}{=}$  (map f)((map g)(Cons z zs)) $\stackrel{def\ map}{=}$ (map f)(Cons (g z) (map g zs)) $\stackrel{def\ map}{=}$ Cons (f (g z)) (map f (map g zs))\\
	Andere Seite\\
	map (f.g) (Cons z zs) $\stackrel{def\ map}{=}$ Cons ((f.g) z) (map (f.g) zs) $\stackrel{def\ (.)}{=}$ Cons (($\lambda x. f(g\ x)$) z) (map (f.g) zs) $\stackrel{\beta}{=}$\\
	Cons (f(g z) ) (map (f.g) zs) $\stackrel{IV}{=}$ Cons (f(g z) ) ((map f.map g) zs) $\stackrel{def\ (.)}{=}$ Cons (f(g z) ) ($\lambda x. (map\ f)((map\ g)(x))$ zs) $\stackrel{\beta}{=}$\\
	Cons (f(g z) ) ((map f)((map g)(zs))) $\stackrel{applikation}{=}$ Cons (f(g z) ) ((map f)((map g zs)) $\stackrel{applikation}{=}$\\
	Cons (f(g z) ) (map f (map g zs))\\
	Somit gilt die Aussage.\\
	\subsubsection{}
	\[\forall f\ ys xs.map\ f\ (xs\oplus ys) = (map\ f\ xs) \oplus (map\ f\ ys)\]
	I.A. xs=Nil\\
	map f $(Nil\oplus ys)$ $\stackrel{def\ \oplus}{=}$ map f ys\\
	(map f $Nil$) $\oplus$ (map f ys) $\stackrel{def\ map}{=}$ Nil $\oplus$ map f ys $\stackrel{def\ oplus}{=}$ map f ys\\
	I.V. xs = Cons z zs\\
	Die Aussage gilt für zs: map f (zs$\oplus$ ys) = (map f zs) $\oplus$ (map f ys)\\
	map f ((Cons z zs)$\oplus$ ys) $\stackrel{def\ \oplus}{=}$ map f ((Cons z ( zs$\oplus$ ys) $\stackrel{def\ map}{=}$  Cons (f z) (map ( zs$\oplus$ ys)) $\stackrel{IV}{=}$\\
	Cons (f z) ((map f zs) $\oplus$ (map f ys))\\
	Andere Seite.\\
	(map f (Cons z zs)) $\oplus$ (map f ys) $\stackrel{def\ map}{=}$\\
	( Cons (f z) (map f zs))) $\oplus$ (map f ys) $\stackrel{def\ \oplus}{=}$ Cons (f z) ((map f zs) $\oplus$ (map f ys))\\
	Aussage gilt.\\
	\section{}
	\subsection{}
	\[\forall t. mirror\ (mirror\ t) = t\]
	I.A. t=Leaf.\\
	mirror (mirror (Leaf)) $\stackrel{def\ mirror}{=}$ mirror (Leaf) $\stackrel{def\ mirror}{=}$ Leaf =Leaf\\
	gilt.\\
	I.V. t = Bin left x right\\
	Die Aussage gilt für left und right.\\
	mirror (mirror left) = left\\
	mirror (mirror right) = right\\
	mirror (mirror (Bin left x right)) $\stackrel{def\ mirror}{=}$ mirror (Bin (mirror right) x (mirror left)) $\stackrel{2\times IV}{=}$ mirror (Bin right x left) $\stackrel{def\ mirror}{=}$ (Bin (mirror left) x (mirror right) $\stackrel{IV}{=}$ Bin left x right
	\subsection{}
	\[\forall t. inorder\ (mirror\ t) = reverse\ (inorder\ t)\]
	I.A. t = Leaf\\
	inorder (mirror (Leaf)) $\stackrel{def\ mirror}{=}$ inorder (Leaf) $\stackrel{def\ inorder}{=}$ Nil\\
	andere Seite:\\
	reverse (inorder Leaf) $\stackrel{def\ indrder}{=}$ reverse Nil $\stackrel{def\ reverse}{=}$ Nil\\
	gilt.\\
	I.V. t = Bin left x right\\
	Die Aussage gilt für left und right.\\ 
	inorder (mirror left) = reverse (inorder left)\\
	inorder (mirror right) = reverse (inorder right)\\
	inorder (mirror (Bin left x right)) $\stackrel{def\ mirror}{=}$ inorder (Bin (mirror right) x (mirror left)) $\stackrel{def\ inorder}{=}$ inorder (mirror right) $\oplus$ (Cons x (inorder (mirror left))) $\stackrel{2\times IV}{=}$  reverse (inorder right) $\oplus$ (Cons x (reverse (inorder left))\\
	Andere Seite:\\
	reverse (inorder (Bin left x right)) $\stackrel{def\ inorder}{=}$ reverse (inorder left $\oplus$ (Cons x (inorder right))) $\stackrel{4.1}{=}$  (reverse (Cons x (inorder right)))) $\oplus$ reverse (inorder left) $\stackrel{def\ reverse}{=}$ (snoc (reverse (inorder right)) x) $\oplus$ reverse (inorder left) $\stackrel{Lemma\ A}{=}$\\
	reverse (inorder right) $\oplus$ (Cons x (reverse (inorder left))\\
	Die Aussage gilt.\\
	\underline{Lemma A:}\\
	(snoc xs x) $\oplus$ ys = xs $\oplus$ (Cons x ys)\\
	I.A. xs = Nil\\
	(snoc Nil x) $\oplus$ ys$\stackrel{def\ snoc}{=}$ (Cons x Nil) $\oplus$ ys $\stackrel{def\ \oplus}{=}$ Cons x (Nil$\oplus$ ys) $\stackrel{def\ \oplus}{=}$ Cons x ys \\
	andere Seite:\\
	Nil $\oplus$ (Cons x ys) $\stackrel{def\ \oplus}{=}$ Cons x ys.\\
	gilt.\\
	I.V. xs=Cons z zs\\
	Die Aussage gilt für zs: (snoc zs x) $\oplus$ ys = zs $\oplus$ (Cons x ys).\\
	(snoc (Cons z zs) x) $\oplus$ ys $\stackrel{def\ snoc}{=}$ (Cons z (snoc  zs x))$\oplus$ ys $\stackrel{def\ \oplus}{=}$ Cons z ((snoc zs x)$\oplus$ ys) $\stackrel{IV}{=}$ Cons z (zs $\oplus$ (Cons x ys))\\
	Andere Seite:\\
	(Cons z zs) $\oplus$ (Cons x ys) $\stackrel{def\ \oplus}{=}$  Cons z (zs$\oplus$(Cons x ys)).\\
	Das Lemma ist gültig.\\
\end{document}




