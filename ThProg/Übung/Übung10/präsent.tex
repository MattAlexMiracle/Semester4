\documentclass{article}
\usepackage{listings}
\usepackage{mathrsfs}
\usepackage[utf8]{inputenc}
\usepackage{amssymb}
\usepackage{lipsum}
\usepackage{amsmath}
\usepackage{fancyhdr}
\usepackage{geometry}
\usepackage{scrextend}
\usepackage[english,german]{babel}
\usepackage{titling}
\setlength{\droptitle}{-3cm}
\usepackage{tikz}
\usepackage{algorithm,algpseudocode}
\usepackage[doublespacing]{setspace}
\usetikzlibrary{datavisualization}
\usetikzlibrary{datavisualization.formats.functions}
\usepackage{polynom}
\usepackage{amsmath}
\usepackage{gauss}
\usepackage{tkz-euclide}
\usepackage{minted}
\usetikzlibrary{datavisualization}
\usetikzlibrary{datavisualization.formats.functions}
\author{
Alexander Mattick Kennung: qi69dube\\
Kapitel 1
}
\usepackage{import}
\date{\today}
\geometry{a4paper, margin=2cm}
\usepackage{stackengine}
\parskip 1em
\newcommand\stackequal[2]{%
  \mathrel{\stackunder[2pt]{\stackon[4pt]{=}{$\scriptscriptstyle#1$}}{%
  $\scriptscriptstyle#2$}}
 }
\makeatletter
\renewcommand*\env@matrix[1][*\c@MaxMatrixCols c]{%
  \hskip -\arraycolsep
  \let\@ifnextchar\new@ifnextchar
  \array{#1}}
\makeatother
\lstset{
  language=haskell,
}
\lstnewenvironment{code}{\lstset{language=Haskell,basicstyle=\small}}{}
\usepackage{enumitem}
\setlist{nosep}
\usepackage{titlesec}
\usepackage{ stmaryrd }
\usepackage{verbatim}
\usepackage{tikz-qtree}
\usepackage{bussproofs}

\titlespacing*{\subsection}{0pt}{2pt}{3pt}
\titlespacing*{\section}{0pt}{0pt}{5pt}
\titlespacing*{\subsubsection}{0pt}{1pt}{2pt}
\title{Übung 7}


\begin{document}
	\maketitle
	1.\\
	zZ.: $\forall x\ xs. length(snoc\ x\ xs) = 1+length\ xs$\\
	Seien x:a, xs:List a. per induktion über xs\\
	I.A.:\\
	length (snoc Nil x) = length (Cons x Nil) =Suc (length Nil) = Suc(0)=1 \\
	1+ length Nil = 1+0 = Suc(0)\\
	Annahme von vorherigen Blatt.\\
	I.S.\\
	Induktionshypothese die Aussage gilt für ein ys:List a und beliebige y:a.\\
	IH length (snoc ys x) = 1+length(ys)\\
	zZ.: length (snoc (Cons y ys) x) =length (Cons y (snoc ys x)) = Suc (length (snoc ys x)) = Suc(1+length ys)\\
	1+length (Cons y ys)= 1+(1+length (ys)) = Suc(1+length ys)\\
	Annahme von vorherigen Blatt.\\
	2. zZ $\forall xs. length (reverse\ xs) = length\ xs$\\
	i.A.:\\
	sei xs=Nil\\
	length(reverse Nil) = length Nil\\
	length Nil\\
	fertig.\\
	I.S.\\
	xs = cons y ys\\
	I.H. length (reverse ys) = length ys\\
	length (reverse (Cons y ys)) = length (snoc (reverse ys) y) $\stackrel{A1}{=}$ 1+length (reverse ys)$\stackrel{IH}{=}$ 1+length ys\\
	length (Cons y ys) = suc (length ys) = 1+length ys\\
	Annahme von vorherigen Blatt.\\
	



\end{document}