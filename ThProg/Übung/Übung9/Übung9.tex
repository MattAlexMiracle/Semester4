\documentclass{article}
\usepackage{listings}
\usepackage{mathrsfs}
\usepackage[utf8]{inputenc}
\usepackage{amssymb}
\usepackage{lipsum}
\usepackage{amsmath}
\usepackage{fancyhdr}
\usepackage{geometry}
\usepackage{scrextend}
\usepackage[english,german]{babel}
\usepackage{titling}
\setlength{\droptitle}{-3cm}
\usepackage{tikz}
\usepackage{algorithm,algpseudocode}
\usepackage[doublespacing]{setspace}
\usetikzlibrary{datavisualization}
\usetikzlibrary{datavisualization.formats.functions}
\usepackage{polynom}
\usepackage{amsmath}
\usepackage{gauss}
\usepackage{tkz-euclide}
\usepackage{minted}
\usetikzlibrary{datavisualization}
\usetikzlibrary{datavisualization.formats.functions}
\author{
Alexander Mattick Kennung: qi69dube\\
Kapitel 1
}
\usepackage{import}
\date{\today}
\geometry{a4paper, margin=2cm}
\usepackage{stackengine}
\parskip 1em
\newcommand\stackequal[2]{%
  \mathrel{\stackunder[2pt]{\stackon[4pt]{=}{$\scriptscriptstyle#1$}}{%
  $\scriptscriptstyle#2$}}
 }
\makeatletter
\renewcommand*\env@matrix[1][*\c@MaxMatrixCols c]{%
  \hskip -\arraycolsep
  \let\@ifnextchar\new@ifnextchar
  \array{#1}}
\makeatother
\lstset{
  language=haskell,
}
\lstnewenvironment{code}{\lstset{language=Haskell,basicstyle=\small}}{}
\usepackage{enumitem}
\setlist{nosep}
\usepackage{titlesec}
\usepackage{ stmaryrd }
\usepackage{verbatim}
\usepackage{tikz-qtree}
\usepackage{bussproofs}
\usepackage{xcolor}

\titlespacing*{\subsection}{0pt}{2pt}{3pt}
\titlespacing*{\section}{0pt}{0pt}{5pt}
\titlespacing*{\subsubsection}{0pt}{1pt}{2pt}
\title{Übung 7}
\newcommand{\Todo}[1]{\underline{\color{red}{#1}}}

\begin{document}
	\section{Übung 3}
	(ignorieren wir, dass Blatt 8 Übung 5 maximum nicht minimum definiert)\\
	minimum arbeitet von $List\ Nat\to Nat$, da map selbst auf listen operiert und auf jedem Element ``f'' (bzw hier length:$List\ a\to Nat$) ausführt. muss map length auf listen von listen operieren und minimum gibt die minimale länge der listen in den listen der Listen zurück:\\
	\begin{minted}{haskell}
	minimum.(map length)::List (List a)->Nat
	\end{minted}
	2.\\
	map ersetzt jedes element der List mit einem transformierten, aber typgleichen element der Liste:\\
	\Todo{benutzen die foldl mit argumentreihenfolge anders herum? MINDFUCK}\\
	\Todo{es gehört $Foldable\ t\Rightarrow (b \to a \to b) \to b \to t a \to b$}\\
	\Todo{UND DAS GLEICHE BEI DER ARGUMENTFUNKTION!!}
	\begin{minted}{haskell}
	-- reverse is necessary because the construction is resolved from back to front (so the first in the list is going to be )
	map f xs =reverse $foldl  Nil (\x-> Cons (f x)) xs
	\end{minted}
	\begin{minted}{haskell}
	reverse xs = foldL c g xs Nil
	 where
	 	--c::List a->List a
	 	c = id
	 	g::a->(List a->List a)->(List a->List a)
	 	g y f ys = Cons (f y) ys
	\end{minted}
	\section{Übung 4}
	Hintergrund:\\
	die n-te quadratzahl kann von der vorherigen über:\\
	\[n^2 = (n-1)^2+(n-1)+n\]
	für diese berechnung brauchen wir also die zahl n selbst, sowie das vorherige quadrat
	\begin{minted}{haskell}
	-- initialfall 0**2 = 0 zu (n,n**2)
	c = (0,0)
	h x = (Suc(fst x), (snd x)+(fst x)+ Suc(fst x))
	g = fst
	\end{minted}
	für die 2. Funktion:
	\begin{minted}{haskell}
	-- wir dürfen annehemen, dass n=0 wahr ist (laufvariable, letzter treffer)
	c= (0,0)
	h x= if p (fst x) then (Suc(fst x), fst x) else (Suc(fst x), snd x)
	g= snd
	\end{minted}
\end{document}