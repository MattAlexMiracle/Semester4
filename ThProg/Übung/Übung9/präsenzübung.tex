\documentclass{article}
\usepackage{listings}
\usepackage{mathrsfs}
\usepackage[utf8]{inputenc}
\usepackage{amssymb}
\usepackage{lipsum}
\usepackage{amsmath}
\usepackage{fancyhdr}
\usepackage{geometry}
\usepackage{scrextend}
\usepackage[english,german]{babel}
\usepackage{titling}
\setlength{\droptitle}{-3cm}
\usepackage{tikz}
\usepackage{algorithm,algpseudocode}
\usepackage[doublespacing]{setspace}
\usetikzlibrary{datavisualization}
\usetikzlibrary{datavisualization.formats.functions}
\usepackage{polynom}
\usepackage{amsmath}
\usepackage{gauss}
\usepackage{tkz-euclide}
\usepackage{minted}
\usetikzlibrary{datavisualization}
\usetikzlibrary{datavisualization.formats.functions}
\author{
Alexander Mattick Kennung: qi69dube\\
Kapitel 1
}
\usepackage{import}
\date{\today}
\geometry{a4paper, margin=2cm}
\usepackage{stackengine}
\parskip 1em
\newcommand\stackequal[2]{%
  \mathrel{\stackunder[2pt]{\stackon[4pt]{=}{$\scriptscriptstyle#1$}}{%
  $\scriptscriptstyle#2$}}
 }
\makeatletter
\renewcommand*\env@matrix[1][*\c@MaxMatrixCols c]{%
  \hskip -\arraycolsep
  \let\@ifnextchar\new@ifnextchar
  \array{#1}}
\makeatother
\lstset{
  language=haskell,
}
\lstnewenvironment{code}{\lstset{language=Haskell,basicstyle=\small}}{}
\usepackage{enumitem}
\setlist{nosep}
\usepackage{titlesec}
\usepackage{ stmaryrd }
\usepackage{verbatim}
\usepackage{tikz-qtree}
\usepackage{bussproofs}

\titlespacing*{\subsection}{0pt}{2pt}{3pt}
\titlespacing*{\section}{0pt}{0pt}{5pt}
\titlespacing*{\subsubsection}{0pt}{1pt}{2pt}
\title{Übung 7}


\begin{document}
	\maketitle
	\begin{minted}{haskell}
	foldb::a->a->Bool->a
	foldb t f True = t
	foldb t f False = f

	data Nat = Zero | Succ Nat deriving (Show, Eq)
	foldn::a->(a->a)->Nat->a
	foldn z s Zero = z
	foldn z s (Succ n) = s (foldn z s n)

	add::Nat->Nat->Nat
	add n = foldn n Succ

	mult::Nat->Nat->Nat
	mult n = foldn Zero (add n)

	exp::Nat->Nat->Nat
	exp n = foldn (Suc Zero) (mult n)

	data Tree a = Leaf a | Node (Tree a) (Tree a) deriving (Show, Eq)
	data List a = Nil | Cons a (List a) deriving (Show, Eq)
	-- sprich: überall wo ``Tree'' steht schreibt man ein b, alles andere lässt man unverändert
	-- Merke: Leaf a ist kurzschreibweise für a->Tree a
	foldL::b->(a->b->b)->List a->b
	foldL n c Nil =n
	foldL n c (Cons x xs) = Cons (c x) (foldL n c xs)

	foldT::(a->b)->(b->b->b)->Tree a-> b
	foldlT l n (Leaf a) = l a
	foldlT l n (Node a b) = n (foltdT l n a) (foltdT l n b)

	scanL::b->(a->b->b)->List a->List b
	scanL n f Nil= Cons (foldL n f Nil) Nil
	scanL n f (Cons x xs)= Cons (foldL n f (Cons x xs)) (scanL n f xs)

	length = foldL 0 (\x->Succ)

	snoc x=  foldL (Cons x Nil) (Cons x xs)

	reverse = foldL Nil snoc

	concat l1 l2 = foldL l2 Cons l1

	front  = foldT (\x -> Cons x Nil) concat
	\end{minted}



\end{document}