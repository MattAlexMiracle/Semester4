\documentclass{article}
\usepackage{listings}
\usepackage{mathrsfs}
\usepackage[utf8]{inputenc}
\usepackage{amssymb}
\usepackage{lipsum}
\usepackage{amsmath}
\usepackage{fancyhdr}
\usepackage{geometry}
\usepackage{scrextend}
\usepackage[english,german]{babel}
\usepackage{titling}
\setlength{\droptitle}{-3cm}
\usepackage{tikz}
\usepackage{algorithm,algpseudocode}
\usepackage[doublespacing]{setspace}
\usetikzlibrary{datavisualization}
\usetikzlibrary{datavisualization.formats.functions}
\usepackage{polynom}
\usepackage{amsmath}
\usepackage{gauss}
\usepackage{tkz-euclide}
\usepackage{minted}
\usepackage{tikz-qtree}

\usetikzlibrary{datavisualization}
\usetikzlibrary{datavisualization.formats.functions}
\author{
Alexander Mattick Kennung: qi69dube\\
Kapitel 1
}
\usepackage{import}
\date{\today}
\geometry{a4paper, margin=2cm}
\usepackage{stackengine}
\parskip 1em
\newcommand\stackequal[2]{%
  \mathrel{\stackunder[2pt]{\stackon[4pt]{=}{$\scriptscriptstyle#1$}}{%
  $\scriptscriptstyle#2$}}
 }
\makeatletter
\renewcommand*\env@matrix[1][*\c@MaxMatrixCols c]{%
  \hskip -\arraycolsep
  \let\@ifnextchar\new@ifnextchar
  \array{#1}}
\makeatother
\lstset{
  language=haskell,
}
\lstnewenvironment{code}{\lstset{language=Haskell,basicstyle=\small}}{}
\usepackage{enumitem}
\setlist{nosep}
\usepackage{titlesec}
\usepackage{ stmaryrd }
\usepackage{verbatim}
\usepackage{tikz-qtree}
\usepackage{bussproofs}

\titlespacing*{\subsection}{0pt}{2pt}{3pt}
\titlespacing*{\section}{0pt}{0pt}{5pt}
\titlespacing*{\subsubsection}{0pt}{1pt}{2pt}
\title{Übung 7}


\begin{document}
	\maketitle
	\section{aufgabe 3}
	Das problem der Subjektexpansion ist typischerweise, dass untypisierbare Terme durch beta-Reduktion gelöscht werden könnten.\\
	Hier ist jedoch ist festgelegt, dass sowohl s als auch t typiserbar ist.\\
	Sei $\Gamma[v\mapsto \alpha]$\\
	$t=a:b$\\
	$s=(\lambda v.v)a$\\
	\\
	\AxiomC{}
	\UnaryInfC{$\Gamma[v\mapsto \alpha]\vdash v:\alpha$}
	\UnaryInfC{$\Gamma\vdash\lambda v.v:\alpha\to\alpha$}
	\AxiomC{}
	\UnaryInfC{$\Gamma\vdash a:\alpha$}
	\BinaryInfC{$\Gamma\vdash(\lambda v.v)a$}
	\DisplayProof\\
	Im Fall t ist das a an keine Weiteren typattribute gebunden (Es ``weis'' nichts mehr davon, dass es evtl. in einem Kontext s gestanden ist, in dem es relevant war ein $\alpha$ zu sein).\\
	Bei s ist es jedoch, aufgrund des Kontexts gezwungenermaßen so, dass man $a:\alpha$ zuweist.\\
	$S= \lambda xyz.xz(yz)$\\
	\begin{tikzpicture}[sibling distance=10em,
	every node/.style = {shape=rectangle, align=center},
	level 1/.style={sibling distance=10em},
	level 2/.style={sibling distance=13em},
	level 3/.style={sibling distance=13em},
	level 4/.style={sibling distance=15em},
	level 4/.style={sibling distance=20em},
	level 5/.style={sibling distance=10em},
	]
	\node{$PT(\emptyset,\lambda xyz.xz(yz),\alpha)$}
	child {node {$PT(x:a_1,\lambda yz.xz(yz),a_2)$}
		child {node {$PT(\{x:a_1,y:a_3\},\lambda z.xz(yz),a_4)$}
			child {node {$PT(\underbrace{\{x:a_1,y:a_3, z:a_5\}}_{\Gamma},xz(yz),a_6)$}
			child {node {$PT(\Gamma,xz, a_7\to a_6)$}
				child {node {$PT(\Gamma,x,a_8\to a_7\to a_6)$}}
				child {node {$PT(\Gamma,z,a_8)$}}
			}
			child {node {$PT(\Gamma,yz, a_7)$}
				child {node {$PT(\Gamma,y, a_9\to a_7)$}}
				child {node {$PT(\Gamma,z, a_9)$}}
			}
			}
			child {node {$\{a_5\to a_6\doteq a_4\}$}}
		}
		child {node {$\{a_3\to a_4\doteq a_2\}$}}
	}
	child {node {$\{a_1\to a_2\doteq \alpha\}$}};
	\end{tikzpicture}\\
	Der letzte schritt wurde bei jedem Teilbaum aus Formatierungsgründen weggelassen.\\
	$\{a_1\to a_2\doteq \alpha,a_3\to a_4\doteq a_2,a_5\to a_6\doteq a_4,a_8\to a_7\to a_6\doteq a_1, a_8\doteq a_5,a_3\doteq a_8\to a_7,a_5\doteq a_9\}$\\
	elim $\{a_1\to a_2\doteq \alpha,a_3\to a_4\doteq a_2,a_9\to a_6\doteq a_4,a_8\to a_7\to a_6\doteq a_1, a_8\doteq a_9,a_3\doteq a_8\to a_7,a_5\doteq a_9\}$\\
	elim $\{(a_8\to a_7\to a_6)\to (a_8\to a_7)\to a_9\to a_6\doteq \alpha,a_3\to a_4\doteq a_2,a_9\to a_6\doteq a_4,a_8\to a_7\to a_6\doteq a_1, a_8\doteq a_9,a_3\doteq a_8\to a_7,a_5\doteq a_9\}$\\
	elim $\{(a_8\to a_7\to a_6)\to (a_8\to a_7)\to a_8\to a_6\doteq \alpha,a_3\to a_4\doteq a_2,a_9\to a_6\doteq a_4,a_8\to a_7\to a_6\doteq a_1, a_8\doteq a_9,a_3\doteq a_8\to a_7,a_5\doteq a_9\}$\\
	also $S= \lambda xyz.xz(yz):(a_8\to a_7\to a_6)\to (a_8\to a_7)\to a_8\to a_6$\\
	$K=true= \lambda xy.x:\alpha\to\beta\to\alpha$\\
	$S(true)\to^*_\beta \lambda yz.z:b\to a\to a = false$\\
	allerdings gilt für SK selbst:\\
	\AxiomC{Umbennenung von oben}
	\UnaryInfC{$K:a_8\to a_7\to a_8$}
	\AxiomC{$a_8=a_6$ setzen, um mit K zu unifizieren}
	\UnaryInfC{$\vdash S: (a_8\to a_7\to a_8)\to (a_8\to a_7)\to a_8\to a_8$}
	\BinaryInfC{$\vdash SK: (a_8\to a_7)\to a_8\to a_8$}
	\DisplayProof\\
	D.h. $SK$ ist mit $(a_8\to a_7)\to a_8\to a_8$ unifizierbar, aber nicht nur mit $b\to a\to a$, das für das Beta-redukt notwendig wäre. Somit gilt die abgeschwächte subjektexpansion nicht.\\
	\section{4}
	1.\\
	Nein:\\
	\[\underbrace{((a\to a)\to a\to a)}_{1.Eingabe}\underbrace{\to a}_{2.Eingabe} \underbrace{\to a}_{ausgabe}\]
	Die erste eingabe verlangt also irgendeine Funktion f mit 2 Argumenten $\underbrace{(a\to a)}_{1.Argument}\underbrace{\to a}_{2.Argument}\to a$\\
	z.B. $\lambda fa.fa$.\\
	Sprich wir haben eine funktion und eine anwendung als Argumente.\\
	Daraus folgt, dass $fix\ f:a\to a$ als typ hat (man braucht noch ein a, dass man einsetzen kann, das $((a\to a)\to a\to a)$ ist f).\\
	Wenn es ein fix gibt, dass diese typisierung besitzt, dann muss, wenn man $fix\ f:(a\to a)$ in fix einsetzt immernoch eine gültige typisierung entstehen. Sei $\Gamma = \{f= ((a\to a)\to a\to a), fix=((a\to a)\to a\to a)\to a \to a\}$\\
	\\
	\AxiomC{}
	\UnaryInfC{$\Gamma\vdash f:((a\to a)\to a\to a)$}
	\AxiomC{}
	\UnaryInfC{$\Gamma\vdash fix:((a\to a)\to a\to a)\to a \to a$}
	\BinaryInfC{$\Gamma\vdash \underbrace{fix\ f:a\to a}_{erste\ typisierung}$}
	\AxiomC{}
	\UnaryInfC{$\Gamma\vdash f:(a\to a)\to (a\to a)$}
	\BinaryInfC{$\Gamma\vdash \underbrace{f(fix\ f):a\to a}_{zweite\ typisierung}$}
	\DisplayProof
	Man kann also beide (f (fix f) und fix f) gleich typisieren, also ist es kein widerspruch, somit existiert ein Term\\
	2.\\
	$\lceil n\rceil = \lambda fa.\underbrace{f\dots f}_{n}\ a$\\
	ja: Jedes f wird auf eine a/f kombination angewandt:\\
	$\underbrace{f (\underbrace{f ( \underbrace{f ( a )}_{erste\ Anwendung})}_{zweite\ Anwendung})}_{Dritte\ Anwendung}$.\\
	Damit dies funktioniert, muss f den gleichen Ein und Ausgabewert haben. Ebenso muss a den Eingabewert von f haben, damit die erste Anwendung funktioniert.\\
	Somit erhalten wir $\lambda fa.\underbrace{f\dots f}_{n}\ a: \underbrace{(t\to t)}_{=f}\underbrace{\to t}_{=a}\to t$\\
	3.\\
	Nein:\\
	Sei $n=2$:\\
	das erste K $\lambda xy.x$ muss im x den gleichen typen haben, wie a (sonst würde)$(\lambda fa.ffa)K = \lambda a. KKa $ bereits nicht typchecken, weil das erste Argument bereits nicht ausführbar wäre.\\
	Der Typ vom ersten Argument von K ist somit ``fixiert''.\\
	Das zweite Argument ist frei b:\\
	\AxiomC{}
	\UnaryInfC{$\{a:\beta\}\vdash K:\gamma$}
	\AxiomC{}
	\UnaryInfC{$\{a:\beta\}\vdash K:\gamma\to\beta\to a$}
	\BinaryInfC{$\{a:\beta\}\vdash KK:\beta\to a$}
	\AxiomC{}
	\UnaryInfC{$\{a:\beta\}\vdash a:\beta$}
	\BinaryInfC{$\{a:\beta\}\vdash KKa:a$}
	\UnaryInfC{$\vdash \lambda a. KKa:\beta\to a$}
	\DisplayProof\\
	Die Beiden Typen, die K zugewiesen werden $\gamma\doteq \gamma \to\beta\to a$ sind durch occurs nicht unifizierbar. Somit gibt es keine solche typisierung.\\
	Dies entsteht dadurch, dass jedes folgende K die übrigen argumente vom vorherigen K ``mitnehmen'' muss.\\
	Da jedes argument jedoch eine Funktion ist die mehr und mehr argumente Erhält, kann es keine Typisierung im einfach getypten lambda-kalkül geben.\\
	\section{5}
	1.\\
	\begin{itemize}
		\item Cons True (Cons True Nil): boolean
		\item Cons True (Cons 35 Nil): Nicht typisierbar, da a und listentyp List a von Cons den gleichen Typ a besitzen müssen (währe hier Int/bool)
		\item Cons True: List Bool $\to$ List Bool, da als zweites argument noch ein List Bool zum anhängen erwartet wird, um die ``echte'' List zu erhalten
		\item Cons Nil ( Cons (Cons 35 Nil) Nil): List (List Int), typchecked: hier haben die zwei äußere Nil den Typen Nil:List(List a) und das innere Nil bei der 35 den typ Nil:List Int.
		\item Cons Nil (Cons 35 Nil): List (List Int), gleicher grund, wie beim vorletzten: Typ des inneren Nil ist Int, Typ des äußeren Nil ist List Int.
	\end{itemize}
	2.\\
	\begin{minted}{haskell}
	length::List a-> Nat
	length Nil = 0
	length (Cons x y) = 1+ length y
	snoc::List a-> a->List a
	snoc Nil x = Cons x Nil
	snoc (Cons x y) z = Cons x (snoc y z)
	reverse::List a-> List a
	reverse (Cons x y)= snoc (reverse y) x
	reverse Nil = Nil
	drop::a->List a-> List a
	drop x Nil = Nil
	drop x (Cons _ y) = if x==z then drop x y else Cons z (drop x y)
	elem::a->List a->Bool
	elem x Nil =False
	elem x (Cons z y) = if x==z then True else elem x y
	maximum::List Nat->Nat
	maximum Nil = 0
	maximum (Cons x y) = if maximum y> x then maximum y else x
	\end{minted}

\end{document}