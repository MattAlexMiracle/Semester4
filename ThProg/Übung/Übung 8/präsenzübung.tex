\documentclass{article}
\usepackage{listings}
\usepackage{mathrsfs}
\usepackage[utf8]{inputenc}
\usepackage{amssymb}
\usepackage{lipsum}
\usepackage{amsmath}
\usepackage{fancyhdr}
\usepackage{geometry}
\usepackage{scrextend}
\usepackage[english,german]{babel}
\usepackage{titling}
\setlength{\droptitle}{-3cm}
\usepackage{tikz}
\usepackage{algorithm,algpseudocode}
\usepackage[doublespacing]{setspace}
\usetikzlibrary{datavisualization}
\usetikzlibrary{datavisualization.formats.functions}
\usepackage{polynom}
\usepackage{amsmath}
\usepackage{gauss}
\usepackage{tkz-euclide}
\usepackage{minted}
\usetikzlibrary{datavisualization}
\usetikzlibrary{datavisualization.formats.functions}
\author{
Alexander Mattick Kennung: qi69dube\\
Kapitel 1
}
\usepackage{import}
\date{\today}
\geometry{a4paper, margin=2cm}
\usepackage{stackengine}
\parskip 1em
\newcommand\stackequal[2]{%
  \mathrel{\stackunder[2pt]{\stackon[4pt]{=}{$\scriptscriptstyle#1$}}{%
  $\scriptscriptstyle#2$}}
 }
\makeatletter
\renewcommand*\env@matrix[1][*\c@MaxMatrixCols c]{%
  \hskip -\arraycolsep
  \let\@ifnextchar\new@ifnextchar
  \array{#1}}
\makeatother
\lstset{
  language=haskell,
}
\lstnewenvironment{code}{\lstset{language=Haskell,basicstyle=\small}}{}
\usepackage{enumitem}
\setlist{nosep}
\usepackage{titlesec}
\usepackage{ stmaryrd }
\usepackage{verbatim}
\usepackage{tikz-qtree}
\usepackage{bussproofs}

\titlespacing*{\subsection}{0pt}{2pt}{3pt}
\titlespacing*{\section}{0pt}{0pt}{5pt}
\titlespacing*{\subsubsection}{0pt}{1pt}{2pt}
\title{Übung 7}


\begin{document}
	\maketitle
	\[\tau,\sigma::=a|b|\tau\to \sigma|\tau\times \sigma\]
	\AxiomC{$\Gamma\vdash t:\tau\times \sigma$}
	\LeftLabel{$\times e_1$}
	\UnaryInfC{$\Gamma\vdash fst\ t:\tau$}
	\DisplayProof\\
	\AxiomC{$\Gamma\vdash t:\tau\times \sigma$}
	\LeftLabel{$\times e_2$}
	\UnaryInfC{$\Gamma\vdash snd\ t:\sigma$}
	\DisplayProof\\
	\AxiomC{$\Gamma\vdash t:\tau$}
	\AxiomC{$\Gamma\vdash s:\sigma$}
	\LeftLabel{$\times i$}
	\BinaryInfC{$\Gamma\vdash \{t,s\}:\tau\times\sigma$}
	\DisplayProof\\
	(zum vergleich) $\land$-Regeln.\\
	\AxiomC{$\Gamma\vdash \tau\land \sigma$}
	\LeftLabel{$\land e_1$}
	\UnaryInfC{$\Gamma\vdash \tau$}
	\DisplayProof\\
	\AxiomC{$\tau\land \sigma$}
	\LeftLabel{$\land e_2$}
	\UnaryInfC{$\Gamma\vdash \sigma$}
	\DisplayProof\\
	\AxiomC{$\Gamma\vdash \theta$}
	\AxiomC{$\Gamma\vdash \sigma$}
	\LeftLabel{$\land i$}
	\BinaryInfC{$\Gamma\vdash \sigma\land \theta$}
	\DisplayProof\\
	Regeln:\\
	$fst \{t,s\}\to t$\\
	$snd \{t,s\}\to s$\\
	Beweis ``$\impliedby$'':\\
	Also es wird angenommen, dass $\bar\Phi$ inhabited, d.h. wir haben t und Beweis für $\vdash t:\bar\Phi$\\
	Wir streichen den Term und ersetzen alle $\times$ durch $\land$.\\
	``$\implies$'' Es gelte $\vdash\Phi$ (also im logischen gültig).\\
	Lösung: Induktion über Herleitung. (F.U. über die zuletzt angewandte Lösung)\\
	Zu geg. Menge an Annahmen.:	$\Gamma=\{\phi_0,\dots,\phi_n\}$ konstruiere Typkonext ( $\bar\Gamma=\{x_0:\bar\phi_0,\dots, x_n:\bar\phi_n\}$)\\
	Die Fälle $\to_i,\to_e,(Ax)$ bleiben gleich (vgl. Vorlesung)\\
	letzte Regel war $(\land-I)$ d.g. letzter Schritt war\\
	\AxiomC{$\Gamma\vdash \theta$}
	\AxiomC{$\Gamma\vdash \sigma$}
	\LeftLabel{$\land i$}
	\BinaryInfC{$\Gamma\vdash \sigma\land \theta$}
	\DisplayProof also aus Vorraussetzungen haben wir im Kontext $\bar \Gamma\vdash x_i:\sigma,\bar \Gamma\vdash x_j:\theta$ durch Anwendung von ($\times i$) gilt $\{x_i,x_j\}:\bar\sigma\times\bar\theta$\\
	$(\land e_1)$ per I.V.  gibt es im Kontext ein $\Gamma\vdash \{x_1,x_2\}:\phi\times\psi$\\
	Darauf kann man jetzt $(\times e_1)$ anwenden, und erhält $x_1:\phi$.\\
	Ebenso mit $(\land E_2)$ analog.\\
	Beweis: Es reicht einen Term anzugeben, der diesen Typ hat\\
	$\lambda xy. \{\{y, fst\ x\},snd\ x\} $\\
	\AxiomC{}
	\UnaryInfC{$x:p\times q, y:r\vdash y:r$}
	\AxiomC{}
	\UnaryInfC{$x:p\times q, y:r\vdash fst\ x:p$}
	\BinaryInfC{$x:p\times q, y:r\vdash \{y,fst\ x\}:(r\times p)$}
	\AxiomC{}
	\UnaryInfC{$x:p\times q, y:r\vdash snd\ x:q$}
	\BinaryInfC{$x:p\times q, y:r\vdash \{\{y, fst\ x\},snd\ x\}:(r\times p)\times q$}
	\UnaryInfC{$x:p\times q\vdash \lambda y. \{\{y, fst\ x\},snd\ x\}:r\to (r\times p)\times q$}
	\UnaryInfC{$\vdash \lambda xy. \{\{y, fst\ x\},snd\ x\}:p\times q\to r\to (r\times p)\times q$}
	\DisplayProof\\
	$sum(Ncons 4(Ncons 89 (Ncons 21 NNil))) \to_\delta 4+sum(Ncons 89 (Ncons 21 NNil))\to_\delta 4+ 89 sum(Ncons 21 NNil))\to_\delta 4+89+21+sum(NNil)\to_\delta 4+89+21+0 \to_\delta 4+110\to_\delta 114$\\
	\begin{minted}{haskell}
	element::Nat->NatList->Bool
	element a NNil = False
	element a (NCons y xs) = y==a or (element a y)
	\end{minted}
\end{document}