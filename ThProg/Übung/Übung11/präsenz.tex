\documentclass{article}
\usepackage{listings}
\usepackage{mathrsfs}
\usepackage{cancel}
\usepackage[utf8]{inputenc}
\usepackage{amssymb}
\usepackage{lipsum}
\usepackage{framed}
\usepackage{fancyhdr}
\usepackage{geometry}
\usepackage{scrextend}
\usepackage[english,german]{babel}
\usepackage{titling}
\usepackage{bm}
\usepackage{verbatim}
\usepackage{fourier}
\setlength{\droptitle}{-3cm}
\usepackage{tikz}
\usepackage{algorithm,algpseudocode}
\usepackage[doublespacing]{setspace}
\usepackage{minted}
\usetikzlibrary{datavisualization}
\usetikzlibrary{datavisualization.formats.functions}
\usepackage{polynom}
\usepackage{amsmath,amsthm}
\usepackage{gauss}
\usepackage{euscript}
\usepackage{tkz-euclide}
\usepackage{stackengine}
\usepackage{bussproofs}
\usepackage{tikz-cd}

\usetikzlibrary{datavisualization}
\usetikzlibrary{datavisualization.formats.functions}
\title{Übungsblatt 5}
\author{
Alexander Mattick Kennung: qi69dube\\
Kapitel 1
}
\usepackage{import}
\date{\today}
\geometry{a4paper, margin=2cm}
\usepackage{stackengine}
\parskip 1em
\newcommand\stackequal[2]{%
  \mathrel{\stackunder[2pt]{\stackon[4pt]{=}{$\scriptscriptstyle#1$}}{%
  $\scriptscriptstyle#2$}}
 }
\makeatletter
\renewcommand*\env@matrix[1][*\c@MaxMatrixCols c]{%
  \hskip -\arraycolsep
  \let\@ifnextchar\new@ifnextchar
  \array{#1}}
\makeatother
\lstset{
  language=haskell,
}
\lstnewenvironment{code}{\lstset{language=Haskell,basicstyle=\small}}{}
\usepackage{enumitem}
\setlist[itemize]{noitemsep, topsep=0pt}
\usepackage{titlesec}
\newcommand{\nto}{\nrightarrow}
\newcommand{\smallAscr}{\scriptscriptstyle\mathcal{A}}
%\newcommand{\nsqsubseteq}{\xout{\sqsubseteq}}
\title{Vorlesung 2}
\titlespacing*{\subsection}{0pt}{2pt}{3pt}
\titlespacing*{\section}{0pt}{0pt}{5pt}
\titlespacing*{\subsubsection}{0pt}{1pt}{2pt}
\newtheorem{satz}{Satz}
\newtheorem{korrolar}{Korrolar}[section]
\newtheorem{lemma}{Lemma}[section]

\theoremstyle{definition}
\newtheorem{beweis}{Beweis}[section]
\newtheorem{beispiel}{Beispiel}[section]
\newtheorem{definition}{Definition}[section]


\begin{document}
	\maketitle
	sprite( advance (advance (advance (loop [s1,s2,s3,s4,s5,s6]))))=\\
	= sprite (advance advance loop [s2,s3,s4,s5,s6,s1])=\\
	= sprite (loop [s4,s5,s6,s1,s2,s3])=\\
	= s4.\\
	\begin{minted}{haskell}
	sprite (delay a)  = sprite a
	advance (delay a) = a
	sprite (halfspeed a) = sprite a
	advance (halfspeed a) = delayed a
		where
			sprite (delayed a)  = sprite a
			advance (delayed a) = halfspeed (advance a)
	sprite (doublespeed_e a) = sprite a
	advance (doublespeed_e a) = doublespeed_e (advance (advance a))
	sprite (doublespeed_o a) =sprite  (advance a)
	advance (doublespeed_o a) = doublespeed_o (advance (advance (delay a)))
	\end{minted}
	1.\\
	$R\subseteq Animation\times Animation$ mit $sRt\implies $\\
	i) sprite(s) = sprite(t)\\
	ii) advance(s)R advance(t)\\
	gilt R bisimulation $sRt\implies s=t$.\\
	2 a)\\
	$\forall a.$ doublespeede (halfspeed a) =a\\
	$R=\{(doublespeede(halfspeed\ a), a)|a\in Animation\}$\\
	Zeige R Bisimulation, (doublespeede (halfspeed a)) R a\\
	i)  sprite(doublespeede (halfspeed a)) = sprite(halfspeed a)  =sprite(a)\\
	ii)	advance(doublespeede (halfspeed a)) = doublespeed(advance (advance (halfspeed a))) =\\
	doublespeed(advance (delayed a)) = doublespeed(halfspeed (advance a)) R (advance a)\\
	$\forall a$ doublespeede a = doublespeedo(delayed a)\\
	$R=\{(doublespeede\ a,doublespeedo(delayed\ a))|\forall a\in Animation\}$\\
	i)  sprite(doublespeede a) = sprite(a) = sprite(delayed a) = sprite(doublespeedo(delayed a))\\
	ii) advance(doublespeede a) = doublespeede( advance(advance(a)))\\
	advance(doublespeedo(delayed a)) = doublespeedo(advance(advance(delay a))=doublespeedo(advance a)\\
	R' = $R\cup \{(doublespeede(advance(advance\ a)), doublespeedo(advance\ a))|a\in Animation\}$\\
	die vorher bewiesenen gelten jetzt (advance per definition).\\
	i) sprite(doublespeede(advance(advance a))) =sprite( advance(advance a))\\
	sprite(doublespeedo(advance a)) = sprite(advance(advance a)) $\checkmark$\\
	ii) advance(doublespeede (advance (advance a))) =doublespeede (advance(advance(advance(advance a))))\\
	advance(doublespeedo(advance (a))) = doublespeedo(advance(advance(advance(a))))\\
	Diese stehen per oben in relation (advance(advance(a)) ist auch ein element aus Anim).\\
	R' ist eine Bisimulation.\\
	


\end{document}



