\documentclass{article}
\usepackage{listings}
\usepackage{mathrsfs}
\usepackage[utf8]{inputenc}
\usepackage{amssymb}
\usepackage{lipsum}
\usepackage{amsmath}
\usepackage{fancyhdr}
\usepackage{geometry}
\usepackage{scrextend}
\usepackage[english,german]{babel}
\usepackage{titling}
\usepackage{verbatim}
\setlength{\droptitle}{-3cm}
\usepackage{tikz}
\usepackage{algorithm,algpseudocode}
\usepackage[doublespacing]{setspace}
\usetikzlibrary{datavisualization}
\usetikzlibrary{datavisualization.formats.functions}
\usepackage{polynom}
\usepackage{amsmath}
\usepackage{gauss}
\usepackage{euscript}
\usepackage{tkz-euclide}
\usepackage{stackengine}
\usetikzlibrary{datavisualization}
\usetikzlibrary{datavisualization.formats.functions}
\title{Übungsblatt 5}
\author{
Alexander Mattick Kennung: qi69dube\\
Kapitel 1
}
\usepackage{import}
\date{\today}
\geometry{a4paper, margin=2cm}
\usepackage{stackengine}
\parskip 1em
\newcommand\stackequal[2]{%
  \mathrel{\stackunder[2pt]{\stackon[4pt]{=}{$\scriptscriptstyle#1$}}{%
  $\scriptscriptstyle#2$}}
 }
\makeatletter
\renewcommand*\env@matrix[1][*\c@MaxMatrixCols c]{%
  \hskip -\arraycolsep
  \let\@ifnextchar\new@ifnextchar
  \array{#1}}
\makeatother
\lstset{
  language=haskell,
}
\lstnewenvironment{code}{\lstset{language=Haskell,basicstyle=\small}}{}
\usepackage{enumitem}
\setlist{nosep}
\usepackage{titlesec}
\newcommand{\nto}{\nrightarrow}
\newcommand{\smallAscr}{\scriptscriptstyle\mathcal{A}}
\title{Präsenzübung 2}
\titlespacing*{\subsection}{0pt}{2pt}{3pt}
\titlespacing*{\section}{0pt}{0pt}{5pt}
\titlespacing*{\subsubsection}{0pt}{1pt}{2pt}
\newtheorem{satz}{Satz}
\newtheorem{korrolar}{Korrolar}[section]
\newtheorem{lemma}{Lemma}[section]
\newtheorem{beweis}{Beweis}[section]
\newtheorem{beispiel}{Beispiel}[section]
\newtheorem{definition}{Definition}[section]

\usepackage{ marvosym }
\begin{document}
	\maketitle\noindent
	\underline{Reduktionsordnung:}\\
	Ordnungsrelation $R\subseteq T_\Sigma(V)\times T_\Sigma(U)$ die wohlfundiert, stabil und kontextabgeschlossen ist.\\
	\underline{Stark normalisierend}:\\
	Wenn $\succ$ eine Reduktionsordnung ist und es gilt $\forall s,t(s\to_0 t\implies s\succ t)$, dann ist $\to_0$ SN.\\
	(montone) polynomielle interpretation $\mathscr{A}$ für $\Sigma$:\\
	- für jedes Funktionssymbol $f/n \in\Sigma$ ein $p_f\in\mathbb{N}[x_1,\dots,x_n]$\\
	- $A\subseteq \mathbb{N}$ die unter $p_f$ abgeschlossen ist.\\
	induzierte polynomordnung.\\
	$p_x = x$\\
	$p_{f(t_1,\dots,t_n)} p_f(p(t_1),\dots, p(t_n))$\\
	$s\succ_{\mathscr{A}} t \iff p_s >_{\mathscr{A}} p_t\iff p_f $\\
	\section{Übung 1}
	1. $<\mathbb{N}, p_\odot, p_\oplus>$ gilt nicht.\\
	$p((x\oplus y)\oplus z) = 2(2X+Y+1)+Z+1\to_0 2x+(2y+z+1)+1$\\
	Falsch für $x=y=z=0$\\
	$p_{(x\oplus y)\odot z} = p_{(x\odot y)\oplus z}= p_\odot(p_\plus(x,y), z) = (2x+y+1)Z$
	$p_{(x\odot y)\oplus (x\odot z)} = 2XY+XZ+1$ Falsch für $x\leq 1$\\
	2.\\
	Deshalb $\mathbb{N}_{\geq 2}$ als gültige Menge\\
	3. Für umgedrehte regel, auch umgedrehte Norm.\\
	$p_\oplus (X,Y) = X+2Y+1$\\
	$xy+2xz+X> xy+2xz+1$\\
	\section{Übung 2}
	HASKELL: als TES\\
	$\Sigma=\{0/0,s/1,+/2,d/1,q/1\}$\\
	$x+0\to_0 0$\\
	$x+s(y)\to_0 s(x+y)$\\
	verdopplung\\
	$d(0)\to_0 0$\\
	$d(s(x))\to_0 s(s(d(x)))$\\
	quadrierung\\
	$q(0)\to_0 0$\\
	$q(s(x))\to_0 q(x)+S(d x)$\\
	(also $(n+1)^2 = n^2+2n+1$)\\
	2.\\
	$p_s (x)=x+1$\\
	$p_0 =c_0>0$\\
	$p_+ = x+2y$\\
	$p_d(c_0) = c_1*c_0, c_1>2$\\
	$p_q(c_0) = c_1*c_0^2>c_0$\\
	$p_1(x+1)= c_2(x+1)^2 = c_2x^2+2c_2+c_2>p_q(x)+2(c_1*c_0+1)=p_q(x)+2c_1x+2 = c_2x^2+2c_1x+2$\\
	$c_2>c_1$ und $c_2>2$\\
	wähle also $c_2=4, c_1=3$
	für die polynomielle interpretation $<\mathbb{N}, p_+, p_d, p_q,p_0,p_s>$

\end{document}





