\documentclass{article}
\usepackage{listings}
\usepackage{mathrsfs}
\usepackage[utf8]{inputenc}
\usepackage{amssymb}
\usepackage{lipsum}
\usepackage{amsmath}
\usepackage{fancyhdr}
\usepackage{geometry}
\usepackage{scrextend}
\usepackage[english,german]{babel}
\usepackage{titling}
\usepackage{verbatim}
\setlength{\droptitle}{-3cm}
\usepackage{tikz}
\usepackage{algorithm,algpseudocode}
\usepackage[doublespacing]{setspace}
\usetikzlibrary{datavisualization}
\usetikzlibrary{datavisualization.formats.functions}
\usepackage{polynom}
\usepackage{amsmath}
\usepackage{gauss}
\usepackage{euscript}
\usepackage{tkz-euclide}
\usepackage{stackengine}
\usetikzlibrary{datavisualization}
\usetikzlibrary{datavisualization.formats.functions}
\title{Übungsblatt 5}
\author{
Alexander Mattick Kennung: qi69dube\\
Kapitel 1
}
\usepackage{import}
\date{\today}
\geometry{a4paper, margin=2cm}
\usepackage{stackengine}
\parskip 1em
\newcommand\stackequal[2]{%
  \mathrel{\stackunder[2pt]{\stackon[4pt]{=}{$\scriptscriptstyle#1$}}{%
  $\scriptscriptstyle#2$}}
 }
\makeatletter
\renewcommand*\env@matrix[1][*\c@MaxMatrixCols c]{%
  \hskip -\arraycolsep
  \let\@ifnextchar\new@ifnextchar
  \array{#1}}
\makeatother
\lstset{
  language=haskell,
}
\lstnewenvironment{code}{\lstset{language=Haskell,basicstyle=\small}}{}
\usepackage{enumitem}
\setlist{nosep}
\usepackage{titlesec}
\newcommand{\nto}{\nrightarrow}
\newcommand{\smallAscr}{\scriptscriptstyle\mathcal{A}}
\title{Vorlesung 2}
\titlespacing*{\subsection}{0pt}{2pt}{3pt}
\titlespacing*{\section}{0pt}{0pt}{5pt}
\titlespacing*{\subsubsection}{0pt}{1pt}{2pt}
\newtheorem{satz}{Satz}
\newtheorem{korrolar}{Korrolar}[section]
\newtheorem{lemma}{Lemma}[section]
\newtheorem{beweis}{Beweis}[section]
\newtheorem{beispiel}{Beispiel}[section]
\newtheorem{definition}{Definition}[section]
\setcounter{section}{2}
\usepackage{ marvosym }
\begin{document}
	\maketitle\noindent
	\section{Übung 3}
	$\oslash = y+x^2$\\
	$\infty=2$\\
	\Heart=1\\
	\begin{itemize}
	\item Regel 6:$ x+(y+z)^2 = x+y^2+2zy+z^2 >x+z^2+y^2$, passt
	\item Regel 5: $2+x^2+y^2 > (x+y^2)+1^2$, passt
	\item Regel 4: $2+(x+2^2)^2 = 2+x^2+8x+16 > x+x^2$, passt
	\item Sei $\mathcal{A} = \mathbb{N}_{\geq1}$ (wegen Regel 6)
	\end{itemize}

	\section{Übung 4}
	$p_{\lor} = x+y^2$\\
	$p_{\land} = (x+1)^2+(y+1)^2+1+1$\\
	$p_{\implies} = (x+1)+y^2+1$\\
	$p_{\forall} = x+(y+1)+1+1$\\
	$p_{\exists} = x+y$\\
	\begin{itemize}
		\item Regel 1: $x+1+1> x$, passt
		\item Regel 2: $x+1+(y+1)^2+1+1 > x+1+(y+1)^2+1$, passt
		\item Regel 3: $(x+1)^2+(y+1)^2+1+1+1=x^2+2x+y^2+2y+5 > x+1+(y+1)^2$, passt
		\item Regel 4: $(x+1)+y^2+1> (x+1)+y^2$, passt
		\item Regel 5: $x+(y+z^2)^2 = x+y^2+2yz^2+z^4> (x+y^2)+z^2$, passt
		\item Regel 6: $x+(y+1)+1+1 > (x+(y+1))+1$, passt	
	\end{itemize}
	mit $\mathcal{A} = \mathbb{N}_{\geq 1}$ wegen Regel 5
	Das system ist SN.\\
	\section{Übung 5} 
	$p_a(x) = x^2+1$\\
	$p_b(x) = x^3$\\
	$p_c = x^2$\\
	\begin{itemize}
		\item Regel 7: $(x^3)^2+1 = x^6+1>x$
		\item Regel 8: $(x^2+1)^2 = x^4+2x^2+1^2 > (x^2)^2+1 = x^4+1$
		\item Regel 9: $(x^2)^2=x^4>x^3$
		\item Regel 10: $(x^2+1)^3= x^6+3x^4+3x^2+1>x^2+1$
	\end{itemize}
	Wobei $\mathcal{A} = \mathbb{N}_{\geq 2}$ (wegen Regel 9)

\end{document}





