\documentclass{article}
\usepackage{listings}
\usepackage{mathrsfs}
\usepackage[utf8]{inputenc}
\usepackage{amssymb}
\usepackage{lipsum}
\usepackage{amsmath}
\usepackage{fancyhdr}
\usepackage{geometry}
\usepackage{scrextend}
\usepackage[english,german]{babel}
\usepackage{titling}
\setlength{\droptitle}{-3cm}
\usepackage{tikz}
\usepackage{algorithm,algpseudocode}
\usepackage[doublespacing]{setspace}
\usetikzlibrary{datavisualization}
\usetikzlibrary{datavisualization.formats.functions}
\usepackage{polynom}
\usepackage{amsmath}
\usepackage{verbatim}
\usepackage{gauss}
\usepackage{tkz-euclide}
\usetikzlibrary{datavisualization}
\usetikzlibrary{datavisualization.formats.functions}
\author{
Alexander Mattick Kennung: qi69dube\\
Kapitel 1
}
\usepackage{import}
\date{\today}
\geometry{a4paper, margin=2cm}
\usepackage{stackengine}
\parskip 1em
\newcommand\stackequal[2]{%
  \mathrel{\stackunder[2pt]{\stackon[4pt]{=}{$\scriptscriptstyle#1$}}{%
  $\scriptscriptstyle#2$}}
 }
\makeatletter
\renewcommand*\env@matrix[1][*\c@MaxMatrixCols c]{%
  \hskip -\arraycolsep
  \let\@ifnextchar\new@ifnextchar
  \array{#1}}
\makeatother
\lstset{
  language=haskell,
}
\lstnewenvironment{code}{\lstset{language=Haskell,basicstyle=\small}}{}
\usepackage{enumitem}
\setlist{nosep}
\usepackage{titlesec}

\titlespacing*{\subsection}{0pt}{2pt}{3pt}
\titlespacing*{\section}{0pt}{0pt}{5pt}
\titlespacing*{\subsubsection}{0pt}{1pt}{2pt}
\title{Vorlesung 4}


\begin{document}
	\maketitle
	\section{4}
	\subsubsection{}
	$\lceil 2\rceil = \lambda fa. ffa$\\
	$\lceil 1\rceil = \lambda fa. fa$\\
	$pred\ n= \lambda fa.n(\lambda gh.h(gf))(\lambda u.a)(\lambda u.u)$\\
	$pred\ \lambda f_0a_0. f_0f_0a_0= \lambda fa.(\lambda f_0 a_0. f_0f_0a_0)(\lambda gh.h(gf))(\lambda u.a)(\lambda u.u) \to_\beta \lambda fa.(\lambda a_0. (\lambda gh.h(gf)(\lambda gh.h(gf)a_0))(\lambda u.a)(\lambda u.u) \to_\beta \lambda fa.((\lambda gh.h(gf)(\lambda gh.h(gf)(\lambda u.a)))(\lambda u.u)\to_\beta
	\lambda fa.((\lambda h_0.h_0((\lambda gh.h(gf)f))(\lambda u.a))) (\lambda u.u)\to_\beta
	\lambda fa.((\lambda h_0.h_0((\lambda h.h((\lambda u.a)ff))))) (\lambda u.u)\to_\beta
	\lambda fa.(\lambda h_0.h_0\lambda h.h((\lambda u.a)ff)) (\lambda u.u)\to_\beta
	\lambda fa.(f\lambda h.h((\lambda u.a)f)) (\lambda u.u)\to_\beta
	\lambda fa.f\lambda h.h(\lambda u.a)f (\lambda u.u)\to_\beta
	\lambda fa.f\lambda h.ha (\lambda u.u)\to_\beta
	\lambda fa.f(\lambda u.u)a \to_\beta
	\lambda fa.fa \to_\delta \lceil 1\rceil$\\
	\subsection{}
	sub n m = $m (pred\ n)$\\
	Dazu definiert $true = \lambda xy.x$ $False = \lambda xy.y$\\
	Hilfskonstrukt ``istNull'' = $\lambda n.n\lambda x.(false) (true)$\\
	wenn man 0 einsetzt erhält man $(\lambda n.n\lambda x.(false) (true)) \lambda f a.a\to \lambda f a.a \lambda x.(false) (true) \to\lambda a.a (true)\to true$.\\
	Wenn man einen wert ungleich null einsetzt erhält man:\\
	$(\lambda n.n(false) (true)) \lambda f a.\underbrace{f}_{n mal} a \to (\lambda f a.\underbrace{f}_{n mal} a) (false) (true) \to (\lambda a.\underbrace{\lambda x.false}_{n mal} a) (true) \to \underbrace{false}_{n mal} (true)$\\
	Durch das $\lambda x.false$ werden alle nachfolgenden zeichen ignoriert $\lambda x.false (t)\to false$ somit ``kollabiert'' die ``false, false,\dots, true'' Kette von links nach rechts zu nur einem false.\\
	Zweites Hilfskonstrukt: ``und'' $ \lambda nm. n m (false)$, wenn n true ist, dann wird immer der wert von m gewählt und das letzte false verfällt. $\lambda xy.x m (false)\to \lambda y.m (false)\to m$\\
	Wenn n false ist, dann ist egal, was m ist, das Ergebnis ist immer false $\lambda xy.y m (false)\to \lambda y.y (false)\to (false)$
	le n m= istNull (sub n m)\\
	eq n m= und (istNull (sub n m)) (istNull (sub m n))
	für lt lohnt es sich ein ``Nicht'' zu definieren:\\
	Nicht = $\lambda x.x (false) (true) $ \\
	Wenn man $true$ einsetzt: $(\lambda x. x (false) (true)) (true)\to (true) (false) (true) \to \lambda xy.x (false) (true)\to \lambda y.(false) (true)\to (false)$\\
	Wenn man $false$ einsetzt: $(\lambda x. x (false) (true)) (false)\to (false) (false) (true) \to \lambda xy.y (false) (true) \to \lambda y.y (true) \to (true)$\\
	daraus folgt für lt:\\
	lt n m= und (le n m) (nicht (eq n m))
	3.\\
	Zuerst wendet pred auf den eingesetzen wert $\lambda gh.h(gf)$ an, sodass diese funktion n-mal ineinander geschachtelt wird.\\
	Diese funktion hat die Eigenschaft, dass sie eine zweite funktion nimmt, und diese auf f anwendet, bevor diese wiederolt angewandt wird.\\
	Das zweite, was an unser n angwandt wird (das ``a'' in $\lambda fa.f\dots f a$) ist eine konstante, die, egal was auf sie angewandt wird, immer a zurückliefert.\\
	Dies führt dazu, dass das erste f in unserem n-hohen f-Turm zu einem a reduziert wird, da ``const a'' auf f angewandt wird $\lambda h. h(``const a'' f)\to \lambda h.ha$\\
	jetzt muss noch irgendwie der zweite parameter aufelöst werden, was mit der anwendung der identitätsfunktion $\lambda u.u$ bewerkstelligt wird.\\
	Dadurch dass das h an erste stelle steht, wird die identitätsfunktion von $\lambda gh.h(gf)$ den gesamten turm hinaufgegeben.\\
	Bei allen anwendungen außer der ersten, kann f einfach an konkateniert werden, da g keine konstante ist.\\
	am Ende kann die identität entfernt werden, und der vorgänger steht da, da die konstante im ersten schritt schließlich ein f entfernt hat.\\
\end{document}