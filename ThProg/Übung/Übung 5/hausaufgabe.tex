\documentclass{article}
\usepackage{listings}
\usepackage{mathrsfs}
\usepackage[utf8]{inputenc}
\usepackage{amssymb}
\usepackage{lipsum}
\usepackage{amsmath}
\usepackage{fancyhdr}
\usepackage{geometry}
\usepackage{scrextend}
\usepackage[english,german]{babel}
\usepackage{titling}
\setlength{\droptitle}{-3cm}
\usepackage{tikz}
\usepackage{algorithm,algpseudocode}
\usepackage[doublespacing]{setspace}
\usetikzlibrary{datavisualization}
\usetikzlibrary{datavisualization.formats.functions}
\usepackage{polynom}
\usepackage{amsmath}
\usepackage{gauss}
\usepackage{tkz-euclide}
\usetikzlibrary{datavisualization}
\usetikzlibrary{datavisualization.formats.functions}
\author{
Alexander Mattick Kennung: qi69dube\\
Kapitel 1
}
\usepackage{import}
\date{\today}
\geometry{a4paper, margin=2cm}
\usepackage{stackengine}
\parskip 1em
\newcommand\stackequal[2]{%
  \mathrel{\stackunder[2pt]{\stackon[4pt]{=}{$\scriptscriptstyle#1$}}{%
  $\scriptscriptstyle#2$}}
 }
\makeatletter
\renewcommand*\env@matrix[1][*\c@MaxMatrixCols c]{%
  \hskip -\arraycolsep
  \let\@ifnextchar\new@ifnextchar
  \array{#1}}
\makeatother
\lstset{
  language=haskell,
}
\lstnewenvironment{code}{\lstset{language=Haskell,basicstyle=\small}}{}
\usepackage{enumitem}
\setlist{nosep}
\usepackage{titlesec}

\titlespacing*{\subsection}{0pt}{2pt}{3pt}
\titlespacing*{\section}{0pt}{0pt}{5pt}
\titlespacing*{\subsubsection}{0pt}{1pt}{2pt}
\title{Vorlesung 4}


\begin{document}
	\maketitle
	\section{$\alpha$-Äquivalenz und $\beta$-Reduktion}
	\subsubsection{}
	a) Wahr, umbenennungen dürfen keine neuen Variablen einfangen und zwei variablen dürfen nicht überlappen.\\
	$\sigma = [y/f, f/x, z/y,x/z]$ alle substitutionen passieren zur gleichen Zeit.\\
	b) bei gefangenen Variablen, darf der Name bei einsetzen nicht zu Einfang weiterer Variablen führen:\\
	$\sigma = [x/x,y/y]$\\
	$(\lambda xy(\lambda x.xx)yx ) = (\lambda xy((\lambda x.xx)yx\sigma') )$ wobei $\sigma'=\sigma$ ist.\\
	$=(\lambda xy(((\lambda x.xx)\sigma')(y\sigma')(x\sigma'))) = (\lambda xy(((\lambda x.xx)\sigma')(y)(x))) = \lambda xy((\lambda y.(xx\sigma''))(y)(x)) = \lambda xy((\lambda y.((x\sigma'')(x\sigma''))(y)(x)) = \lambda xy((\lambda y.yy)(y)(x))$ wobei $\sigma'' = [x\mapsto y]$\\
	c) geht nicht, das x im zweiten lambda Term wird vom y eingefangen.\\
	\subsubsection{}
	a) nein, anwendung von außen nach innen:\\
	$(\lambda fgh.fhg)(vv)\to_\beta \lambda gh.(vv)hg$\\
	b) ist richtig. Es ist zu beachten, dass man hier das vom $\lambda$ gebundene äußere x umbennenen muss, da dieses sonst in den inneren Termen gefangen wird.\\
	c) ist richtig, da das y zu u umbennant wurde. Es ist außerdem zu beachten, dass das $(\lambda u.\dots)uv$ geklammert ist, und somit das äußere u nicht einfängt.\\
	\subsubsection{}
	a) $(\lambda fxy.f(fy)(xx))(\lambda uv.u)\stackrel{\text{umbennenung von} \lambda x.\dots}{\to_\beta} (\lambda zy.(xx)((xx)y))(\lambda uv.u)\to_\beta \lambda y.(xx)((xx)y)$.\\
	$\beta-$Reduktion ersetzt nur gebundene Variablen, also ist hier ende.\\
	b)$$(\lambda fxg.g((\lambda y.fyx)(gx))) (\lambda xz.gx)(gz)(\lambda x.x)\to_\beta $$
	$$\lambda xg_0.g_0(\lambda y.\lambda x_0z.gx_0yx(g_0x)) (gz)(\lambda x.x)\to_\beta $$
	$$\lambda g_0.g_0(\lambda y.\lambda x_0z.gx_0y(gz)(g_0(gz))) (\lambda x.x)\to_\beta $$
	$$(\lambda x.x)(\lambda y.\lambda x_0z.gx_0y(gz)((\lambda x.x)(gz))) \to_\beta $$
	$$(\lambda y.\lambda x_0z.gx_0y(gz)((\lambda x.x)(gz))) \to_\beta $$
	$$\lambda y.\lambda x_0z.gx_0y(gz)(gz) \to_\beta $$
	$$\lambda x_0z.gx_0(gz)(gz) \to_\beta $$
	$$\lambda z.g(gz)(gz) \to_\beta $$
	$$g(gz)$$
	\section{Church-Kodierung}
	\subsection{}
	Beweis durch Umformung:\\
	$$case\ s\ t (inl\ u)\to_{\delta}$$
	$$(\lambda fgs.sfg)\ s\ t (inl\ u)\to_{\beta}$$
	$$(\lambda gs_0.s_0sg)\ t (inl\ u)\to_{\beta}$$
	$$(\lambda s_0.s_0st)\ (inl\ u)\to_{\beta}$$
	$$((inl\ u)st)\ \to_{\delta}$$
	$$((\lambda fg.fu)st)\ \to_{\beta}$$
	$$(\lambda g.su)t\ \to_{\beta}$$
	$$su$$
	Ähnlich für 
	$$case\ s\ t (inr\ u)\to_{\delta}$$
	$$(\lambda fgs.sfg)\ s\ t (inr\ u)\to_{\beta}$$
	$$(\lambda gs_0.s_0sg)\ t (inr\ u)\to_{\beta}$$
	$$(\lambda s_0.s_0st)\ (inr\ u)\to_{\beta}$$
	$$((inr\ u)st)\ \to_{\delta}$$
	$$((\lambda fg.gu)st)\ \to_{\beta}$$
	$$((\lambda g.gu)t)\ \to_{\beta}$$
	$$tu$$
	\subsection{}
	$case\ inl\ inr\ t\to_{\delta}$
	$$(\lambda fgs.sfg)\ inl\ inr \ t \to_{\beta}$$
	$$(\lambda gs_0.s_0(inl)g)\ inr \ t\to_{\beta}$$
	$$(\lambda s_0.s_0(inl)( inr))\  t\to_{\beta}$$
	$$t(inl)( inr)$$
	von hier aus beide varianten von t substitutieren.\\
	$$t=inl\ a$$
	$$(inl\ a)(inl)( inr)\to_{\delta}(\lambda fg.fa)(inl)( inr)\to_\beta (\lambda g.(inl)a)( inr)\to_\beta inl\ a=t$$
	$$t=inr\ b$$
	$$(inr\ b)(inl)( inr)\to_{\delta}(\lambda fg.gb)(inl)( inr)\to_\beta (\lambda g.gb)( inr)\to_\beta inr\ b=t$$
	Gegenbeispiel\\
	$$t=\lambda fg.a$$
	$$t(inl)( inr)\to_\delta (\lambda fg.a)(inl)( inr)\to_\beta \lambda g.a(inr)\to_\beta a\neq \lambda fg.a$$
	\section{$\eta$-Reduktion}
	$s=\lceil 0\rceil$ und $t=\lceil 1\rceil$ unterscheiden.\\
	wobei nach präsenzübung 3 gilt:\\
	$\lceil 0\rceil=\lambda fa.a$ und $\lceil 1\rceil=\lambda fa.fa$\\
	$u_1 = \lambda xyz.y$ $u_2 = \lambda xy.x$\\
	$\lambda fa.a u_1 u_2 x y \to_\beta \lambda a.a u_2 x y\to_\beta u_2xy\to_\delta (\lambda xy.x)xy\to_\beta (\lambda y.x)y\to_\beta x$.\\
	$\lambda fa.fa u_1 u_2 x y\to_\beta \lambda a.u_1 a u_2 xy\to_\beta u_1u_2 xy\to_\delta (\lambda xyz.z) u_2 xy\to_\beta (\lambda yz.z) xy\to_\beta (\lambda yz.z) x y\to_\beta \lambda z.zy\to_\beta y$.\\

	$p_c = x^2$\\
	$p_b = x^3+1$\
	$p_a = x^4+2$\\

	$(x^2+1)^3 +2 > x^2+1$\\
	$x^6+2 > (x^4+2)^2 = x^8+4x^4+4$\\
	$(x^2)^2= x^4 > x^3+1$\\
	$(x^4+2)^3+1 >x^3+1$\\
	$(x^3+1)^2 > $
\end{document}