\documentclass{article}
\usepackage{listings}
\usepackage{mathrsfs}
\usepackage[utf8]{inputenc}
\usepackage{amssymb}
\usepackage{lipsum}
\usepackage{amsmath}
\usepackage{fancyhdr}
\usepackage{geometry}
\usepackage{scrextend}
\usepackage[english,german]{babel}
\usepackage{titling}
\usepackage{verbatim}
\setlength{\droptitle}{-3cm}
\usepackage{tikz}
\usepackage{algorithm,algpseudocode}
\usepackage[doublespacing]{setspace}
\usetikzlibrary{datavisualization}
\usetikzlibrary{datavisualization.formats.functions}
\usepackage{polynom}
\usepackage{amsmath}
\usepackage{gauss}
\usepackage{euscript}
\usepackage{tkz-euclide}
\usepackage{stackengine}
\usetikzlibrary{datavisualization}
\usetikzlibrary{datavisualization.formats.functions}
\title{Übungsblatt 5}
\author{
Alexander Mattick Kennung: qi69dube\\
Kapitel 1
}
\usepackage{import}
\date{\today}
\geometry{a4paper, margin=2cm}
\usepackage{stackengine}
\parskip 1em
\newcommand\stackequal[2]{%
  \mathrel{\stackunder[2pt]{\stackon[4pt]{=}{$\scriptscriptstyle#1$}}{%
  $\scriptscriptstyle#2$}}
 }
\makeatletter
\renewcommand*\env@matrix[1][*\c@MaxMatrixCols c]{%
  \hskip -\arraycolsep
  \let\@ifnextchar\new@ifnextchar
  \array{#1}}
\makeatother
\lstset{
  language=haskell,
}
\lstnewenvironment{code}{\lstset{language=Haskell,basicstyle=\small}}{}
\usepackage{enumitem}
\setlist{nosep}
\usepackage{titlesec}
\newcommand{\nto}{\nrightarrow}
\title{Vorlesung 2}
\titlespacing*{\subsection}{0pt}{2pt}{3pt}
\titlespacing*{\section}{0pt}{0pt}{5pt}
\titlespacing*{\subsubsection}{0pt}{1pt}{2pt}



\begin{document}
	%aaron.strahlberger@posteo.de 
	\maketitle
	\section{Übung 1}
	\subsection{1}
	a) Involution $(R^-)^-$\\
	Beweis:\\
	$R^- = \{(y,x)|xRy\}$\\
	also $\{(x,y)|(y,x)\in \{(y,x)| xRy\}\}=\{(x,y)| xRy\}=R$\\
	b) Antiautomorphismus $(R\circ T)^- = T^-\circ R^-$\\
	Beweis:\\
	$\{(x,z)|\exists y( xTy\land yRz)\}^- = \{(z,x)|\exists y( xTy\land yRz)\} =\{(z,x)|\exists y(zR^-y\land yT^-z\} = T^-\circ R^-$\\
	c) Monotonie der Komposition $R\subseteq S\implies R\circ T\subseteq S\circ T$\\
	Beweis:\\
	Von oben nach unten und unten nach oben, in der mitte treffen.\\
	Annahme $R\subseteq S$.\\
	$z(R\circ T)y\\\iff \exists x (zTx\land xRy)\\
	\stackon{$\implies$}{$R\subseteq S$}\\
	\exists x (zTx\land xSy)\\
	\implies z(S\circ T)y$
	d) Monotonie des Inversen:\\
	$R\subseteq S\implies R^-\subseteq S^-$\\
	$yR^-x\iff xRy \stackon{$\implies$}{$R\subseteq S$} xSy\iff yS^-x$\\
	\subsection{2}
	gegenbeispiel für $R\circ S\neq S\circ R$\\
	$S=\{(1,2)\}, R=\{(2,1)\}$\\
	$S\circ R = (1,1)$\\
	$R\circ S = (2,2)$\\
	\subsection{Übung 2}
	1. $(1+(-y))\cdot x^{-1} = (1+(-(-1)))\cdot (z+x\cdot 1)^{-1}$\\
	$\stackon{$\to$}{decomp}\{(1+(-y))=(1+(-(-1))), x^{-1}=(z+x\cdot 1)^{-1}\}$\\
	$\stackon{$\to$}{decomp} \{(1+(-y))=(1+(-(-1))), x=(z+x\cdot 1)\}$\\
	$\stackon{$\to$}{occurs} \bot$\\
	2. $x\cdot (y^{-1}\cdot y)+(z+y) = x\cdot x +(0+0\cdot 1)$\\
	$\stackon{$\to$}{decomp} \{x\cdot (y^{-1}\cdot y)=x\cdot x, (z+y)=(0+0\cdot 1)\}$\\
	$\stackon{$\to$}{decomp} \{x=x,(y^{-1}\cdot y)=x,  (z+y)=(0+0\cdot 1)\}$\\
	$\stackon{$\to$}{orient} \{x=x,x=(y^{-1}\cdot y),  (z+y)=(0+0\cdot 1)\}$\\
	$\stackon{$\to$}{delete} \{x=(y^{-1}\cdot y),  (z+y)=(0+0\cdot 1)\}$\\
	$\stackon{$\to$}{decomp} \{x=(y^{-1}\cdot y),  z=0,y=0\cdot 1)\}$\\
	$\stackon{$\to$}{elim} \{x=((0\cdot 1)^{-1}\cdot 0\cdot 1),  z=0,y=0\cdot 1)\}$\\
	mgu = $[((0\cdot 1)^{-1}\cdot 0\cdot 1)/x,0/z,0\cdot 1/y]$\\
	3.\\
	$y+(x\cdot 1)= (z+0)^{-1} +y$\\
	$\stackon{$\to$}{decomp} \{y=(z+0)^{-1},(x\cdot 1)=y\}$\\
	$\stackon{$\to$}{elim}\{y=(z+0)^{-1},(x\cdot 1)=(z+0)^{-1}\}$\\
	$\stackon{$\to$}{conflict}\bot$
\end{document}