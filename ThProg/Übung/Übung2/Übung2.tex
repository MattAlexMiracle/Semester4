\documentclass{article}
\usepackage{listings}
\usepackage{mathrsfs}
\usepackage[utf8]{inputenc}
\usepackage{amssymb}
\usepackage{lipsum}
\usepackage{amsmath}
\usepackage{fancyhdr}
\usepackage{geometry}
\usepackage{scrextend}
\usepackage[english,german]{babel}
\usepackage{titling}
\setlength{\droptitle}{-3cm}
\usepackage{tikz}
\usepackage{algorithm,algpseudocode}
\usepackage[doublespacing]{setspace}
\usetikzlibrary{datavisualization}
\usetikzlibrary{datavisualization.formats.functions}
\usepackage{polynom}
\usepackage{amsmath}
\usepackage{gauss}
\usepackage{tkz-euclide}
\usetikzlibrary{datavisualization}
\usetikzlibrary{datavisualization.formats.functions}
\author{
Alexander Mattick Kennung: qi69dube\\
Kapitel 1
}
\usepackage{import}
\date{\today}
\geometry{a4paper, margin=2cm}
\usepackage{stackengine}
\parskip 1em
\newcommand{\nto}{\nrightarrow}
\newcommand\stackequal[2]{%
  \mathrel{\stackunder[2pt]{\stackon[4pt]{=}{$\scriptscriptstyle#1$}}{%
  $\scriptscriptstyle#2$}}
 }
\makeatletter
\renewcommand*\env@matrix[1][*\c@MaxMatrixCols c]{%
  \hskip -\arraycolsep
  \let\@ifnextchar\new@ifnextchar
  \array{#1}}
\makeatother
\lstset{
  language=haskell,
}
\lstnewenvironment{code}{\lstset{language=Haskell,basicstyle=\small}}{}
\usepackage{enumitem}
\setlist{nosep}
\usepackage{titlesec}

\titlespacing*{\subsection}{0pt}{2pt}{3pt}
\titlespacing*{\section}{0pt}{0pt}{5pt}
\titlespacing*{\subsubsection}{0pt}{1pt}{2pt}
\title{Vorlesung 4}


\begin{document}
	\maketitle
	1.\\
	Grammatik:\\
	Sei $e=c|d$
	$f(g(c,(\cdot)))$, $g(d,f((\cdot)))$, $g(f((\cdot)),c)$, $h(f((\cdot)),e)$,\\
	$h(f(e),(\cdot))$, $h(f(d),(\cdot))$, $h(f((\cdot)), d)$, $f(h((\cdot),c))$, $g((\cdot), f(e))$,
	$g(e, f((\cdot)))$.\\
	2.\\
	a)\\
	$h(f(d),d)\to g(d,f(d))\to h(f(d),d)\to\dots $\\
	b)\\
	$g(f(c),c)\to g(c,c)$\\
	$g(f(c),c)\to f(h(c,c))$\\
	c)\\
	Für eine endlosschleife muss eine Kombination aus Regeln wiederholt werden, da es nur endlich viele Regeln gibt.\\
	Eigenschaft 1:\\
	Man kann sehen, dass in keiner der Regeln f auf der linken Seite öfter steht, als auf der Rechten. \\
	Eigenschaft 2:\\
	Insbesondere Regel (9) ist die einzige Regel, die ein f entfernt.\\
	Für eine endlosschleife muss also gelten:\\
	\# f linke Seite $=$ \# f rechte Seite.\\
	Wenn auf der linken Seite mehr f sind, als auf der Rechten, muss Regel (9) angewandt worden sein. Dies kann aber aufgrund von Eigenschaft 1 nicht Teil einer Schleife sein. Hierbei wird ``Schleife'' als unendlicher zyklus von regelanwendungen gesehen, vor diesem endlichen Zyklus ist es möglich, dass (9) vorkommt.\\
	Eigenschaft 3:\\
	Eine Regel in einem Zyklus muss nach 4 schritten wiederholt werden, da es nur 4 verschiedene Regeln in einem Zyklus geben kann (Regel 9 fällt raus, vgl oben)\\
	(Dies bedeutet nicht, dass der Term auf den die Regel angewandt wird gleich ist)\\
	Wir betrachten alle Vereinbaren Kombinationen von 2 zyklus-Regeln:\\
	$(13)\circ (10): f(g(c,y))\to g(d,f(y))$\\
	$(13)\circ (11): g(d,f(y))\to g(y,f(d))$\\
	mit Substitution/Kontext erhält man außerdem:\\
	$(13) \circ (12): g(f(f(x)),c)\to f(g(x,f(c)))$\\
	$(11)\circ (13): h(f(d),y)\to h(f(y),d)$\\
	Eine Dreierkette ohne Substitution/Kontext existiert nicht.\\
	mit Substitution/Kontext gibt es:\\
	$(13)\circ (11)\circ(13): h(f(d),y)\to g(d,f(y))$\\
	$(10)\circ (13)\circ(12): g(f(f(c)), c)\to h(f(d),c)$\\
	$(11)\circ(13)\circ (11): g(d,f(d))\to h(f(d),d)$\\
	Die viererketten mit Substitution/Kontext:\\
	$(11)\circ(13)\circ (11)\circ(13): h(f(d),d)\to h(f(d),d)$\\
	$(13)\circ(11)\circ (13)\circ(11): g(d,f(d))\to g(d,f(d))$\\
	Die einzigen viererketten sind solche, die eine schleife ohne freie Variable beschreiben.\\
	Es gibt also keine möglichkeit aus dieser auszubrechen. (wenn z.B. $f(c)$ in einer dieser loops wäre, könnte man ausbrechen)\\
	Aufgrund von Kontextabgeschlossenheit, gibt es also keine Kette, die (c) erfüllt, selbst wenn man diese formeln in einen größeren Kontext $C(\cdot)$ einsetzt.\\
	$g(f(g(c,d)),c)\stackrel{10, C=g((\cdot),c),\sigma=[d/y]}{\to} g(h(f(d),d), c) \to\dots\text{wie bei a)}$\\
	$g(f(g(c,d)),c)\stackrel{12, \sigma=[g(c,d)/x]}{\to} f(h(g(c,d),c))\nrightarrow$
	\section{Präsenzübung}
	TES:\\
	Terme:\\
	Menge von Funktionssymbolen. $\Sigma$-Signatur\\
	für jedes Funktionssymbol eine Stetigkeit $ar\Sigma\to \mathbb{N}$ (wir haben $0\in\mathbb{N}$ definiert)\\
	Notation für $f\in\Sigma$ mit $ar(f)=n: f/n\in\Sigma$\\
	$t::=x|f(t_1,\dots,t_n), f/n\in\Sigma, x\in V$\\
	z.B.: $\Sigma = \{\cdot/2,c/0\}$, wäre $c\cdot c$ oder $c\cdot c\cdot c$\\
	Kontexte:\\
	$C(\cdot)::= (\cdot)|f(t_1,\dots,(\cdot),t_n)$\\
	Bsp.: $C(\cdot) = c+(\cdot)$ und wenn $t=c\cdot c$, dann wäre $C(t)=c+t=c+c\cdot c$\\
	Substitution\\
	$\sigma:V\to T_{\Sigma}(V)$\\
	Termersetzungssysteme: $\to_0\subseteq T_\Sigma\times T_\Sigma$\\
	Einschrittrelation: $\to$ ist der Kontextabgeschlossen und stabile Abschluss von $\to_0$\\
	stabile $s\to t\implies s\sigma\to t\sigma$, Kontextabgeschlossen $s\to t\implies C(s)\to C(t)$\\
	$\{(C(s\sigma),C(t\sigma))|s\to t, C=\ Kontext, \sigma\ substitution\}$\\
	\\
	$t\in T_\Sigma(V)$ heißt normal, wenn man t nicht mehr reduzieren kann $\nto$
	s heißt Normalform von t, wenn $t\to^* s$ und $s\nto$\\
	$t\in T_\Sigma(V)$ schwach normalisierend, wenn es eine NF gibt.\\
	$t\in T_\Sigma(V)$ stark Normalisierend (SN), wenn jede Ableitung in einer NF endet.\\
	$\to_0$ heißt (stark/schwach) normalisierend, wenn jeder term t (stark/schwach) normalisierend ist.\\
	$A\cdot C\stackrel{1)\sigma = [c/x]}{\to} B\cdot(C\cdot C)\stackrel{3)\sigma = [c/x,c/y]}{\to} A\cdot (D \cdot C)\stackrel{1)\sigma = [DC/x]}{\to} B\cdot (C\cdot D\cdot C)\stackrel{3)\sigma = [C/x,DC/y]}{\to} A\cdot (D\cdot C)\to loop $\\
	alternativ $A\cdot (D\cdot C)\stackrel{1) C(\cdot) = (\cdot),\sigma=[DC/x]}{\to} B\cdot(C\cdot(D\cdot C))\stackrel{2)C(\cdot)=B\cdot (\cdot),\sigma = [C/x]}{\to} B\cdot (B\cdot (C\cdot C))\to  loop$ wie vorher\\
	oder wenn man regel 4 anwendet $B\cdot (B\cdot (C\cdot C))\stackrel{4) \sigma=[c\cdot C/x]}{\to} D\cdot C\cdot C\nto$\\
	Übung 2:\\
	1. man kann nicht alle Terme bilden $((x_1\Delta x_2)\Delta x_3) \Delta x_4$ lässt sich mit nur 3 variablen nicht bilden. Also eingeschränkte Ausdrucksmöglichkeit.\\
	verhält sich wie begrenzter Speicher.\\
	2.\\
	$C(\cdot) = (\cdot)| t\Delta (\cdot)|(\cdot)\Delta t$\\
	3.\\
	$t=x_1\Delta (c\Delta x_2)\stackrel{6) \sigma=[c/x_2,x_2/x_3]}{\to} (x_1\Delta c)\Delta x_2\stackrel{5) C(\cdot)=(\cdot)\Delta x_2}{\to}x_1\Delta c\stackrel{5)}{\to} x_1 \nto$\\
	$t=x_1\Delta (c\Delta x_2)\stackrel{7) C=x_1\Delta (\cdot),\sigma[x_1/x_2]}{\to} x_1\Delta x_2\nto$\\
	4.\\
	$C(\cdot)\Delta c$ für beliebigen Kontext.\\
	(Dann kann man am ende, egal welchen pfad man gelaufen ist die Regel 5/6 anwenden um auf den selben Wert zu kommen)\\
	5.\\
	nicht stark normalisierend $(c\Delta c)\Delta c$ und dann 6/8 unendlich oft iterieren.\\
	Aber: jeder Term mit ``Endlosschleife'' hat  ``am Ende'' ein c, d.h. (5) anwendbar. (5) und (7) machen den Term stets kleiner. Es ist also schwach normalisierend.\\
	3)\\
	1. $xRy\land yRx\implies xRx$ das ist aufgrund der irreflexivität verboten, also darf nur einer dieser beiden Terme $xRy$ $yRx$ existieren.\\
	Beweis:\\
	Sei R irreflexiv und transitiv, sei $x,y\in X $ und $xRy$:\\
	zZ $\forall x,y: xRy\implies \lnot yRx$\\
	Annahme es gilt $yRx$, dann folgt aus transitivität $yRx\land xRy\implies xRx$ was im widerspruch zu irreflexivität von R steht. Somit ($\lnot$ Elim) R ist antisymmetrisch.\\
	2. Sei R eine transitive und asymmetrische Relation.\\
	zZ: R ist irreflexiv.\\
	Annahme: es gilt $xRx$.\\
	Aus antisymmetrie folgt insbesonder $xRx\implies \lnot xRx$ WIDERSPRUCH zur annahme( oder alternativ: dies kann nur wahr sein, wenn $xRx$ falsch ist.)\\
\end{document}