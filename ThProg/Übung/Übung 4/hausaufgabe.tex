\documentclass{article}
\usepackage{listings}
\usepackage{mathrsfs}
\usepackage[utf8]{inputenc}
\usepackage{amssymb}
\usepackage{lipsum}
\usepackage{amsmath}
\usepackage{fancyhdr}
\usepackage{geometry}
\usepackage{scrextend}
\usepackage[english,german]{babel}
\usepackage{titling}
\usepackage{verbatim}
\setlength{\droptitle}{-3cm}
\usepackage{tikz}
\usepackage{algorithm,algpseudocode}
\usepackage[doublespacing]{setspace}
\usetikzlibrary{datavisualization}
\usetikzlibrary{datavisualization.formats.functions}
\usepackage{polynom}
\usepackage{amsmath}
\usepackage{gauss}
\usepackage{euscript}
\usepackage{tkz-euclide}
\usepackage{stackengine}
\usetikzlibrary{datavisualization}
\usetikzlibrary{datavisualization.formats.functions}
\title{Übungsblatt 5}
\author{
Alexander Mattick Kennung: qi69dube\\
Kapitel 1
}
\usepackage{import}
\date{\today}
\geometry{a4paper, margin=2cm}
\usepackage{stackengine}
\parskip 1em
\newcommand\stackequal[2]{%
  \mathrel{\stackunder[2pt]{\stackon[4pt]{=}{$\scriptscriptstyle#1$}}{%
  $\scriptscriptstyle#2$}}
 }
\makeatletter
\renewcommand*\env@matrix[1][*\c@MaxMatrixCols c]{%
  \hskip -\arraycolsep
  \let\@ifnextchar\new@ifnextchar
  \array{#1}}
\makeatother
\lstset{
  language=haskell,
}
\lstnewenvironment{code}{\lstset{language=Haskell,basicstyle=\small}}{}
\usepackage{enumitem}
\setlist{nosep}
\usepackage{titlesec}
\newcommand{\nto}{\nrightarrow}
\newcommand{\smallAscr}{\scriptscriptstyle\mathcal{A}}
\title{Präsenzübung 2}
\titlespacing*{\subsection}{0pt}{2pt}{3pt}
\titlespacing*{\section}{0pt}{0pt}{5pt}
\titlespacing*{\subsubsection}{0pt}{1pt}{2pt}
\newtheorem{satz}{Satz}
\newtheorem{korrolar}{Korrolar}[section]
\newtheorem{lemma}{Lemma}[section]
\newtheorem{beweis}{Beweis}[section]
\newtheorem{beispiel}{Beispiel}[section]
\newtheorem{definition}{Definition}[section]

\usepackage{ marvosym }
\begin{document}
	\maketitle\noindent
	\section{Übung 3}
	1. $C(\cdot )=(\cdot)$\\
	7) $\infty \oslash (x\oslash\infty)\to_0 x\oslash x$\\
	9) $x \oslash (y\oslash z)\to_0 (x \oslash y)\oslash z$\\
	Term: $\infty \oslash (x\oslash\infty)\nto$\\
	Erste ableitung $\infty \oslash (x\oslash\infty)\stackrel{7)\sigma=[]}{\to_0} x\oslash x$\\
	Zweite (mit leerem kontext) $\infty \oslash (x\oslash\infty)\stackrel{9) \sigma=[\infty/x,y/x,\infty/z]}{\to_0} (\infty \oslash x)\oslash\infty\stackrel{8) \sigma=[\infty/y]}{\to_0} (x\oslash y)\oslash \heartsuit\nto$\\
	2. 7) und 8)\\
	$C(\cdot) = (\infty\oslash x)\oslash (\cdot)$\\
	$t=\infty\oslash (x\oslash \infty)$\\
	$t= \infty\oslash (x\oslash \infty) \stackrel{7) \sigma=[]}{\to_0} x\oslash x\implies C(x\oslash x)= (\infty\oslash x)\oslash (x\oslash x)\stackrel{8) \sigma=[(x\oslash x)/y]}{\to_0} (x\oslash (x\oslash x))\oslash \heartsuit\stackrel{9) \sigma=[x/y,x/z] C(\cdot) = x\oslash\heartsuit}{\to_0} ((x\oslash x)\oslash x)\oslash \heartsuit\nto$\\
	also $C(t)= (\infty\oslash x)\oslash (((x\oslash x)\oslash x)\oslash \heartsuit)$ (Umklammern hier nach regeln noch möglich, aber nicht notwendig, vgl unten)\\
	$C(t) =(\infty\oslash x)\oslash (\infty\oslash (x\oslash \infty)) \stackrel{9) \sigma = [(\infty\oslash x)/x,\infty/y,(x\oslash \infty)/z]}{\to_0} ((\infty\oslash x)\oslash \infty)\oslash (x\oslash \infty)\stackrel{9) \sigma = [((\infty\oslash x)\oslash \infty)/x,x/y,\infty/z]}{\to_0}(((\infty\oslash x)\oslash \infty)\oslash x)\oslash \infty\stackrel{8) \sigma=[\infty/y], C(\cdot)=((\cdot)\oslash x)\oslash \infty}{\to_0} (((x\oslash\infty)\oslash\heartsuit)\oslash x)\oslash \infty\nto$\\
	Sind nicht zf, da die Reihenfolge von symbolen nie getauscht wird und das herz nie entfernt wird.\\
	Da das Herz beim ersten ganz am ende des Terms, bei anderen direkt am anfang steht, sind sie nicht zf.\\
	\section{Übung 4}
	Zuerst kritische Paare:\\
	Regel 10:\\
	\begin{tabular}{cccc}
	$l_1=$  & $\lnot$&$\lnot$ &x\\
	  &$l_2=$&$\lnot$ &$\lnot x'$\\
	  &$l_2=$&$\lnot$&$(\lnot x'\lor \lnot y)$\\
	\end{tabular}\\
	Regel 11 hat keine.\\
	Regel 12:\\
	\begin{tabular}{ccccc}
	$l_1=$ &$\lnot$ & ($x$&$\land$ &y)\\
		    &$l_2=$&($x'$ & $\land$ & $y'$)\\
	\end{tabular}\\
	Regel 13 hat keine.\\
	Regel 14 hat nur eine (sonst hat man trivialen Kontext bei selbstanwendung):\\
	\begin{tabular}{cccccc}
	$l_1=$ &$x$& $\lor$& ($x$&$\lor$ &y)\\
	 &$l_2=$& &$x'$& $\lor$&(y'$\lor$ z')\\
	\end{tabular}\\
	Regel 15 hat keine.\\
	Beweis für Regel 10.1:\\
	$\sigma = [\lnot x'/x]$ liefert $r_1\sigma = \lnot x'$ und $C(r_2)\sigma = \lnot x'$ ist also z.f.\\
	Beweis Regel 10.2\\
	$\sigma = [(\lnot x'\lor \lnot y)/x]$ liefert $r_1\sigma = \lnot (\lnot x'\lor \lnot y)$ un $C(r_2)\sigma = \lnot (\lnot x\lor\lnot y)$ ist also auch sofort wahr.\\
	Beweis Regel 12.1:\\
	$\sigma = [x'/x, y'/y]$ liefert $r_1\sigma = (\lnot x'\lor \lnot y')$ und $C(r_2)\sigma = \lnot(\lnot(\lnot x'\lor \lnot y'))\stackrel{10)}{\to_0} (\lnot x'\lor \lnot y')$ ist also z.f.\\
	Beweis Regel 14.1:\\
	$\sigma = [x'/x, (y'\lor z')/y]$ liefert $r_1\sigma = (x\lor x')\lor(y'\lor z')\stackrel{14)}{\to_0} ((x\lor x')\lor y')\lor z'$ und $C(r_2)\sigma = x\lor ((x'\lor y')\lor z')\stackrel{14)}{\to_0} (x\lor (x'\lor y'))\lor z'$ ist gleich, also z.f.\\
	Es gibt keine weiteren kritischen Paare, da alle weiteren trivial sind $l_2$ liegt vollkommen unterhalb von $l_1$, z.B.: bei $l_2=\lnot \lnot x$ und $l_1=\lnot \lnot x$ liefert $\lnot \lnot \lnot \lnot x$ also regel 10 2 mal anwendenden und z.B. Regel 15 kann sich mit gar nichts beissen, da $\forall$ in keiner anderen Regel vorkommt. Es folgt, dass logischerweise auch nichts ``kaputtreduziert'' werden kann.\\
	Alle kritischen paare sind z.f. ($\implies$ WCR) und das TES ist SN (letzte Hausaufgabe), es folgt also CR über Newmann's Lemma.\\

\end{document}





