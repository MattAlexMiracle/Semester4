\documentclass{article}
\usepackage{listings}
\usepackage{mathrsfs}
\usepackage[utf8]{inputenc}
\usepackage{amssymb}
\usepackage{lipsum}
\usepackage{amsmath}
\usepackage{fancyhdr}
\usepackage{geometry}
\usepackage{scrextend}
\usepackage[english,german]{babel}
\usepackage{titling}
\usepackage{verbatim}
\setlength{\droptitle}{-3cm}
\usepackage{tikz}
\usepackage{algorithm,algpseudocode}
\usepackage[doublespacing]{setspace}
\usetikzlibrary{datavisualization}
\usetikzlibrary{datavisualization.formats.functions}
\usepackage{polynom}
\usepackage{amsmath}
\usepackage{gauss}
\usepackage{euscript}
\usepackage{tkz-euclide}
\usepackage{stackengine}
\usetikzlibrary{datavisualization}
\usetikzlibrary{datavisualization.formats.functions}
\title{Übungsblatt 5}
\author{
Alexander Mattick Kennung: qi69dube\\
Kapitel 1
}
\usepackage{import}
\date{\today}
\geometry{a4paper, margin=2cm}
\usepackage{stackengine}
\parskip 1em
\newcommand\stackequal[2]{%
  \mathrel{\stackunder[2pt]{\stackon[4pt]{=}{$\scriptscriptstyle#1$}}{%
  $\scriptscriptstyle#2$}}
 }
\makeatletter
\renewcommand*\env@matrix[1][*\c@MaxMatrixCols c]{%
  \hskip -\arraycolsep
  \let\@ifnextchar\new@ifnextchar
  \array{#1}}
\makeatother
\lstset{
  language=haskell,
}
\lstnewenvironment{code}{\lstset{language=Haskell,basicstyle=\small}}{}
\usepackage{enumitem}
\setlist{nosep}
\usepackage{titlesec}
\newcommand{\nto}{\nrightarrow}
\title{Übung 1}
\titlespacing*{\subsection}{0pt}{2pt}{3pt}
\titlespacing*{\section}{0pt}{0pt}{5pt}
\titlespacing*{\subsubsection}{0pt}{1pt}{2pt}



\begin{document}
	\maketitle
	\section{Aufgabe 1.1}
	1. $d_{trans} = \frac{L}{R}$\\
	2. $d_{prop} = \frac{l}{v}$\\
	3. $d_{gesamt} = d_{trans}+d_{prop}+d_{queue}$\\
	4. Bei $t=d_{trans}$ ist das letzte bit gerade auf den Link gelegt worden. (oder wird gerade gelegt, abhängig davon, ob man 0 = ersten bit auf den Link gelegt, oder t=vor dem ersten bit auf den Link)\\
	5. das erste bit ist irgendwo auf dem Link.\\
	6. das erste bit ist beim Ziel B.\\
	7. $d_{trans} = \frac{L}{R} = \frac{200bit}{1Mbps} = \frac{200bit}{1,000,000bps} = 10^{-6} s$, $d_{prop} = \frac{l}{v}\stackequal{!}{}d_{trans}\implies d_{trans}*v = l \implies 10^{-6} s* 2*10^8\frac{m}{s} = 2000m$\\
	\section{Aufgabe 1.2}
	Jeweils zwischen A und switch und B und switch gibt es einen Link.\\
	Das erste Paket kommt nach $2\frac{L}{R}$ am ziel an, das zweite nach $3\frac{L}{R}$ (war bei $2\frac{L}{R}$ schon bei switch), das dritte nach $4\frac{L}{R}$\dots\\
	also kommt das n-te paket nach
	\[(n+1)\frac{L+h}{R} = (\frac{O}{L}+1)\frac{L+h}{R} = \]
	(Die anzahl der Pakete ist nur von O abhängig).\\
	ableiten $\implies L=\sqrt{150\cdot O}$
	\section{Aufgabe 1.3}
	1. $(N+E-1)d_{trans}+d_{con} = (N+E-1)\frac{\frac{O}{N}+h}{R}+d_{con}$\\
	2. $(N+E-1)d_{trans} = (N+E-1)\frac{\frac{O}{N}+2h}{R}$\\
	3. $(N+E-1)d_{trans} = (1+E-1)\frac{O+2h}{R}$\\
	4. $d_{conn}+\frac{h+O}{R}$
\end{document}