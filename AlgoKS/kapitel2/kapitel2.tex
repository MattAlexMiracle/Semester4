\documentclass{article}
\usepackage{listings}

\usepackage[utf8]{inputenc}
\usepackage{amssymb}
\usepackage{lipsum}
\usepackage{amsmath}
\usepackage{fancyhdr}
\usepackage{geometry}
\usepackage{scrextend}
\usepackage[english,german]{babel}
\usepackage{titling}
\setlength{\droptitle}{-3cm}
\usepackage{tikz}
\usepackage{algorithm,algpseudocode}
\usepackage[doublespacing]{setspace}
\usetikzlibrary{datavisualization}
\usetikzlibrary{datavisualization.formats.functions}
\usepackage{polynom}
\usepackage{amsmath}
\usepackage{gauss}
\usepackage{tkz-euclide}
\usetikzlibrary{datavisualization}
\usetikzlibrary{datavisualization.formats.functions}
\title{Übungsblatt 5}
\author{
Alexander Mattick Kennung: qi69dube\\
Kapitel 2
}
\usepackage{import}
\date{\today}
\geometry{a4paper, margin=2cm}
\usepackage{stackengine}
\parskip 1em
\newcommand\stackequal[2]{%
  \mathrel{\stackunder[2pt]{\stackon[4pt]{=}{$\scriptscriptstyle#1$}}{%
  $\scriptscriptstyle#2$}}
 }
\makeatletter
\renewcommand*\env@matrix[1][*\c@MaxMatrixCols c]{%
  \hskip -\arraycolsep
  \let\@ifnextchar\new@ifnextchar
  \array{#1}}
\makeatother
\lstset{
  language=haskell,
}
\lstnewenvironment{code}{\lstset{language=Haskell,basicstyle=\small}}{}
\usepackage{minted}
\usepackage{enumitem}
\setlist{nosep}
\usepackage{titlesec}

\titlespacing*{\subsection}{0pt}{2pt}{3pt}
\titlespacing*{\section}{0pt}{0pt}{5pt}
\titlespacing*{\subsubsection}{0pt}{1pt}{2pt}



\begin{document}
	\maketitle
	\section{SIR-Modell}
	besteht aus ODE:\\
	\begin{itemize}
		\item $\frac{dS(t)}{dt} = -\theta I(t)S(t)$
		\item $\frac{dI(t)}{dt} = \theta I(t)S(t)-\gamma I(t)$
		\item $\frac{dR(t)}{dt} = \gamma I(t)$
	\end{itemize}
	S=susceptible, I=Infectuous, R=Recovered, $\theta$ erkrankungsrate, $\gamma$ genesungsrate\\
	Krankheitsdauer ist $\frac{1}{\gamma}$ $R_0 = \frac{\theta}{\gamma}$.\\
	Alternativ zur letzten gleichung kann auch gefordert werden, dass die gesamtzahl konstant bleibt: $R+S+I=1\implies R= 1-S-I$. (wenn man diese differentiert und von oben einsetzt, erhält man die ursprüngliche bedingung)\\
	Lösung über Eulerverfahren: $y'\approx \frac{y(t+h)-y(t)}{h}\implies y(t+h)\approx y(t)+h\cdot y'$\\
	Für die DGL	$y'=f(t,y)$ gilt also $\tilde y(t+h)= \tilde y+hf(t,\tilde y)$ um eine approximation $\tilde y$ für y zu finden.\\
	Eulerverfahren hat ordnung h (also für jeden schritt erhält man linear genauigkeit, gute DGL löser haben $h^4$ bis $h^8$)\\
	Die genauigkeit steigt mit $O(h^2)$ bei mehr schritten.\\
	Die Anzahl der schritte für eine verbesserung verhält sich nach $O(h^{-1})$ (wird also immer schlechter)\\
	Dies könnte die höhere granularität überwiegen (tuts auch, wenn nicht Lipschitz!)\\


\end{document}