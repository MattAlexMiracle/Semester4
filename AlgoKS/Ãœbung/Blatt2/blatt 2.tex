\documentclass{article}
\usepackage{listings}
\usepackage{mathrsfs}
\usepackage{cancel}
\usepackage[utf8]{inputenc}
\usepackage{amssymb}
\usepackage{lipsum}
\usepackage{amsmath}
\usepackage{fancyhdr}
\usepackage{geometry}
\usepackage{scrextend}
\usepackage[english,german]{babel}
\usepackage{titling}
\usepackage{verbatim}
\setlength{\droptitle}{-3cm}
\usepackage{tikz}
\usepackage{algorithm,algpseudocode}
\usepackage[doublespacing]{setspace}
\usetikzlibrary{datavisualization}
\usetikzlibrary{datavisualization.formats.functions}
\usepackage{polynom}
\usepackage{amsmath}
\usepackage{gauss}
\usepackage{euscript}
\usepackage{tkz-euclide}
\usepackage{stackengine}
\usetikzlibrary{datavisualization}
\usetikzlibrary{datavisualization.formats.functions}
\title{Übungsblatt 5}
\author{
Alexander Mattick Kennung: qi69dube\\
Kapitel 1
}
\usepackage{import}
\date{\today}
\geometry{a4paper, margin=2cm}
\usepackage{stackengine}
\parskip 1em
\newcommand\stackequal[2]{%
  \mathrel{\stackunder[2pt]{\stackon[4pt]{=}{$\scriptscriptstyle#1$}}{%
  $\scriptscriptstyle#2$}}
 }
\makeatletter
\renewcommand*\env@matrix[1][*\c@MaxMatrixCols c]{%
  \hskip -\arraycolsep
  \let\@ifnextchar\new@ifnextchar
  \array{#1}}
\makeatother
\lstset{
  language=haskell,
}
\lstnewenvironment{code}{\lstset{language=Haskell,basicstyle=\small}}{}
\usepackage{enumitem}
\setlist{nosep}
\usepackage{titlesec}
\newcommand{\nto}{\nrightarrow}
\newcommand{\smallAscr}{\scriptscriptstyle\mathcal{A}}
%\newcommand{\nsqsubseteq}{\xout{\sqsubseteq}}
\title{Vorlesung 2}
\titlespacing*{\subsection}{0pt}{2pt}{3pt}
\titlespacing*{\section}{0pt}{0pt}{5pt}
\titlespacing*{\subsubsection}{0pt}{1pt}{2pt}
\newtheorem{satz}{Satz}
\newtheorem{korrolar}{Korrolar}[section]
\newtheorem{lemma}{Lemma}[section]
\newtheorem{beweis}{Beweis}[section]
\newtheorem{beispiel}{Beispiel}[section]
\newtheorem{definition}{Definition}[section]

\begin{document}
	\maketitle
	$A = {\begin{bmatrix} 
	4 & 4 & 6 \\
	8 & 4 & 4 \\
	3 & 9 & 6 \end{bmatrix}}$
	Tauschen:\\
	$P = \begin{bmatrix}
	0&1&0\\
	1&0&0\\
	0&0&1
	\end{bmatrix}
	$\\
	$PA = \begin{bmatrix} 
	8 & 4 & 4 \\
	4 & 4 & 6 \\
	3 & 9 & 6
	 \end{bmatrix}$\\
	Elimination:\\
	$L_0=
	\begin{bmatrix} 
	1    & 0 & 0 \\
	1/2  & 1 & 0 \\
	3/8  & 0 & 1
	 \end{bmatrix} $\\	
	$A_{rest} = 
	\begin{bmatrix} 
	8 & 4   & 4 \\
	0 & 2   & 4 \\
	0 & 15/2 & 9/2
	 \end{bmatrix}$
	$P_2 = \begin{bmatrix}
	1&0&0\\
	0&0&1\\
	0&1&0
	\end{bmatrix}
	$\\
	Also Ergebniss:\\
	$L_1=
	\begin{bmatrix} 
	1     & 0 & 0 \\
	1/2  & 1 & 0 \\
	3/8 & 4/15 & 1
	 \end{bmatrix}$\\
	$R = \begin{bmatrix} 
	8 & 4   & 4 \\
	0 & 15/2 & 9/2\\
	0 & 0   & 14/5	
	 \end{bmatrix}$\\
	$P_{ges} = P_1 P_2 = \begin{bmatrix}
	0&0&1\\
	1&0&0\\
	0&1&0\end{bmatrix}
	$
	$4x-2/5-10/3=-2$
\end{document}



