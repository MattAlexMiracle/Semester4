\documentclass{article}
\usepackage{listings}
\usepackage{mathrsfs}
\usepackage[utf8]{inputenc}
\usepackage{amssymb}
\usepackage{lipsum}
\usepackage{amsmath}
\usepackage{fancyhdr}
\usepackage{geometry}
\usepackage{scrextend}
\usepackage[english,german]{babel}
\usepackage{titling}
\setlength{\droptitle}{-3cm}
\usepackage{tikz}
\usepackage{algorithm,algpseudocode}
\usepackage[doublespacing]{setspace}
\usetikzlibrary{datavisualization}
\usetikzlibrary{datavisualization.formats.functions}
\usepackage{polynom}
\usepackage{amsmath}
\usepackage{gauss}
\usepackage{tkz-euclide}
\usetikzlibrary{datavisualization}
\usetikzlibrary{datavisualization.formats.functions}
\author{
Alexander Mattick Kennung: qi69dube\\
Kapitel 1
}
\usepackage{import}
\date{\today}
\geometry{a4paper, margin=2cm}
\usepackage{stackengine}
\parskip 1em
\newcommand\stackequal[2]{%
  \mathrel{\stackunder[2pt]{\stackon[4pt]{=}{$\scriptscriptstyle#1$}}{%
  $\scriptscriptstyle#2$}}
 }
\makeatletter
\renewcommand*\env@matrix[1][*\c@MaxMatrixCols c]{%
  \hskip -\arraycolsep
  \let\@ifnextchar\new@ifnextchar
  \array{#1}}
\makeatother
\lstset{
  language=haskell,
}
\lstnewenvironment{code}{\lstset{language=Haskell,basicstyle=\small}}{}
\usepackage{enumitem}
\setlist{nosep}
\usepackage{titlesec}

\titlespacing*{\subsection}{0pt}{2pt}{3pt}
\titlespacing*{\section}{0pt}{0pt}{5pt}
\titlespacing*{\subsubsection}{0pt}{1pt}{2pt}
\title{Vorlesung 3}


\begin{document}
	\maketitle
	Für Wachstumsprozesse ist das geometrische Mittel zu benutzen
	\textbf{Zufallsexperiment}: geplanter, gesteuerter oder beobachteter Vorgang, der ein genau abgrenzbares Ergebnis besitzt, das vom Zufall beeinflusst sein kann.\\
	Wichtig: Grundmenge definieren, mehrmaliges Würfeln $\neq$ einmaliges würfeln!\\
	z.B. Reihenfolge $(e,i,\pi,=,i)$ vs. ``nur gerade Zahlen'' \{2,4,6\}\\
	Rechnen mit Ergebnissen:\\
	$\{2,4,6\}=\{2\}\cup\{4\}\cup\{6\}$\\
	oder $\{2,4,6\}=\{1,2,3,4,5,6\}\setminus \{1,3,5\}$\\
	``Eine gerade Zahl, die nicht durch drei teilbar ist''.\\
	$C_A=$ komplement w.r.t. zu A.\\
	$\{2,4,6\}\cap C_\Omega \{3,6\}=\{2,4\}$
	\begin{itemize}
		\item $\Omega$ alle möglichen Ausgänge
		\item $\omega\in \Omega$ ein möglicher Ausgang
		\item $A\subset \Omega$ Menge möglicher Ergebnisse
		\item Elementarereignisse $\{\omega\}\in \Omega$
		\item Ereignissystem $\mathscr{A}$ abgeschlossenes Mengensystem über $\Omega$
	\end{itemize}
	Die gesamtheit aller Teilmengen ist Potenzmenge von $\Omega$: $\mathscr{P}(\Omega)$\\
	Drei funktionen sollen getestet werden:\\
	$\Omega = \{0,1\}^3$ oder $\Omega\tabularnewline
	=\sum\limits^3_\{i=1\}\omega_i = n_i\in\{0,1,2,3\}$
	Es gibt wege um zwischen $\Omega \leftrightarrow \Omega'$ zu kommen.\\
	Elementarereignisse sind die einzelnen Ereignisse eines Ergebnismenge: $\{1,2,3,4,5,6\}$ sind die Elementarereignisse des Würfels. ``ist gerade'' $\{2,4,6\}$ ist ein zusammengesetztes Ereignis.\\
	Ein abgeschlossenes Mengensystem oder $\sigma$-Algebra.\\
	\begin{itemize}
		\item $\Omega\in A$
		\item $A\in\mathscr{A}\implies A^C\in \mathscr{A}$
		\item $A_1,A_2,\dots \in \mathscr{A}\implies \bigcup^\infty_{i=1} A_i\in\mathscr{A}$
	\end{itemize}
	Bei eindlicher anzahl von $A$ kann man einfach die Potzenmenge $P(\Omega)$ wählen also $\{\emptyset,\Omega,\dots\}$ die kleinste $\sigma$-Algebra ist 2-Elementig.\\
	WICHTIG: grenzübergang zwischen $\bigcup^\infty_{i=1}$ und $\bigcup^n_{i=1}$\\
	Das heißt, dass die summenformulierung von oben immer geht!\\
	Zufallsvariable:\\
	Ist X eine Abb. $\Omega\to\Omega'$ und $A'\subset\Omega'$ wird definiert: $\{X\in A'\}:=\{\omega\in\Omega: X(\omega)\in A'\}$\\
	(Also nur eine Kurzschreibweise für ``das Ereignis ist in A' '')\\
	Eine Teilmenge $A\in\Omega$ der Form $A:= \{X\in A'\}$ heißt durch X beschreibbar.\\
	Eine Zufallsvariable (ZV) ist eine Abb von $X:(\Omega,\mathscr{A}\to(\Omega',\mathscr{A})$ für die gefordert wird:\\
	$\{X\in A'\}\in\mathscr{A}$ für alle $A'\in\mathscr{A}'$\\
	Weiterführende Fragen:\\
	1. $\sigma$-Algebra $\mathscr{A}$ gilt für $A_1,A_2\in\mathscr{A}$ die Aussage $A_1\cap A_2\in \mathscr{A}$: Ja, denn es gilt für jede Aussage $\bigcup^\infty_{i=1} A_i$ und die Negation $A\in \mathscr{A} \implies A^C\in \mathscr{A}$. Nach de-morgan gilt $A^C\cup B^C = (A\cap B)^C\implies A\cap B\in \mathscr{A}$.\\
	2. $\mathbb{B}$ ist die $\sigma$-Borel-Algebra über $\mathbb{R}$ die aus halboffenen intervallen $(a,b]\subset \mathbb{R}$ erzeugt wird. Gilt $(a,b)\in\mathbb{B}$.\\
	Ja.\\
	$n\in\mathbb{N}, n\geq n_0: (a,b-\frac{1}{n}]$
	3. Beschreibe das Zufallsexperiment ``Summe aus drei Würfeln mit einem Würfel'' durch eine Zufallsvariable. Dazu wird ein n-Seitiger würfel betrachtet.\\
	Sei $W=\{i|n\in\mathbb{N}, 1\leq i\leq n \}$ die Würfelseiten (1 bis n).\\
	bei drei würfen können theoretisch alle 3 tuple aus W vorkommen, also:\\
	$W\times W\times W = W^3=\Omega$.\\
	Die Zufallsvariable/funktion X ordnet jetzt alle Werte aus $W^3$ einen Wert aus $[3,3\cdot n] =\Omega'$ zu:\\
	\[X_n = \{\omega|\forall \omega\in\Omega, \omega\cdot (1,1,1)^T = n\}\]
	wobei die Werte aus $\Omega$ als vektor angesehen werden und ``$\cdot$'' als euclidesche Scalarprodukt ist.\\
	Die zufallsvariable ist:\\
	\[X(\omega) = \omega\cdot (1,1,1)^T\]

\end{document}