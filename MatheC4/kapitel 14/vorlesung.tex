\documentclass{article}
\usepackage{listings}
\usepackage{mathrsfs}
\usepackage[utf8]{inputenc}
\usepackage{amssymb}
\usepackage{lipsum}
\usepackage{amsmath}
\usepackage{fancyhdr}
\usepackage{geometry}
\usepackage{scrextend}
\usepackage[english,german]{babel}
\usepackage{titling}
\setlength{\droptitle}{-3cm}
\usepackage{tikz}
\usepackage{algorithm,algpseudocode}
\usepackage[doublespacing]{setspace}
\usetikzlibrary{datavisualization}
\usetikzlibrary{datavisualization.formats.functions}
\usepackage{polynom}
\usepackage{amsmath}
\usepackage{gauss}
\usepackage{tkz-euclide}
\usetikzlibrary{datavisualization}
\usetikzlibrary{datavisualization.formats.functions}
\author{
Alexander Mattick Kennung: qi69dube\\
Kapitel 1
}
\usepackage{import}
\date{\today}
\geometry{a4paper, margin=2cm}
\usepackage{stackengine}
\parskip 1em
\newcommand\stackequal[2]{%
  \mathrel{\stackunder[2pt]{\stackon[4pt]{=}{$\scriptscriptstyle#1$}}{%
  $\scriptscriptstyle#2$}}
 }
\makeatletter
\renewcommand*\env@matrix[1][*\c@MaxMatrixCols c]{%
  \hskip -\arraycolsep
  \let\@ifnextchar\new@ifnextchar
  \array{#1}}
\makeatother
\lstset{
  language=haskell,
}
\lstnewenvironment{code}{\lstset{language=Haskell,basicstyle=\small}}{}
\usepackage{enumitem}
\setlist{nosep}
\usepackage{titlesec}

\titlespacing*{\subsection}{0pt}{2pt}{3pt}
\titlespacing*{\section}{0pt}{0pt}{5pt}
\titlespacing*{\subsubsection}{0pt}{1pt}{2pt}
\title{Vorlesung 4}


\begin{document}
	\maketitle
	bsp.:\\
	gem. Dichte:
	\[f^{(X,Y)}(x,y)= \begin{cases}3y&x\in[-1,1],y\in[0,1-|x|]\\0&sonst\end{cases}\]
	Randdichte $f^X$, $f^Y$\\
	und Übergangsdichten von $f^{(X)} (x)f^{(Y)}_{(X)} (x;y)$\\
	Die Übergangsdichten von i nach i-1 ist eine bedingte Dichte, die abhängig von genau diesen ist.\\
	Dazu Randdichten $f^{(Y_1,\dots,Y_{n-1})},f^{(Y_1,\dots,Y_{n-2})},\dots f^{(Y_1,Y_2)}, f^{(Y_1)}=f_1$.\\
	Dazu kann $f^{(Y_1,\dots,Y_{n-1})}$ einfach berechnet werden:
	Idee: Fasse $f_1(x_1),\dots f^{n-2}_{n-1}(y_1\dots,y_{n-1})= f^{(Y_1,\dots,Y_{n-1})}(y_1\dots,y_n)$\\
	\[f^{(Y_1,\dots,Y_{n-1})}(y_1\dots,y_n) =\int^\infty_{-\infty}\dots\int^\infty_{-\infty}f^{(Y_1,\dots,Y_{n})}(y_1\dots,y_n)dy_n\]
	allgemein durch dividieren, kann man rekursiv berechnen:
	\[f^{(Y_1,\dots,Y_{n})}(y_1\dots,y_n) =f^{(Y_1,\dots,Y_{n-1})}(y_1\dots,y_{n-1})f^{i-1}_i(y_1\dots,y_n)\]
	\[\iff f^{i-1}_i(y_1\dots,y_n) =\frac{f^{(Y_1,\dots,Y_{n})}(y_1\dots,y_n) }{f^{(Y_1,\dots,Y_{n-1})}(y_1\dots,y_{n-1})}\]
	weiter im Beispiel von oben:\\
	\[f^{(X,Y)}(x,y)= \begin{cases}3y&x\in[-1,1],y\in[0,1-|x|]\\0&sonst\end{cases}\]
	\[f^1_2(x,y)=\frac{f^{(X,Y)}(x,y)}{f^{(X)}(x)}\]
	und 
	\[f^X(x) = \int_\mathbb{R} f^{(X,y)}(x,y)dy = \int^\infty_{-\infty}f^{(X,y)}(x,y)dy\]
	\[=\int ^{1-|x|}_0 3y dy=\frac{3}{2}[y^2]_{y=1}^{y=1-|x|} = \frac{3}{2}(1-|x|)^2\; x\in[-1,1]\]
	Ergebnis $f^1_2(x,y)=\frac{3y}{\frac{3}{2}(y-|x|)^2}$ für $x\in[-1,1], y\in[0,1-|x|]$\\
	(Bei 3 variablen könnte man jetzt $f^1_2(x,y)$ für die Randdichte benutzen, um das nächste zu berechnen)
\end{document}
