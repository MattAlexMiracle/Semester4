\documentclass{article}
\usepackage{listings}

\usepackage[utf8]{inputenc}
\usepackage{amssymb}
\usepackage{lipsum}
\usepackage{amsmath}
\usepackage{fancyhdr}
\usepackage{geometry}
\usepackage{scrextend}
\usepackage[english,german]{babel}
\usepackage{titling}
\setlength{\droptitle}{-3cm}
\usepackage{tikz}
\usepackage{algorithm,algpseudocode}
\usepackage[doublespacing]{setspace}
\usetikzlibrary{datavisualization}
\usetikzlibrary{datavisualization.formats.functions}
\usepackage{polynom}
\usepackage{amsmath}
\usepackage{gauss}
\usepackage{tkz-euclide}
\usetikzlibrary{datavisualization}
\usetikzlibrary{datavisualization.formats.functions}
\title{Übungsblatt 5}
\author{
Alexander Mattick Kennung: qi69dube\\
Kapitel 2
}
\usepackage{import}
\date{\today}
\geometry{a4paper, margin=2cm}
\usepackage{stackengine}
\parskip 1em
\newcommand\stackequal[2]{%
  \mathrel{\stackunder[2pt]{\stackon[4pt]{=}{$\scriptscriptstyle#1$}}{%
  $\scriptscriptstyle#2$}}
 }
\makeatletter
\renewcommand*\env@matrix[1][*\c@MaxMatrixCols c]{%
  \hskip -\arraycolsep
  \let\@ifnextchar\new@ifnextchar
  \array{#1}}
\makeatother
\lstset{
  language=haskell,
}
\lstnewenvironment{code}{\lstset{language=Haskell,basicstyle=\small}}{}
\usepackage{minted}
\usepackage{enumitem}
\setlist{nosep}
\usepackage{titlesec}

\titlespacing*{\subsection}{0pt}{2pt}{3pt}
\titlespacing*{\section}{0pt}{0pt}{5pt}
\titlespacing*{\subsubsection}{0pt}{1pt}{2pt}



\begin{document}
	\maketitle
	Fragen:\\
	\section{Streuung $\frac{1}{n}$ vs empirische Streuung $\frac{1}{n-1}$}
	$\frac{1}{n}\sum\limits^n_{i=1}(x_i-\overline{x_i})^2$ vs $\frac{1}{n-1}\sum\limits^n_{i=1}(x_i-\overline{x_i})^2$\\
	Wahrscheinlichkeitsverteilungen\\
	$\to$ Erwartungswert\\
	$\to$ Varianz\\
	Zum ermitteln dieser muss i.a. Stichproben aus der Echten Verteilung gezogen werden.\\
	Für mehr Stichproben n konvergiert die empirische Varianz besser gegen die echte Varianz!\\
	Die $\frac{1}{n-1}$ ist immer größer als $\frac{1}{n}$!\\
	Das ist eine \textbf{pessimistischere Schätzung der Varianz}, woraus die ungewissenheit besser gehandeled werden kann.\\
	Stichprobe $\to$ Bild von der Grundverteilung!\\
	$\to$ Wie gut ist das Bild das wir gesammelt haben?
	\section{Rangwert vs. Ordnungsstatistik.}
	Rangliste schmeist alle doppelten Werte raus, Ordnungsstatistik behält diese.\\
	Ordnungsstatistik: $\{1,2,2,2,3,4\}$ Rangwert $\{1,2,3,4\}$\\
	Der Rangwert kommt aus der Ordnungsstatistik mit $r+\frac{s-1}{2}$, wobei s=Anzahl der Werte
	\section{Mittelwerte}
	Arithmetisches mittel oder Durchschnitt $\overline{x}=\frac{1}{n}\sum x_i$\\
	Harmonisches mittel $\frac{1}{n}(\sum \frac{1}{x_i})^{-1}$ für z.B. geschwindigkeiten und verhältnisse dieser. Wenn man durchschnittliche Raten haben. Es betrachtet die geschwindigkeit der Veränderung.\\
	geometrisches Mittel $\sqrt[n]{\prod\limits^n_{i=1} x_i}$ Für exponentielle zunahmen von z.B. Kontostand und prozentuale Zunahme. Verhältnisse beleiben vorhanden.
	\section{kovarianz}
	Wenn die Kovarianz null ist, dann sind die Werte vollkommen unabhängig.\\
	Die umkehrung gilt nicht, außer bei multinormalen Verteilungen
	\section{Wie sollte man Klassen aufteilen?}
	Gibt kein allheilmittel (mein Vorschlag: aufteilen, dass die Varianz innerhalb jeder Klasse grad kleiner ist, als die der Gesamtvarianz)
	\section{Korrelation}
	Wenn der korrelationskoeff $r_{xy} = \frac{s_{xy}}{s_xs_y}$ nahe an 1 ist, dann ist die korrelation optimal, bei -1 ist sie invers-optimal.\\
	Bestimmtheitsgrad ist das Verhältniss des zweiten moments normalisiert um den mittelwert der Echten Daten:\\
	\[B = \frac{\sum(\hat{y}-\overline{y})}{\sum(y-\overline{y})}\]
	\section{Lineares Ausgleichsproblem}
	Anpassen einer minimalen funktion (z.b. polynom von grad-m $m<<n$)\\
	$\min\sum\limits^n_{j=1} (y_j-(p_mx_j^m+\dots +p_0))$\\
	Dies liefert eine\\
	$\begin{pmatrix}
	x_1&x_1^2&\dots&x_1^m\\
	\dots&\dots&\dots&\dots\\
	x_n&x_n^2&\dots&x_n^m
	\end{pmatrix}$\\
	Also eine $N\times M$ matrix.\\
	Dies liefert $\nabla_p z(p)=A^TAp-A^T_y\stackon{!}{=}$
	Für optimalität zweiter Ordnung: $Hf(x)$ muss positiv definit sein, für ein absolutes min!\\
	Das $H_z(p)=A^TA$\\
	$A^TA$ ist immer symmetrisch! Also auch postiv (semi)-definit!\\
	\\
	\\
	\textbf{Für den einfachen Fall von 1d-(x,y):\\}
	$\min \frac{1}{2}\sum\limits^n_{k-1}(ax_i+n-y_i)^2$
	minimiert $a=\frac{s_{xy}}{(s_x)^2}$ und $b=\overline{y}-a\overline{x}$\\
	Diese Funktion.\\
\end{document}