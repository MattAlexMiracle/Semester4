\documentclass{article}
\usepackage{listings}
\usepackage{mathrsfs}
\usepackage[utf8]{inputenc}
\usepackage{amssymb}
\usepackage{lipsum}
\usepackage{amsmath}
\usepackage{fancyhdr}
\usepackage{geometry}
\usepackage{scrextend}
\usepackage[english,german]{babel}
\usepackage{titling}
\setlength{\droptitle}{-3cm}
\usepackage{tikz}
\usepackage{algorithm,algpseudocode}
\usepackage[doublespacing]{setspace}
\usetikzlibrary{datavisualization}
\usetikzlibrary{datavisualization.formats.functions}
\usepackage{polynom}
\usepackage{amsmath}
\usepackage{gauss}
\usepackage{tkz-euclide}
\usepackage{minted}
\usepackage{mathrsfs}
\usetikzlibrary{datavisualization}
\usetikzlibrary{datavisualization.formats.functions}
\author{
Alexander Mattick Kennung: qi69dube\\
Kapitel 1
}
\usepackage{import}
\date{\today}
\geometry{a4paper, margin=2cm}
\usepackage{stackengine}
\parskip 1em
\newcommand\stackequal[2]{%
  \mathrel{\stackunder[2pt]{\stackon[4pt]{=}{$\scriptscriptstyle#1$}}{%
  $\scriptscriptstyle#2$}}
 }
\makeatletter
\renewcommand*\env@matrix[1][*\c@MaxMatrixCols c]{%
  \hskip -\arraycolsep
  \let\@ifnextchar\new@ifnextchar
  \array{#1}}
\makeatother
\lstset{
  language=haskell,
}
\lstnewenvironment{code}{\lstset{language=Haskell,basicstyle=\small}}{}
\usepackage{enumitem}
\setlist{nosep}
\usepackage{titlesec}
\usepackage{ stmaryrd }
\usepackage{verbatim}
\usepackage{tikz-qtree}
\usepackage{bussproofs}

\titlespacing*{\subsection}{0pt}{2pt}{3pt}
\titlespacing*{\section}{0pt}{0pt}{5pt}
\titlespacing*{\subsubsection}{0pt}{1pt}{2pt}
\title{Übung 7}


\begin{document}
	\maketitle
	a)\\
	da alle Störungen unabhängig sind, erhält man eine unabhängige Kopplung
	\[f(y) = N((1,1,2,3)^T,\begin{bmatrix}4&0&0&0\\0&1&0&0\\0&0&9&0\\0&0&0&4\end{bmatrix})\]
	b)\\
	wir haben eine Transformation von $Y_1,Y_2,Y_3,Y_4$ zu einem Vektor $Z_1,Z_2,Z_3$\\
	dabei ist $b=(10,20,5)^T$ und $B=\begin{bmatrix}3&0&1&2\\0&-2&5&3\\0&-4&5&-1\end{bmatrix}$ (also ich gehe davon aus, dass mit $+-Y_4$ in der angabe einfach minus gemeint ist)\\
	in $b+BY=Z$\\
	dabei gilt\\
	$N(b+Ba,BKB^T)$ wobei $Y\sim N(a,k)$ ist\\
	\[Ba = \begin{bmatrix}3&0&1&2\\0&-2&5&3\\0&-4&5&-1\end{bmatrix}(1,1,2,3)^T = (11,17,3)^T\]
	somit ist
	\[b+Ba = (10,20,5)^T+(11,17,3)^T=(21,37,8)^T\]
	\[BK=\begin{bmatrix}3&0&1&2\\0&-2&5&3\\0&-4&5&-1\end{bmatrix}\begin{bmatrix}4&0&0&0\\0&1&0&0\\0&0&9&0\\0&0&0&4\end{bmatrix}=\begin{bmatrix}12&0&9&8\\0&-2&45&12\\0&-4&45&-4\end{bmatrix}\]
	\[BKB^T = \begin{bmatrix}12&0&9&8\\0&-2&45&12\\0&-4&45&-4\end{bmatrix}
	\begin{bmatrix}
	3&0&0\\
	0&-2&-4\\
	1&5&5\\
	2&3&-1
	\end{bmatrix}=\begin{bmatrix}61&69&37\\69&265&221\\37&221&245\end{bmatrix} \]
	das inverse ist
	Daraus folgt\\
	\[Z\sim N((21,37,8)^T, \begin{bmatrix}61&69&37\\69&265&221\\37&221&245\end{bmatrix})\]
	c)\\
	Die Kovarianzen stehen in $BKB^T$\\
	$Kov(Z_1,Z_2)=Kov(Z_2,Z_1)=69$\\
	$Kov(Z_1,Z_3)=Kov(Z_3,Z_1)=37$\\
	$Kov(Z_2,Z_3)=Kov(Z_2,Z_3)=221$\\
	
	\[\mathscr{N}\]


\end{document}
