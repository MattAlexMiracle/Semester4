\documentclass{article}
\usepackage{listings}
\usepackage{mathrsfs}
\usepackage[utf8]{inputenc}
\usepackage{amssymb}
\usepackage{lipsum}
\usepackage{amsmath}
\usepackage{fancyhdr}
\usepackage{geometry}
\usepackage{scrextend}
\usepackage[english,german]{babel}
\usepackage{titling}
\setlength{\droptitle}{-3cm}
\usepackage{tikz}
\usepackage{algorithm,algpseudocode}
\usepackage[doublespacing]{setspace}
\usetikzlibrary{datavisualization}
\usetikzlibrary{datavisualization.formats.functions}
\usepackage{polynom}
\usepackage{amsmath}
\usepackage{gauss}
\usepackage{tkz-euclide}
\usepackage{minted}
\usetikzlibrary{datavisualization}
\usetikzlibrary{datavisualization.formats.functions}
\author{
Alexander Mattick Kennung: qi69dube\\
Kapitel 1
}
\usepackage{import}
\date{\today}
\geometry{a4paper, margin=2cm}
\usepackage{stackengine}
\parskip 1em
\newcommand\stackequal[2]{%
  \mathrel{\stackunder[2pt]{\stackon[4pt]{=}{$\scriptscriptstyle#1$}}{%
  $\scriptscriptstyle#2$}}
 }
\makeatletter
\renewcommand*\env@matrix[1][*\c@MaxMatrixCols c]{%
  \hskip -\arraycolsep
  \let\@ifnextchar\new@ifnextchar
  \array{#1}}
\makeatother
\lstset{
  language=haskell,
}
\lstnewenvironment{code}{\lstset{language=Haskell,basicstyle=\small}}{}
\usepackage{enumitem}
\setlist{nosep}
\usepackage{titlesec}
\usepackage{ stmaryrd }
\usepackage{verbatim}
\usepackage{tikz-qtree}
\usepackage{bussproofs}

\titlespacing*{\subsection}{0pt}{2pt}{3pt}
\titlespacing*{\section}{0pt}{0pt}{5pt}
\titlespacing*{\subsubsection}{0pt}{1pt}{2pt}
\title{Übung 7}


\begin{document}
	\maketitle
	\section{}
	a)
	\[G_1,G_2,G_3\sim N(0,1)\]
	\[X = (x_1,x_2,x_3)^T =\begin{bmatrix}
	3 & -a & b\\
	b & a  & 1\\
	a & b & -4
	\end{bmatrix} \begin{bmatrix}G_1\\G_2\\G_3 \end{bmatrix}+\begin{bmatrix}2\\1\\0 \end{bmatrix}\]
	$X = AG+a$\\
	\[K := AA^T =\begin{bmatrix}
	3 & -a & b\\
	b & a  & 1\\
	a & b & -4
	\end{bmatrix}\begin{bmatrix}
	3 & b & a\\
	-1 & a  & b\\
	b & 1 & -4
	\end{bmatrix} = \begin{bmatrix}
	9+a^2+b^2 & 4b-a^2 & 3a-ab-4b\\
	4b^2-a^2 & a^2+b^2+1 & 2ab-4\\
	3a-ab-4b &2ab-4 & a^2+b^2+16
	 \end{bmatrix}\]
	b)\\
	n-dim ist Normalverteilung $N(a,k)$ ist\\
	\[f^X(x)= (\frac{1}{\sqrt{2\pi}})^n \frac{1}{\sqrt{det(K)}}e^{-\frac{1}{2}(x-a)^TK^{-1} (x-a)}\]
	hier ist n=3
	\[f^X(x)= (\frac{1}{\sqrt{2\pi}})^3 \frac{1}{\sqrt{det(K)}}e^{-\frac{1}{2}(x-a)^TK^{-1} (x-a)}\]
	\[det(k)  = det(\begin{bmatrix}k_{22}&k_{23}\\k_{32}&k_{33}\end{bmatrix})-det(\begin{bmatrix}k_{12}&k_{13}\\k_{32}&k_{33}\end{bmatrix})+det(\begin{bmatrix}k_{12}&k_{13}\\k_{22}&k_{23}\end{bmatrix})\]
	$x = (x_1,x_2,x_3)^T$ ist bekannt\\
	$a = (2,1,0)^T$ ist mittelwert\\
	somit haben wir $(x-a)^T = (x_1-2,x_2-1,x_3)$ (kein transponiert am ende, wir kippen mit $\^T$\\
	$K^{-1} =\frac{1}{det(k)}K^T$ weil K positiv definit sein muss!\\
	c)\\
	K muss diagonal sein.\\
	geht, wenn $b=1$ und $a=2$ sind die nichtdiagonalen Null.\\
	\section{}
	wir haben eine diskrete gleichverteilung T aus $\{1,2,\dots,10\}$. jede minute treffen $Y_i$  $i\in\{1,2,\dots,10\}$ Fahrgäste ein $Y_i\sim B(5,0.2)$\\
	T und $Y_i$ sind stochastisch unabhängig.\\
	\[P_T(T=t)=\frac{1}{10}\text{ für alle }t\in\{1,2,\dots,10\}\]
	wegen gleichverteilung \\
	Sei Z die anzahl der Fahrgäste bis zum eintreffen des Busses.\\
	Der Bus kommt nach T minuten, also ist der Wert von Z abhängig von T
	\[Z = \sum^T_{t=1}Y_t\sim B(5,0.2)\]
	also $B(5,0.2) $ ist $5\pm 0.2$\\
	wir wollen jetzt die Bedingte Wahrscheinlichkeit
	\[P(Z=k|T=t)=b(5t,0.2,k)\]
	dazu gemeinsame Wahrscheinlichkeit.
	\[P(z=k,T=t) = P(Z=k|T=t)P_T(T=t) = b(5t,0.2,k)\frac{1}{10} = \frac{b(5t,0.2,k)}{10}\]
	b)\\
	\[E[Z|T=t]=t\cdot 5\cdot 0.2\]
	\[Var(z|T=t) = t\cdot 5\cdot (1-0.2)\]
	\[E[Z] = \sum^{10}_{t=1} E[Z|T=t]=\sum_{t=1}^{10}\frac{t}{10}\]
	\[Var(Z)=E(T)Var(Y_1)+Var(T)E(Y_1)^2\]
	geht wegen identischer verteilung von Z.









\end{document}