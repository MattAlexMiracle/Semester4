\documentclass{article}
\usepackage{listings}
\usepackage{mathrsfs}
\usepackage[utf8]{inputenc}
\usepackage{amssymb}
\usepackage{lipsum}
\usepackage{amsmath}
\usepackage{fancyhdr}
\usepackage{geometry}
\usepackage{scrextend}
\usepackage[english,german]{babel}
\usepackage{titling}
\setlength{\droptitle}{-3cm}
\usepackage{tikz}
\usepackage{algorithm,algpseudocode}
\usepackage[doublespacing]{setspace}
\usetikzlibrary{datavisualization}
\usetikzlibrary{datavisualization.formats.functions}
\usepackage{polynom}
\usepackage{amsmath}
\usepackage{gauss}
\usepackage{tkz-euclide}
\usepackage{minted}
\usetikzlibrary{datavisualization}
\usetikzlibrary{datavisualization.formats.functions}
\author{
Alexander Mattick Kennung: qi69dube\\
Kapitel 1
}
\usepackage{import}
\date{\today}
\geometry{a4paper, margin=2cm}
\usepackage{stackengine}
\parskip 1em
\newcommand\stackequal[2]{%
  \mathrel{\stackunder[2pt]{\stackon[4pt]{=}{$\scriptscriptstyle#1$}}{%
  $\scriptscriptstyle#2$}}
 }
\makeatletter
\renewcommand*\env@matrix[1][*\c@MaxMatrixCols c]{%
  \hskip -\arraycolsep
  \let\@ifnextchar\new@ifnextchar
  \array{#1}}
\makeatother
\lstset{
  language=haskell,
}
\lstnewenvironment{code}{\lstset{language=Haskell,basicstyle=\small}}{}
\usepackage{enumitem}
\setlist{nosep}
\usepackage{titlesec}
\usepackage{ stmaryrd }
\usepackage{verbatim}
\usepackage{tikz-qtree}
\usepackage{bussproofs}

\titlespacing*{\subsection}{0pt}{2pt}{3pt}
\titlespacing*{\section}{0pt}{0pt}{5pt}
\titlespacing*{\subsubsection}{0pt}{1pt}{2pt}
\title{Übung 7}


\begin{document}
	Das Gewicht einer Kiste ist $X\sim \mathcal{R}(105,135)$ verteilt.\\
	a)\\
	\[E(X) = \frac{135+105}{2} = 120kg\]
	\[Var(X) = \frac{(135-105)^2}{12} = 75kg^2\]
	b)\\
	Die Summe für $ES_{64} = E\sum_{i=1}^{64} X_i = 64\cdot \underbrace{\frac{1}{64} \sum_{i=1}^{64} X_i}_{EX} = 64\cdot 120kg = 7680kg$\\
	ist der Erwartungswert der Summe der 64 Kisten.\\
	Das ist gleich $64\cdot E(X_1)$ ist identisch st.u. also ist es egal welche Kiste speziell wir verwenden.\\
	$Var(S_{64}) =Var(\sum^{64}_{i=1} X_i) = 64Var(X_1) = 64\cdot 75kg^2 = 4875kg^2$\\
	Mit Hilfe der \underline{Chebychev-Markov Ungleichung} ($P(|X-\mu|\geq k)\leq \frac{\sigma^2}{k^2}$.\\
	\[P(7560kg\leq S_{64}\leq 7800kg) = P(7560kg-ES_{64}\leq S_{64}-ES_{64}\leq 7800-ES_{64}) =P(-120\leq S_{64}-7680\leq 120) \]
	also einfach um mittelwert verschieben
	\[= P(|S_{64}-ES_{64}|\leq 120)\]
	wir haben also effektiv einen Kreis mit Radius 120 um $ES_{64}$. Wir stellen um, um das äußere des Kreises zu erhalten
	\[1-P(|S_{64}-\underbrace{7680}_{ES_{64}}|>120)\]
	(WICHTIG $>$ und nicht $\geq$! sonst sind Ereignis/Gegenereignis nicht distinkt)\\
	\[\underbrace{1-P(|S_{64}-ES_{64}|>120)\geq 1-\frac{Var(S_{64})}{(EX)^2}}_{Chebychew-Markow\ mit\ k = ES_{64}}\]
	\[\underbrace{1-P(|S_{64}-ES_{64}|>120)}_{=P(|S_{64}-ES_{64}|\leq 120)} = 1-\frac{4875}{(120)^2}=\frac{2}{3}\]
	Also ist die Lösung $P(|S_{64}-ES_{64}|\leq 120) =\frac{2}{3}$\\
	c)\\
	Gesucht $P(S_{64}\leq 7800)$ Zentrale Grenzwertsatz (oder 64 mal falten\dots)\\
	Die Anwendung des Zentralen Grenzwertsatzes ergibt:
	\[\frac{S_{64}-ES_{64}}{\sqrt{Var(S_{64})}} = \frac{S_{64}-ES_{64}}{Str(S_{64})}\sim N(0,1)\]
	\[P(S_{64}\leq 7800) = P(\frac{S_{64}-ES_{64}}{Str(S_{64})}\leq \frac{7800-ES_{64}}{Var(S_{64})})\]
	(hier kann man sogar schön den Zusammenhang von Chebychew-Markow mit dem Zentralen Grenzwertsatz sehen)
	\[= \Phi(\frac{120}{Str(S_{64})}) = \]
	wobei $\Phi$ die FV der (standard)-Normalverteilung ist
	\[f(x)=(\frac{1}{\sqrt{2\pi}})^n\frac{1}{\sqrt{det(K)}}e^{\frac{-1}{2}(x-a)^TK^{-1}(x-a)\]
	

\end{document}
