\documentclass{article}
\usepackage{listings}
\usepackage{mathrsfs}
\usepackage[utf8]{inputenc}
\usepackage{amssymb}
\usepackage{lipsum}
\usepackage{amsmath}
\usepackage{fancyhdr}
\usepackage{geometry}
\usepackage{scrextend}
\usepackage[english,german]{babel}
\usepackage{titling}
\setlength{\droptitle}{-3cm}
\usepackage{tikz}
\usepackage{algorithm,algpseudocode}
\usepackage[doublespacing]{setspace}
\usetikzlibrary{datavisualization}
\usetikzlibrary{datavisualization.formats.functions}
\usepackage{polynom}
\usepackage{amsmath}
\usepackage{gauss}
\usepackage{tkz-euclide}
\usepackage{minted}
\usetikzlibrary{datavisualization}
\usetikzlibrary{datavisualization.formats.functions}
\author{
Alexander Mattick Kennung: qi69dube\\
Kapitel 1
}
\usepackage{import}
\date{\today}
\geometry{a4paper, margin=2cm}
\usepackage{stackengine}
\parskip 1em
\newcommand\stackequal[2]{%
  \mathrel{\stackunder[2pt]{\stackon[4pt]{=}{$\scriptscriptstyle#1$}}{%
  $\scriptscriptstyle#2$}}
 }
\makeatletter
\renewcommand*\env@matrix[1][*\c@MaxMatrixCols c]{%
  \hskip -\arraycolsep
  \let\@ifnextchar\new@ifnextchar
  \array{#1}}
\makeatother
\lstset{
  language=haskell,
}
\lstnewenvironment{code}{\lstset{language=Haskell,basicstyle=\small}}{}
\usepackage{enumitem}
\setlist{nosep}
\usepackage{titlesec}
\usepackage{ stmaryrd }
\usepackage{verbatim}
\usepackage{tikz-qtree}
\usepackage{bussproofs}

\titlespacing*{\subsection}{0pt}{2pt}{3pt}
\titlespacing*{\section}{0pt}{0pt}{5pt}
\titlespacing*{\subsubsection}{0pt}{1pt}{2pt}
\title{Übung 7}


\begin{document}
	\maketitle
	X$\sim EXP(1)$ und $Y\sim EXP(2)$ verteilt.\\
	$EX = \frac{1}{1}$ und $EY = \frac{1}{2}$ (logisch $\lambda$ ist die Erwartete Zahl Ereignisse pro Zeitintervall.)\\
	Die kovarianz ist definiert als $E((X-EX)(Y-EY))$ außerdem ist der Erwartungswert linear.\\
	$U=2X+3Y$ und $V=3X-Y$\\
	Die Varianz der Exponentialverteilung ist bekannt: $Var(EXP(\lambda))=\frac{1}{\lambda^2}$. Außerdem ist die Varianz additiv, wenn die zugrundeliegenden ZV st.u:\\
	\[Var(U) = Var(2X+3Y) \stackrel{X,Y\ st.u}{=} Var(2X)+Var(3Y) = 4Var(X)+9Var(Y) = 4\frac{1}{1}+\frac{9}{2^2} = 4+\frac{9}{4}=\frac{25}{4}\]
	ähnlich V
	\[Var(V) = Var(3X)+Var(-1Y) = 9Var(X)+Var(Y) = 9\cdot 1 +\frac{1}{4}=\frac{37}{4}\]
	Die Kovarianz ist definiert als 
	\[Kov(U,V) = EUV-EU\cdot EV\]
	Die Erwartungswerte sind linear
	\[EU = E(2X+3Y) = 2EX+3EY = \frac{2}{1}+3\frac{1}{2}\]
	\[EV = E(3X-Y) = 3EX-EY = \frac{3}{1}-\frac{1}{2}\]
	\[EUV = E((2X+3Y)(3X-Y))= E(6X^2+7XY-3Y^2)\stackrel{additivitaet}{=}6EX^2+\underbrace{7EXEY}_{X,Y, st.u}-3EY^2\]
	Wir benötigen also
	\[EX^2 = \int_{-\infty}^\infty x^2 f^X(x)dx = \int_0^\infty x^2 1e^{-1x}dx = [-x^2e^{-1x}]_0^\infty+\int_0^\infty2xe^{-1x}dx = [-x^2e^{-1x}]_0^\infty- ([2xe^{-x}]_0^\infty + \int_0^\infty2e^{-1x}dx )\]
	\[\underbrace{[-x^2e^{-1x}]_0^\infty}_{=0}- (\underbrace{[2xe^{-x}]_0^\infty}_{=0} + \underbrace{\int_0^\infty2e^{-1x}dx}_{[-2e^{-x}]_0^\infty =-2} ) = 2\]
	\[EY^2 =\int_{-\infty}^\infty y^2 f^Y(y)dy =\int_0^\infty y^2 2e^{-2y}dy =\]
	\[ 2([\frac{1}{-2}y^2e^{-2y}]_0^\infty +\int_0^\infty ye^{-2y} dy= 2([\frac{1}{-2}y^2e^{-2y}]_0^\infty +(\frac{1}{-2}ye^{-2y}]_0^\infty+\int_0^\infty \frac{1}{2}e^{-2y} dy))=\]
	\[ 2(\underbrace{[\frac{1}{-2}y^2e^{-2y}]_0^\infty}_{0} +(\underbrace{\frac{1}{-2}ye^{-2y}]_0^\infty}_{0}+\underbrace{\int_0^\infty \frac{1}{2}e^{-2y} dy}_{[-\frac{1}{4}e^{-2y}]_0^\infty =\frac{1}{4} })) = \frac{1}{2}\]
	Somit ist
	\[EUV = 6\cdot 2+7\cdot 1\cdot \frac{1}{2}-3\cdot\frac{1}{2}=14\]
	somit ist die Kovarianz:
	\[Kov(U,V) = EUV-EUEV = 14-\frac{7}{2}\cdot \frac{5}{2} = 14-\frac{35}{4}=\frac{21}{4}=5.25\]
	Die Korrelation ist die Normierte Kovarianz
	\[Korr(U,V)=\frac{Kov(U,V)}{std(U)std(V)} =\frac{Kov(U,V)}{\sqrt{Var(U)Var(V)}} = \frac{5.25}{\sqrt{\frac{37}{4}\frac{25}{4} }}\approx 0.6904\dots\]
	Die Quantile entstehen durch den Wert der jeweiligen FV
	\[F(y) = (1-e^{-2y})1_{y\geq 0}\stackrel{!}{=} quantil\]
	das 5\% quantil ist also
	$u_{5\%}$\\
	$F(y) = 0.05\iff (1-e^{-2y})1_{y\geq 0} = 0.05 \iff 1-e^{-2y}=0.05\iff e^{-2y}=0.95\iff y=\frac{ln(0.95)}{-2} \approx 0.02564\dots$\\
	$u_{75\%}$\\
	$F(y) = 0.75\iff (1-e^{-2y})1_{y\geq 0} = 0.75 \iff 1-e^{-2y}=0.75\iff e^{-2y}=0.25\iff y=\frac{ln(0.25)}{-2}\approx 0.69314\dots$\\
\end{document}
