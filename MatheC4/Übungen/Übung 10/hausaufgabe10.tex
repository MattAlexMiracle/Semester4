\documentclass{article}
\usepackage{listings}
\usepackage{mathrsfs}
\usepackage[utf8]{inputenc}
\usepackage{amssymb}
\usepackage{lipsum}
\usepackage{amsmath}
\usepackage{fancyhdr}
\usepackage{geometry}
\usepackage{scrextend}
\usepackage[english,german]{babel}
\usepackage{titling}
\setlength{\droptitle}{-3cm}
\usepackage{tikz}
\usepackage{algorithm,algpseudocode}
\usepackage[doublespacing]{setspace}
\usetikzlibrary{datavisualization}
\usetikzlibrary{datavisualization.formats.functions}
\usepackage{polynom}
\usepackage{amsmath}
\usepackage{gauss}
\usepackage{tkz-euclide}
\usepackage{minted}
\usetikzlibrary{datavisualization}
\usetikzlibrary{datavisualization.formats.functions}
\author{
Alexander Mattick Kennung: qi69dube\\
Kapitel 1
}
\usepackage{import}
\date{\today}
\geometry{a4paper, margin=2cm}
\usepackage{stackengine}
\parskip 1em
\newcommand\stackequal[2]{%
  \mathrel{\stackunder[2pt]{\stackon[4pt]{=}{$\scriptscriptstyle#1$}}{%
  $\scriptscriptstyle#2$}}
 }
\makeatletter
\renewcommand*\env@matrix[1][*\c@MaxMatrixCols c]{%
  \hskip -\arraycolsep
  \let\@ifnextchar\new@ifnextchar
  \array{#1}}
\makeatother
\lstset{
  language=haskell,
}
\lstnewenvironment{code}{\lstset{language=Haskell,basicstyle=\small}}{}
\usepackage{enumitem}
\setlist{nosep}
\usepackage{titlesec}
\usepackage{ stmaryrd }
\usepackage{verbatim}
\usepackage{tikz-qtree}
\usepackage{bussproofs}

\titlespacing*{\subsection}{0pt}{2pt}{3pt}
\titlespacing*{\section}{0pt}{0pt}{5pt}
\titlespacing*{\subsubsection}{0pt}{1pt}{2pt}
\title{Übung 7}


\begin{document}
	\maketitle
	X$\sim EXP(1)$ und $Y\sim EXP(2)$ verteilt.\\
	$EX = \frac{1}{1}$ und $EY = \frac{1}{2}$ (logisch $\lambda$ ist die Erwartete Zahl Ereignisse pro Zeitintervall.)\\
	Die kovarianz ist definiert als $E((X-EX)(Y-EY))$ außerdem ist der Erwartungswert linear.\\
	$U=2X+3Y$\\
	\[\int_{-\infty}^\infty\int_{-\infty}^\infty f^{(X,Y)}(x,y)(x-EX)(y-EY)dxdy\]
	\[\int_{0}^\infty\int_{0}^\infty 2\cdot e^{-x}3\cdot 2e^{-2y}(x-1)(y-\frac{1}{2})dxdy\]
	\[6\cdot\int_{0}^\infty(\int_{0}^\infty e^{-x}2e^{-2y}(x-1)(y-\frac{1}{2})dx)dy\]
	\[6\cdot\int_{0}^\infty 2e^{-2y}(y-\frac{1}{2})(\int_{0}^\infty e^{-x}(x-1)dx)dy\]
	\[6\cdot\int_{0}^\infty 2e^{-2y}(y-\frac{1}{2})([-e^{-x}(x-1)]_0^\infty-\int_{0}^\infty -e^{-x}dx)dy\]
	\[6\cdot\int_{0}^\infty 2e^{-2y}(y-\frac{1}{2})([-e^{-x}(x-1)]_0^\infty-[ -e^{-x}]_{0}^\infty)dy\]
	\[6\cdot\int_{0}^\infty 2e^{-2y}(y-\frac{1}{2})([\lim_{x\to\infty}-e^{-x}(x-1)-(-1)]-[\lim_{x\to\infty} -e^{-x}-(-1)])dy\]
	\[6\cdot\int_{0}^\infty 2e^{-2y}(y-\frac{1}{2})([0+1]-[1])dy\]
	\[6\cdot\int_{0}^\infty 2e^{-2y}(y-\frac{1}{2})(0)dy=0\]
	Da die ZV X und Y stochstisch unabhängig sind, ist ihre kovarianz 0:\\
	per definition hüber Seite 105 unten c) X,Y s.t.u. $\implies$ kov(X,Y) sind s.t.u.: X,Y s.t.u\\
	\[Var(X+Y) =Var(X)+Var(Y)\]
	im allgemeinen Fall (also bei nicht s.t.u) gilt
	\[Var(X+Y) = Var(X)+Var(Y)-2(EXY-EXEY)\]
	mit Vorfaktoren (hier a,b = 2,3)
	\[Var(aX+bY) =Var(aX)+Var(bY) =a^2Var(X)+b^2Var(Y)\]
	und im allgemeinen
	\[Var(aX+bY) = Var(aX)+Var(bY)-2(EaXbY-EaXEbY)\]
	aus der ersten und 2. Gleichung zusammen, folgt Kov(aX,bY) =0, wenn X,Y s.t.u\\
	Die Korrelation ist daher auch null: die Varianz von $Var(EXP(\lambda))=\frac{1}{\lambda^2}\implies std(EXP(\lambda))=\frac{1}{\lambda}$ existiert somit. Der zähler (=Kovarianz) ist null, somit ist Korrelation auch null\\
	im Fall V=3X-Y\\
	\[\int_{\mathbb{R}^2}3e^{-x}\cdot 2\cdot(-1)e^{-2y}(x-1)(y-\frac{1}{2})dxdy\]
	\[-6\int_{\mathbb{R}^2}e^{-x}e^{-2y}(x-1)(y-\frac{1}{2})dxdy\]
	\[-6\int_{0}^\infty e^{-2y}(y-\frac{1}{2})(\int_{0}^\infty e^{-x}(x-1))dxdy\]
	\[-6\int_{0}^\infty e^{-2y}(y-\frac{1}{2})\underbrace{(0)}_{vgl.\ oben}dy=0\]
	kovarianz gleiches spiel wie beim ersten.\\
	korrelation ist null, gleiches spiel wie beim ersten.\\


\end{document}
