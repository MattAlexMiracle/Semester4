\documentclass{article}
\usepackage{listings}
\usepackage{mathrsfs}
\usepackage[utf8]{inputenc}
\usepackage{amssymb}
\usepackage{lipsum}
\usepackage{amsmath}
\usepackage{fancyhdr}
\usepackage{geometry}
\usepackage{scrextend}
\usepackage[english,german]{babel}
\usepackage{titling}
\setlength{\droptitle}{-3cm}
\usepackage{tikz}
\usepackage{algorithm,algpseudocode}
\usepackage[doublespacing]{setspace}
\usetikzlibrary{datavisualization}
\usetikzlibrary{datavisualization.formats.functions}
\usepackage{polynom}
\usepackage{amsmath}
\usepackage{gauss}
\usepackage{tkz-euclide}
\usepackage{minted}
\usetikzlibrary{datavisualization}
\usetikzlibrary{datavisualization.formats.functions}
\author{
Alexander Mattick Kennung: qi69dube\\
Kapitel 1
}
\usepackage{import}
\date{\today}
\geometry{a4paper, margin=2cm}
\usepackage{stackengine}
\parskip 1em
\newcommand\stackequal[2]{%
  \mathrel{\stackunder[2pt]{\stackon[4pt]{=}{$\scriptscriptstyle#1$}}{%
  $\scriptscriptstyle#2$}}
 }
\makeatletter
\renewcommand*\env@matrix[1][*\c@MaxMatrixCols c]{%
  \hskip -\arraycolsep
  \let\@ifnextchar\new@ifnextchar
  \array{#1}}
\makeatother
\lstset{
  language=haskell,
}
\lstnewenvironment{code}{\lstset{language=Haskell,basicstyle=\small}}{}
\usepackage{enumitem}
\setlist{nosep}
\usepackage{titlesec}
\usepackage{ stmaryrd }
\usepackage{verbatim}
\usepackage{tikz-qtree}
\usepackage{bussproofs}

\titlespacing*{\subsection}{0pt}{2pt}{3pt}
\titlespacing*{\section}{0pt}{0pt}{5pt}
\titlespacing*{\subsubsection}{0pt}{1pt}{2pt}
\title{Übung 7}


\begin{document}
	\maketitle
	a)\\
	$P(X=1) = \sum_{y\in\{-1,0,1\}}f^{(X,Y)}(i,j)=\frac{1}{4}+\frac{1}{6}+\frac{1}{12}= \frac{1}{2}$\\
	$P(X=c) = \sum_{y\in\{-1,0,1\}}f^{(X,Y)}(i,j)=\frac{1}{12}+\frac{1}{12}+\frac{1}{12}= \frac{3}{12}=\frac{1}{4}$\\
	$P(X=0) = 1-(P(X=c)+P(X=1)) = \frac{3}{4}$\\
	$EX = \sum_{i=0}^c i\cdot p(x_i) = 0\cdot p(x=0)+1\cdot p(x=1)+c\cdot p(x=c) = 0+\frac{1}{2}+\frac{c}{4} =\frac{2+c}{4}$\\
	für $EX=1$ muss $c=2$ gewählt werden.\\
	b)\\
	$P(Y>0|x>0)$ dazu bedingte wahrscheinlichkeit $P(B|A)=\frac{P(A\cap B)}{P(A)}$\\
	somit $P(Y>0|x>0)=\frac{P(X>0\land Y>0)}{P(X>0)}$\\
	nach Tabelle $\frac{P(X=1,Y=1)+P(X=1,Y=c)}{P(X=1)+P(X=c)}=\frac{\frac{1}{12}+\frac{2}{12}}{\frac{1}{2}+\frac{1}{4}}=\frac{2}{9}$\\
	c)\\
	$Z=XY$\\
	$EZ=E(XY)=\sum_{i,j,f^{(X,Y)}(i,j)>0}ijf^{(X,Y)}(i,j) = -\frac{1}{4}+\frac{1}{12}-c\frac{1}{12}+c\frac{1}{12} = -\frac{1}{6}$\\
	$Var(Z) = E(Z^2)-(EZ)^2$\\
	$E(Z^2) = E(X^2Y^2) = \sum_{i,j,f^{(X,Y)}(i,j)>0}i^2j^2f^{(X,Y)}(i,j) = \frac{1}{4}+\frac{1}{12}+c^2\frac{1}{12}+c^2\frac{1}{12} = \frac{c^2}{6}+\frac{1}{3} = \frac{c^2+2}{6}$\\
	$(EX)^2 = (-\frac{1}{6})^2 = \frac{1}{36}$\\
	$Var(Z)=\frac{c^2+2}{6}-\frac{1}{36}$\\
	$Var(Z)\stackrel{!}{=}1\iff c^2 = \frac{25}{\sqrt{6}}\implies c=\frac{5\sqrt{6}}{6}$ da $c>1$ sein muss fällt die negative Lösung weg.\\
	d)\\
	$EY=-\frac{1}{6}$\\
	$EY = \sum_{j=-1}jp(Y=j)=-1p(y=-1)+0p(Y=0)+1p(Y=1)=-p(Y=-1)+p(Y=1)=-\frac{1}{6}\implies p(Y=-1)=\frac{1}{6}+p(Y=1)$\\
	$p(Y=-1)=\frac{1}{6}+\sum_{x\in\{0,1,c\}} f^{(X,Y)}(j,-1)=\frac{1}{6}+\frac{1}{12}+\frac{1}{12}+\frac{1}{12}=\frac{5}{12}$\\
	$P(X=0,Y=-1)=P(y=-1)-(f^{(X,Y)}(1,1)+f^{(X,Y)}(1,c)) = \frac{5}{12}$\\
	$1=\sum_{(i,j)\in\Omega} f^{(X,y)}(i,j)$\\
	e)\\
	$$Kov(X,Y)=E(XY)-EXEY = -\frac{1}{6}-(\frac{2+c}{4})(-\frac{1}{6}) =-\frac{1}{6}(\frac{1}{2}-\frac{1}{4}c)$$
	unkorreliert $Kov(X,Y) =0\iff -\frac{1}{6}(\frac{1}{2}-\frac{1}{4}c)=0\iff c=2$.\\
	\section{Aufgabe 2}
	der verkaufspreis muss größer als der einkaufspreis sein$P_1<P_2$\\
	X=demanded quantity, Q=ordered quantity\\
	um über/unter-produktion zu vermeiden, match von X und Q $min(X,Q)$\\
	profit ist $G(X)=p_2\cdot min(X,Q)-p_1\cdot q$\\
	Wenn $X\sim EXP(\lambda)$, dann wird der erwartete Profit
	$Y=G(X)$\\
	\[g(q) = EY = E(G(X)) = \int_{-\infty}^\infty G(x)f(x)dx = \int_{-\infty}^\infty (p_2min(x,q)-p_1q)(x)dx \]
	\[ p_2\int_{-\infty}^\infty min(x,q)f(x)dx-p_1q\int_{-\infty}^\infty f(x) dx\]
	\[ p_2\int_{-\infty}^\infty min(x,q)\lambda e^{-\lambda x}dx-p_1q\int_{-\infty}^\infty \lambda e^{-\lambda x} dx\]
	\[ p_2\int_{0}^\infty min(x,q)\lambda e^{-\lambda x}dx-p_1q\cdot 1\]
	\[ p_2\underbrace{\int_{0}^\infty min(x,q)\lambda e^{-\lambda x}dx}_{einzeln}-p_1q\cdot 1\]
	\[\int_{0}^\infty min(x,q)\lambda e^{-\lambda x}dx = \int_0^q\underbrace{x\lambda e^{-\lambda x}}_{uuuuh\ Feynman's\ Trick}dx+\int_q^\infty q\lambda e^{-\lambda x}dx =  \frac{1}{\lambda}(1-e^{-\lambda q})\]
	$g(q) = \frac{p_2}{\lambda}(1-e^{-\lambda q})-p_1q$\\
	damit haben wir g(q) modelliert. Natürlich wollen wir den gewinn maximieren!\\
	\[g'(q)\stackrel{!}{=} 0\iff p_2e^{-\lambda q}-p_1=0\iff p_2e^{-\lambda q}=p_1\iff q = \frac{ln(\frac{p_1}{p_2})}{\lambda}\]
	Wenn $\lambda =0.001$\\
	$U_{2,5\%} = F^{-1}(0.025) = -\frac{1}{\lambda}(1-0.025)\approx 25$\\
	$U_{95\%} = F^{-1}(0.95) = -\frac{1}{\lambda}(1-0.95)\approx 2995$\\
	WICHTIG FÜR KLAUSUR\\
	relationen zwischen: F(x), f(x), P(x), E
\end{document}
