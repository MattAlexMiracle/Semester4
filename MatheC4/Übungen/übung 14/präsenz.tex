\documentclass{article}
\usepackage{listings}
\usepackage{mathrsfs}
\usepackage[utf8]{inputenc}
\usepackage{amssymb}
\usepackage{lipsum}
\usepackage{amsmath}
\usepackage{fancyhdr}
\usepackage{geometry}
\usepackage{scrextend}
\usepackage[english,german]{babel}
\usepackage{titling}
\setlength{\droptitle}{-3cm}
\usepackage{tikz}
\usepackage{algorithm,algpseudocode}
\usepackage[doublespacing]{setspace}
\usetikzlibrary{datavisualization}
\usetikzlibrary{datavisualization.formats.functions}
\usepackage{polynom}
\usepackage{amsmath}
\usepackage{gauss}
\usepackage{tkz-euclide}
\usepackage{minted}
\usepackage{mathrsfs}
\usetikzlibrary{datavisualization}
\usetikzlibrary{datavisualization.formats.functions}
\author{
Alexander Mattick Kennung: qi69dube\\
Kapitel 1
}
\usepackage{import}
\date{\today}
\geometry{a4paper, margin=2cm}
\usepackage{stackengine}
\parskip 1em
\newcommand\stackequal[2]{%
  \mathrel{\stackunder[2pt]{\stackon[4pt]{=}{$\scriptscriptstyle#1$}}{%
  $\scriptscriptstyle#2$}}
 }
\makeatletter
\renewcommand*\env@matrix[1][*\c@MaxMatrixCols c]{%
  \hskip -\arraycolsep
  \let\@ifnextchar\new@ifnextchar
  \array{#1}}
\makeatother
\lstset{
  language=haskell,
}
\lstnewenvironment{code}{\lstset{language=Haskell,basicstyle=\small}}{}
\usepackage{enumitem}
\setlist{nosep}
\usepackage{titlesec}
\usepackage{ stmaryrd }
\usepackage{verbatim}
\usepackage{tikz-qtree}
\usepackage{bussproofs}

\titlespacing*{\subsection}{0pt}{2pt}{3pt}
\titlespacing*{\section}{0pt}{0pt}{5pt}
\titlespacing*{\subsubsection}{0pt}{1pt}{2pt}
\title{Übung 7}


\begin{document}
	\maketitle
	\[L(x_1,\dots,x_n,\mu,\sigma^2) = \prod^n_{i=1}f^{X_i}(x_i,\mu,\sigma^2)\]
	hier
	\[L(x_1,\dots,x_n,\mu,\sigma^2) =(\frac{1}{\sqrt{2\pi\sigma^2}})^n\prod^n_{i=1}\exp(\frac{1}{2\sigma^2}(x_i-\mu)^2)\]
	jetzt log-likelyhodd
	\[\ln(L(x_1,\dots,x_n,\mu,\sigma^2)) =\frac{-n}{2}ln(2\pi)-\frac{-n}{2}ln(\sigma^2)+\sum^n_{i=1}\frac{-1}{2\sigma^2}(x_i-\mu)^2\]
	maximieren:\\
	ableiten
	\[\implies \frac{\partial}{\partial\mu} \ln(L(x_1,\dots,x_n,\mu,\sigma^2)) =\frac{1}{\sigma^2}\sum^n_{i=1}(x_i-\mu)^2\]
	\[\implies \frac{\partial}{\partial\sigma^2} \ln(L(x_1,\dots,x_n,\mu,\sigma^2)) =-\frac{-n}{2\sigma^2}+\frac{1}{2\sigma^4}\sum^n_{i=1}(x_i-\mu)^2\]
	\[\implies \mu_{ML} =\frac{1}{n}\sum^n_{i=1}x_i=\bar x\]
	\[\implies \sigma_{ML} =\frac{1}{n}\sum^n_{i=1}(x_i-\mu_{ML})^2=\frac{1}{n}\sum^n_{i=1}(x_i-\bar x)^2\]
	Hess($\ln(L(x_1,\dots,x_n,\mu,\sigma^2))$)\\
	$\begin{bmatrix} -n/\sigma^2&-\frac{1}{\sigma^4}\sum(x_i-\mu)^2\\-\frac{1}{\sigma^4}\sum(x_i-\mu)^2& \end{bmatrix}$\\
	b) Schätzer für $\sigma^2$ Erwartungstreu machen.\\
	\[\sigma_{ML} = \frac{1}{n}\sum^n_{i=1} (x_i-\bar x)^2= \frac{1}{n}\sum^n_{i=1} x_i^2-(\bar x)^2\]
	\[\implies E(\sigma_{ML}) = \frac{1}{n}\sum^n_{i=1} E(x_i^2)-E(\bar x^2) \]
	wir wissen im allgemeinen
	\[E(x_i^2) = Var(x_i)+(E(x_i))^2\]
	daraus folgt auch
	\[E(\bar x_i^2) = Var(\bar x)+(E(\bar x))^2 = \frac{1}{n^2}nVar(x_1)+\mu^2\]
	aus $ \frac{1}{n}\sum^n_{i=1} E(x_i^2)-E(\bar x^2)$ und einsetzen folgt
	\[E(\sigma_{ML}) = \frac{n-1}{n}\sigma^2\]
	\[s^2 = \frac{n}{n-1} \sigma = \frac{1}{n-1}\sum^n_{i=1}(x_i-\bar x)^2\]
	\[\implies E(S^2) = \frac{n}{n-1} E(\sigma_{ML})\]
	\[\implies E(S^2) = \frac{n}{n-1}  \frac{n-1}{n}\sigma^2 = \sigma^2\]
	a) Gesucht ist ein $\epsilon>0$ mit
	\[P(-\infty <\mu\leq \bar x+\epsilon) = 1-\alpha\]
	Punktschätzer für $\mu$
	\[T_\mu (x_1,\dots,x_n)=\frac{1}{n} \sum^n_{i=1} x_i = \bar x\]
	$x_1,\dots, x_n$ sind $N(\mu,\sigma^2)$-verteilt $\bar x\sim N(\mu,\sigma^2/n)$\\
	\[P(-\infty\leq \mu\leq \bar x+\epsilon) = P(\mu\leq \bar x+\epsilon) = P(\bar x-\mu\geq -\epsilon) = P(\frac{\bar x-\mu}{\sqrt{\sigma^2/n}}\geq -\frac{\epsilon}{\sqrt{\sigma^2/n}})\]
	\[= P(Z \geq -\frac{\epsilon}{\sqrt{\sigma^2/n}}) = 1-P(Z < -\frac{\epsilon}{\sqrt{\sigma^2/n}})\]
	mit Standardisierter ZV $Z\sim N(0,1)$\\
	\[ 1-P(Z < -\frac{\epsilon}{\sqrt{\sigma^2/n}}) =  1-P(Z < -\frac{\epsilon\sqrt{n}}{\sigma})\]
	\[= 1-\Phi(-\frac{\epsilon\sqrt{n}}{\sigma}) = 1-\alpha\]
	das $1-\alpha$-Quantil $Z_{(1-\alpha)}=\frac{\epsilon\sqrt{n}}{\sigma}$\\
	$\implies \epsilon = Z_{(1-\alpha)}/\sqrt{n}$\\
	$K I_\mu = (-\infty, X+\epsilon] = (-\infty, X+Z_{(1-\alpha)}/\sqrt{n}]$\\
	$K I_\mu = (-\infty, \bar x + z_{0.95}\sigma/\sqrt{n}] = (-\infty, 2.648 + 1.644\cdot 2/\sqrt{11}] = (-\infty,3.640]$\\
	zweiseitig
	\[K I = [\bar x-z_{(1-\alpha/2)}\cdot\sigma/\sqrt{n},\bar x+z_{(1-\alpha/2)}\cdot\sigma/\sqrt{n}]\]
	\[K I = [\bar x-z_{(0.975)}\cdot\sigma/\sqrt{n},\bar x+z_{(0.975)}\cdot\sigma/\sqrt{n}]\]
	\[K I = [\bar x-z_{(0.975)}\cdot\sigma/\sqrt{n},\bar x+z_{(0.975)}\cdot\sigma/\sqrt{n}]\]
	\[K I = [2.644-1.9600\cdot2/\sqrt{11},2.644+1.9600\cdot2/\sqrt{11}]\]
	\[K I 0 [1.4663,3.830]\]

\end{document}
