\documentclass{article}
\usepackage{listings}
\usepackage{mathrsfs}
\usepackage[utf8]{inputenc}
\usepackage{amssymb}
\usepackage{lipsum}
\usepackage{amsmath}
\usepackage{fancyhdr}
\usepackage{geometry}
\usepackage{scrextend}
\usepackage[english,german]{babel}
\usepackage{titling}
\setlength{\droptitle}{-3cm}
\usepackage{tikz}
\usepackage{algorithm,algpseudocode}
\usepackage[doublespacing]{setspace}
\usetikzlibrary{datavisualization}
\usetikzlibrary{datavisualization.formats.functions}
\usepackage{polynom}
\usepackage{amsmath}
\usepackage{gauss}
\usepackage{tkz-euclide}
\usetikzlibrary{datavisualization}
\usetikzlibrary{datavisualization.formats.functions}
\author{
Alexander Mattick Kennung: qi69dube\\
Kapitel 1
}
\usepackage{import}
\date{\today}
\geometry{a4paper, margin=2cm}
\usepackage{stackengine}
\parskip 1em
\newcommand\stackequal[2]{%
  \mathrel{\stackunder[2pt]{\stackon[4pt]{=}{$\scriptscriptstyle#1$}}{%
  $\scriptscriptstyle#2$}}
 }
\makeatletter
\renewcommand*\env@matrix[1][*\c@MaxMatrixCols c]{%
  \hskip -\arraycolsep
  \let\@ifnextchar\new@ifnextchar
  \array{#1}}
\makeatother
\lstset{
  language=haskell,
}
\lstnewenvironment{code}{\lstset{language=Haskell,basicstyle=\small}}{}
\usepackage{enumitem}
\setlist{nosep}
\usepackage{titlesec}

\titlespacing*{\subsection}{0pt}{2pt}{3pt}
\titlespacing*{\section}{0pt}{0pt}{5pt}
\titlespacing*{\subsubsection}{0pt}{1pt}{2pt}
\title{Vorlesung 4}


\begin{document}
	\section{Aufgabe 20}
	a) Bekannt:\\
	A,B sind stochastisch unabh. also $P(A\cap B)=P(A)P(B)$\\
	außerdem gilt nach bedingter wahrscheinlichkeit: $P(A|B)P(B) = P(A\cap B)$, also $P(A|B)=P(A)$,\\
	bzw $P(B|A)P(A)=P(A\cap B)\iff P(B) =P(B|A)$\\
	zZ.: $P(A^c\cap B)=P(A^c)P(B)$\\
	allgemein gilt $P(A^c\cap B) = P(B|A^c)P(A^c)$ für alle wahrscheinlichkeiten (abhängig oder unabhängig)\\
	$P(A^c\cap B)=P(A^c|B)P(B)\stackrel{\text{\tiny bedingte wahrscheinlichkeit ist auch ein W-Maß}}{\iff} P(A^c\cap B) = (1-P(A|B))P(B)\\\stackrel{\scriptscriptstyle\text{stoch. unabh.} P(A|B)=P(A)}{\iff} P(A^c\cap B) = (1-P(A))P(B)\iff P(A^c\cap B)=P(A^c)P(B)$\\
	b) mit drei\\
	
	$P(A\cap B\cap C)= P(A)P(B)P(C)$ ist bekannt.\\
	Daraus folgt $P(A\cap B\cap C)= P(A)P(B)P(C)= P(A)P(B\cap C) = P(A|B\cap C)P(B\cap C)$, also $P(A|B\cap C)=P(A)$\\
	\textbf{(1)} Es gilt hier, dass auch $P(B),P(C),P(A)$ paarweise unabhängig sind:\\
	$P(A\cap B\cap C)= P(A)P(B)P(C) = P(A)P(B|C)P(C) = P(A)P(B\cap C)\implies P(B)P(C)=P(B\cap C)$\\
	 (analog für die anderen paare).\\
	\begin{comment}
	$P(A^c\cap B^c\cap C) = P(A^c|B^c\cap C) P(B^c\cap C)\iff P(A^c\cap B^c\cap C) =P((A^c|C)|B^c)P(B^c\cap C)\stackrel{\text{\tiny Paarweise unabhängig und a)}}{\iff}  P(A^c\cap B^c\cap C) =P(A^c|B^c)P(B^c\cap C) \iff P(A^c\cap B^c\cap C) = (1-P(A)) P(B^c\cap C) \iff P(A^c\cap B^c\cap C) = P(A^C) P(B^c\cap C)$ aus 1. wissen wir, dass $P(B^c\cap C)=P(B^c)P(C)$ gilt, wenn B und C stochastisch unabhängig sind (gilt nach \textbf{(1)}). Also folgt $P(A^c\cap B^c\cap C) = P(A^c) P(B^c\cap C)\iff P(A^c\cap B^c\cap C) = P(A^c) P(B^c)P(C)$
	\end{comment}
	(1) $(A\cup B)$ und $C$ sind stochastisch unabhängig:\\
	$P((A\cup B)\cap C) = P((A\cap C)\cup (B\cap C)) = P(A\cap C)+P(B\cap C)-P(A\cap B\cap C) =P(A)P(C)+P(B)P(C)-P(A)P(B)P(C) = P(C)(P(A)+P(B)-P(A)P(B))=P(C)P(A\cup B) $\\
	Daraus folgt $P(A^c\cap B^c\cap C) = P((A\cup B)^c\cap C) = P((A\cup B)^c) * P(C)$ ist nach a) auch linear unabhängig, wenn $P(A\cup B)P(C)$ unabhängig ist, was bei (1) bewiesen wurde. Somit sind die drei werte auch stochastisch unabhängig\\
	\section{Hausaufgabe 21}
	Es ist bekannt:	Sei F=``Falsche Dosierung'' und $P(F)=1\%$\\
	Heilung $P(A|F^c)=80\%$,\\
	Nebenwirkung $P(B|F^c)=30\%$\\
	(sprich ``unter der Bedingung, dass es keinen defekt gibt, ist Heilung/Nebenwirkung X wahrscheinlich'')\\
	Bei fehlerhaften gilt, dass die Heilwirkung nur in $P(A|F)=20\%$ und $P(B|F)=50\%$\\
	Dies sind bedingte wahrscheinlichkeiten, da die Wirkungszahlen unter dem Vorwissen, dass die Tabletten defekt sind, ermittelt wurden. (man nimmt eine Tablette, sieht dass sie defekt ist, und testet ihre Wirkung)
	Gesucht ist $P(A|B)$, $P(A|B^c)$\\
	%$P(B)$ ist mit der totalen wahrscheinlichkeit ausrechenbar. Dafür werden $P(B\cap \dots)$ benötigt.\\
	%$P(B\cap F^c)= P(B|F^c)(P(F^c)) = P(B|F^c)(1-P(F)) = 0.3*0.99 = 0.297$\\
	%$P(B\cap F) = P(B|F)P(F) = 0.5*0.01=0.005$\\
	%$P(B) = P(B\cap F)+P(B\cap F^c) = 0.297+0.005=0.302$\\
	$P(A\cap F) = P(A|F)P(F)=0.2*0.01 = 0.002$\\
	$P(A\cap F^c) = P(A|F^c)P(F^c) = 0.8 * 0.99 = 0.792$\\
	$P(A)=P(A\cap F)+P(A\cap F^c) = 0.002+0.792 = 0.794$\\
	Stichwort ``unter der Bedingung'':\\
	$P(A|B) = P(A)$, da das Medikament unabhängig voneinander zwei Wirkungen hat. (bei stoch. unabhängigkeit gilt $P(A)=P(A|B) = P(A|B^c)$, beweis vgl aufgabe 1)\\
	Also auch $P(A|B^C) = P(A)$
\end{document}