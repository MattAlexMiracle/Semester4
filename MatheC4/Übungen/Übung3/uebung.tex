\documentclass{article}
\usepackage{listings}
\usepackage{mathrsfs}
\usepackage[utf8]{inputenc}
\usepackage{amssymb}
\usepackage{lipsum}
\usepackage{amsmath}
\usepackage{fancyhdr}
\usepackage{geometry}
\usepackage{scrextend}
\usepackage[english,german]{babel}
\usepackage{titling}
\setlength{\droptitle}{-3cm}
\usepackage{tikz}
\usepackage{algorithm,algpseudocode}
\usepackage[doublespacing]{setspace}
\usetikzlibrary{datavisualization}
\usetikzlibrary{datavisualization.formats.functions}
\usepackage{polynom}
\usepackage{amsmath}
\usepackage{gauss}
\usepackage{tkz-euclide}
\usetikzlibrary{datavisualization}
\usetikzlibrary{datavisualization.formats.functions}
\author{
Alexander Mattick Kennung: qi69dube\\
Kapitel 1
}
\usepackage{import}
\date{\today}
\geometry{a4paper, margin=2cm}
\usepackage{stackengine}
\parskip 1em
\newcommand\stackequal[2]{%
  \mathrel{\stackunder[2pt]{\stackon[4pt]{=}{$\scriptscriptstyle#1$}}{%
  $\scriptscriptstyle#2$}}
 }
\makeatletter
\renewcommand*\env@matrix[1][*\c@MaxMatrixCols c]{%
  \hskip -\arraycolsep
  \let\@ifnextchar\new@ifnextchar
  \array{#1}}
\makeatother
\lstset{
  language=haskell,
}
\lstnewenvironment{code}{\lstset{language=Haskell,basicstyle=\small}}{}
\usepackage{enumitem}
\setlist{nosep}
\usepackage{titlesec}

\titlespacing*{\subsection}{0pt}{2pt}{3pt}
\titlespacing*{\section}{0pt}{0pt}{5pt}
\titlespacing*{\subsubsection}{0pt}{1pt}{2pt}
\title{Blatt 3}


\begin{document}
	\section{Hausaufgabe 17}
	a) $P(A)=\sum P(A\cap B_i)\implies P(A)=0 = \sum P(A\cap B_i)$, da $P(A\cap B_i)\geq 0$ muss $P(A\cap B_i)$ auch null sein. (Bzw $A\cap B\subseteq A$)\\
	b) $P(A)=\frac{1}{2}$, $P(B)=\frac{1}{3}$ folgt daraus disjunkt $A\cap B = \emptyset$\\
	Nein, Würfel $\Omega = \{1,2,3,4,5,6\}$ $P(A):= \frac{|A|}{|\Omega|}$\\
	$A=\{\omega\in\Omega| \omega=0\ (mod 2)\}$, $B=\{\omega\in\Omega| \omega<3\}$\\
	$P(A) = \frac{3}{6}$\\
	$P(B)=\frac{2}{6}$\\
	aber $A\cap B  = \{2\}\neq \emptyset$\\
	c) Nein, Würfel: $\Omega = \{1,2,3,4,5,6\}$\\
	$A= \{1,2\}$, $B=\{1,2,3,4\}$\\
	$P(A)=\frac{2}{6}$, $P(B)=\frac{4}{6}$\\
	$P(A)=P(B^c)$, aber $A^c = \{3,4,5,6\}$\\
	\section{Hausaufgabe 18}
	Ereignis A: Positives Ergebnis\\
	Ereignis B: Person hat Krebs\\
	Bekannt ist:\\
	$P(A|B) = 0.8$(also der Test sagt bei leuten, die Brustkrebs haben zu $80\%$ ``JA'')\\
	$P(A|B^c) = 0.096$ (also, die wahrscheinlichkeit, dass eine nicht kranken Person einen Positiven Test hat)\\
	$P(B)=0.01$ ist die wahrscheinlichkeit, dass eine Person krebs hat.\\
	ges $P(B|A)$, $P(A|B^C)$\\
	satz von Bayes $P(B|A) = \frac{P(A\cap B}{P(A)} = \frac{P(B)P(A|B)}{P(B)P(A|B)+P(B^c)P(A|B^c)}$ (genauigkeits Test)\\
	Wir folgern $P(B^c)=1-P(B)=0.99$\\
	$P(B|A) = \frac{P(B)P(A|B)}{P(B)P(A|B)+P(B^c)P(A|B^c)}=\frac{0.01*0.8}{0.01*0.8+0.99*0.096}\approx 0.078$\\
	\section{Hausaufgabe 19}
	a) Die Losnummer $3494$ wird gezogen\\
	Sei ``N = zahl N gezogen''\\
	$P(3)P(4|3)P(9|4,3)P(4|9,4,3)$\\
	$\frac{4}{40}*\frac{4}{39}\frac{4}{38}\frac{3}{37}$\\
	b) nur 1 und 7 dürfen vorkommen (beides mindestens 1 mal)\\
	Menge aller Losnummer aus 1 und 7.\\
	$\frac{4*4*3*3}{}$\\
	c) $\frac{40*36*32*26}{40*39*38*37}$
\end{document}