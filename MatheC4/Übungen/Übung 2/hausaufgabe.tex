\documentclass{article}
\usepackage{listings}
\usepackage{mathrsfs}
\usepackage[utf8]{inputenc}
\usepackage{amssymb}
\usepackage{lipsum}
\usepackage{amsmath}
\usepackage{fancyhdr}
\usepackage{geometry}
\usepackage{scrextend}
\usepackage[english,german]{babel}
\usepackage{titling}
\setlength{\droptitle}{-3cm}
\usepackage{tikz}
\usepackage{algorithm,algpseudocode}
\usepackage[doublespacing]{setspace}
\usetikzlibrary{datavisualization}
\usetikzlibrary{datavisualization.formats.functions}
\usepackage{polynom}
\usepackage{amsmath}
\usepackage{gauss}
\usepackage{tkz-euclide}
\usetikzlibrary{datavisualization}
\usetikzlibrary{datavisualization.formats.functions}
\author{
Alexander Mattick Kennung: qi69dube\\
Kapitel 1
}
\usepackage{import}
\date{\today}
\geometry{a4paper, margin=2cm}
\usepackage{stackengine}
\parskip 1em
\newcommand\stackequal[2]{%
  \mathrel{\stackunder[2pt]{\stackon[4pt]{=}{$\scriptscriptstyle#1$}}{%
  $\scriptscriptstyle#2$}}
 }
\makeatletter
\renewcommand*\env@matrix[1][*\c@MaxMatrixCols c]{%
  \hskip -\arraycolsep
  \let\@ifnextchar\new@ifnextchar
  \array{#1}}
\makeatother
\lstset{
  language=haskell,
}
\lstnewenvironment{code}{\lstset{language=Haskell,basicstyle=\small}}{}
\usepackage{enumitem}
\setlist{nosep}
\usepackage{titlesec}

\titlespacing*{\subsection}{0pt}{2pt}{3pt}
\titlespacing*{\section}{0pt}{0pt}{5pt}
\titlespacing*{\subsubsection}{0pt}{1pt}{2pt}
\title{Vorlesung 4}


\begin{document}
	\maketitle
	\section{Hausaufgabe 12}
	a) $\Omega = \{(x,y)|x,y\in \mathbb{R},\sqrt{ x^2+y^2}\leq r\}$\\
	b) \\
	$A=\{\omega=(x,y)|\sqrt{x^2+y^2}<1\}$\\
	$B=\{\omega=(x,y)|x\geq 0, y\geq 0\}$\\
	$C=\{\omega=(x,y)|\sqrt{x^2+y^2}>0.5\}$
	\section{Hausaufgabe 13}
	$A\cap B= \{(0,1),(1,0),(1,1)\}\cap \{(0,0),(0,1)\}=\{(0,1)\}\notin \mathcal{E}$\\
	Also ist $\mathcal{E}$ keine $\sigma$-Algebra.\\
	Eine option wäre natürlich $\mathcal{A} = \mathcal{P}(\Omega)$ (dies ist trivialerweise immer eine $\sigma$-Algebra).\\
	Es geht allerdings auch kleiner:\\
	Wir fügen successive fehlende Elemente hinzu: zuerst also $\{(0,1)\}$\\
	$\mathcal{E}' = \{\emptyset, \Omega, A,B,\{(0,1)\}\}$\\
	Jetzt benötigen wir noch die Negation von $C_\Omega \{(0,1)\} = \{(0,0),(1,0),(1,1)\}$:\\
	$\mathcal{E}'' = \{\emptyset, \Omega, A,B,\{(0,1)\}, \{(0,0),(1,0),(1,1)\}\}$\\
	Die Vereinigung von $B\cup\{0,1\}=B, B\cup\{(0,0),(1,0),(1,1)\}=\Omega$.\\
	Die Vereinigung von $A\cup\{0,1\}=\Omega$, $A\cup\{(0,0),(1,0),(1,1)\}=\Omega$.\\
	Die Vereinigungen mit $\emptyset$ und $\Omega$ sind trivialerweise die Menge selbst, bzw $\Omega$.\\
	Die Schnitte sind $A\cap\{0,1\}=\{(0,1)\}\in\mathcal{E}'$, $A\cap\{(0,0),(1,0),(1,1)\}=\{(1,0),(1,1)\}$.\\
	und von $B\cap\{0,1\}=\{0,1\}\in\mathcal{E}'$ $B\cap\{(0,0),(1,0),(1,1)\}=\{0,0\}$\\
	Aufnahme von $\{(1,0),(1,1)\}$ und $\{(0,0)\}$:\\
	\[\mathcal{E}''' = \{\emptyset, \Omega, A,B,\{(0,1)\}, \{(0,0),(1,0),(1,1)\},\{(1,0),(1,1)\},\{(0,0)\}\}\]\\
	$A\cap \{(1,0),(1,1)\} = \{(1,0),(1,1)\}$ $B\cap \{(1,0),(1,1)\} = \emptyset$\\
	$A\cap \{(0,0)\} = \emptyset$ $B\cap \{(0,0)\} = \{(0,0)\}\in\mathcal{E}'''$\\
	Vereinigungen $A\cup\{(1,0),(1,1)\}=A$ und $B\cup\{(1,0),(1,1)\}=\emptyset$\\
	Vereinigungen $A\cup\{(0,0)\}=\Omega$ und $B\cup\{(0,0)\}=B$\\
	Komplemente $\overline{\{(0,0)\}} = A\in\mathcal{E}$ und $\overline{\{(1,0),(1,1)\}}=B$\\
	Also ist $\mathcal{E}'''$ eine (minimale) abgeschlossene $\sigma$-Mengenalgebra. (im vergleich zu $\mathcal{P}(\Omega)$ fehlt z.B. $\{(1,1)\}$)

\end{document}