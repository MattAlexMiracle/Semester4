\documentclass{article}
\usepackage{listings}
\usepackage{mathrsfs}
\usepackage[utf8]{inputenc}
\usepackage{amssymb}
\usepackage{lipsum}
\usepackage{amsmath}
\usepackage{fancyhdr}
\usepackage{geometry}
\usepackage{scrextend}
\usepackage[english,german]{babel}
\usepackage{titling}
\setlength{\droptitle}{-3cm}
\usepackage{tikz}
\usepackage{algorithm,algpseudocode}
\usepackage[doublespacing]{setspace}
\usetikzlibrary{datavisualization}
\usetikzlibrary{datavisualization.formats.functions}
\usepackage{polynom}
\usepackage{amsmath}
\usepackage{gauss}
\usepackage{tkz-euclide}
\usetikzlibrary{datavisualization}
\usetikzlibrary{datavisualization.formats.functions}
\author{
Alexander Mattick Kennung: qi69dube\\
Kapitel 1
}
\usepackage{import}
\date{\today}
\geometry{a4paper, margin=2cm}
\usepackage{stackengine}
\parskip 1em
\newcommand\stackequal[2]{%
  \mathrel{\stackunder[2pt]{\stackon[4pt]{=}{$\scriptscriptstyle#1$}}{%
  $\scriptscriptstyle#2$}}
 }
\makeatletter
\renewcommand*\env@matrix[1][*\c@MaxMatrixCols c]{%
  \hskip -\arraycolsep
  \let\@ifnextchar\new@ifnextchar
  \array{#1}}
\makeatother
\lstset{
  language=haskell,
}
\lstnewenvironment{code}{\lstset{language=Haskell,basicstyle=\small}}{}
\usepackage{enumitem}
\setlist{nosep}
\usepackage{titlesec}

\titlespacing*{\subsection}{0pt}{2pt}{3pt}
\titlespacing*{\section}{0pt}{0pt}{5pt}
\titlespacing*{\subsubsection}{0pt}{1pt}{2pt}
\title{Vorlesung 4}


\begin{document}
	\maketitle
	\section{Ablaufschritte bei bestimmung einer Riemann-Dichte.}
	\section{Gleichverteilung auf einer Menge, z.B. Rechteck-V}
	1. Bestimme Maß von M
	\[\int_M 1dM=\mu(M)=|M|\]
	2. X gleichverteilt auf M:
	\[f^X(x) = \frac{1}{|M|}1_M(x)\]
	``beliebige Verteilung''\\
	$X\sim N(a,\sigma^2)$ $y\sim Beta(5,7)$ $z\sim \Chi^2(3)$ $V\sim \Gamma_{\alpha,\beta}$\\
	\[f^X(x)=\frac{1}{\sqrt{2\pi}\sigma}\exp(-\frac{(x-a)^2}{2\sigma^2})\]
	\section{Poission zur approximation der Bino-verteilung}
	\[f(k,\lambda) = \frac{\lambda^k}{k!} e^{-\lambda}\]
	\textbf{verteilung seltener Ereignisse}\\
	Also z.B. #Anrufe in callcenter pro Zeiteinheit.\\
	Der Erwartungswert des Eintretens pro Zeiteinheit z.B. 5Anrufe/h reicht. Das ist $\lambda$, hier also $\lambda = 5$\\
	Wir sprechen hier von z.B. Ankunftsprozessen und Materialfehler pro Flächeneinheit.\\
	\section{Gamma Verteilung}
	\[f(x;\alpha,v) =\frac{\alpha^v}{\Gamma(v)}x^{v-1} e^{-\alpha x} 1_{(0,\infty)} (x)\]
	$\alpha$ ist ein inverser Skalenparameter.\\
	v ist de Formparameter.\\
	Wenn v= 1 $\frac{\alpha}{1} e^{-\alpha} 1_{(0,\infty)} (x) = \alpha e^{-\alpha x}$\\
	Verwendungen in Warteschlangentheorie, um Bedienzeiten oder Reperaturzeiten zu beschreiben.\\
	Versicherungsmathematik zur Modellierung kleinerer bis mittlere Schäden.\\
	Lebensdauer von Geräten.\\
	\section{Exponentialverteilung}
	$f(x;\lambda) = \lambda e^{-\lambda x}1_{(0,\infty)}(x) $\\
	Verwendung:\\
	Ankunftsprozesse und Bedienprozesse.\\
	Beschreiben von Zerfallsprozessen. (Prop von einem Atom, kann auch über Poisson modeliert werden)\\
	Beschreibung von Lebensdauer (ohne Gedächtnis$\to$ Ausfallrate ist Zeitunabhängig, P Blatt 4, P27), Wartezeit bis zum Ausfall.\\
	Eigenschaft: Additive Ereignisse sind Unabhängig $P(x\geq a+b) = P(a)P(b)$\\
	\section{Erlang Verteilung}
	\[f(x,k,\lambda) = \frac{\lamda^k}{(k-1)!} x^{k-1}e^{-\lambda x}\]
	sonderfall der Gamma verteilung.\\
	Anwendung:\\
	Warteschlangentheorie, Verteilung der Zeitpsanne zwischen Ereignissen eines Poisson-Prozesses\\
	Beschreibung von Lebensdauer. (mit alterung)\\
	Wahrscheinlichkeit dafür, dass nach Verstreichen des Orts oder Zeitabstandes x das n-te Ereigniss eintritt.\\
	$\lambda$ Ereignisse pro Einheitsintervall.\\
	\subsection{Ausblick}
	Seien $X_i$ iid Zufallsvariablen.\\
	Frage: $\sum^n_{i=1}X_i$ (also wann sind alle z.B. sicherheitssysteme Ausgefallen?)\\
	Summenverteilung/Faltung.\\
	\section{Weibull-Verteilung}
	Z.B. für modellierung von Ausfallraten über Zeit.\\

\end{document}
