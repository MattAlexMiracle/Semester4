\documentclass{article}
\usepackage{listings}
\usepackage{mathrsfs}
\usepackage[utf8]{inputenc}
\usepackage{amssymb}
\usepackage{lipsum}
\usepackage{amsmath}
\usepackage{fancyhdr}
\usepackage{geometry}
\usepackage{scrextend}
\usepackage[english,german]{babel}
\usepackage{titling}
\setlength{\droptitle}{-3cm}
\usepackage{tikz}
\usepackage{algorithm,algpseudocode}
\usepackage[doublespacing]{setspace}
\usetikzlibrary{datavisualization}
\usetikzlibrary{datavisualization.formats.functions}
\usepackage{polynom}
\usepackage{amsmath}
\usepackage{gauss}
\usepackage{tkz-euclide}
\usetikzlibrary{datavisualization}
\usetikzlibrary{datavisualization.formats.functions}
\author{
Alexander Mattick Kennung: qi69dube\\
Kapitel 1
}
\usepackage{import}
\date{\today}
\geometry{a4paper, margin=2cm}
\usepackage{stackengine}
\parskip 1em
\newcommand\stackequal[2]{%
  \mathrel{\stackunder[2pt]{\stackon[4pt]{=}{$\scriptscriptstyle#1$}}{%
  $\scriptscriptstyle#2$}}
 }
\makeatletter
\renewcommand*\env@matrix[1][*\c@MaxMatrixCols c]{%
  \hskip -\arraycolsep
  \let\@ifnextchar\new@ifnextchar
  \array{#1}}
\makeatother
\lstset{
  language=haskell,
}
\lstnewenvironment{code}{\lstset{language=Haskell,basicstyle=\small}}{}
\usepackage{enumitem}
\setlist{nosep}
\usepackage{titlesec}
\usepackage{ stmaryrd }
\usepackage{verbatim}


\titlespacing*{\subsection}{0pt}{2pt}{3pt}
\titlespacing*{\section}{0pt}{0pt}{5pt}
\titlespacing*{\subsubsection}{0pt}{1pt}{2pt}
\title{Vorlesung 4}


\begin{document}
	\maketitle
	Die Kovarianzmatrix ist eine Matrix aus n Normalverteilten ZV, wobei in Zeile x und spalte y die Kovarianz zwischen der x-ten und y-ten ZV ist\\
	Ausgangspunkt ist also ein Zufalls\underline{vektor} X der länge m.\\
	Wie ist X verteilt: $X_i$ sind Normalverteilt $X_i\sim N(a_i,k_{ij})$\\
	$Kov(X_i, X_j) = k_{ij}$ dann sagen wir $X\sim N(a,K)$\\
	(Diese Definition kann theoretisch auf beliebige Verteilungen erweitert werden. Dann ebenfalls $Kov(X_i,X_j)=:k_{ij}$)\\
	Auch hier gilt $k_{ii} =Var(X_i)$, $k_{ij}=Kov(X_j,X_i)$\\
	Eigenschaften der \\
	Var (x) = Kov(X,X)\\
	X,Y st.u EX,EY$<\infty\implies$  Kov(X,Y)=0 (ES IST NUR $\iff$ für Normalverteilung)\\
	$Kov(X,Y)=E((X-EX)(Y-EY)) = E(XY-Y(EX)- X(EY)+(EX)(EY))= E(XY)-E(X(EY))-E(Y(EX))+E(EX)(EY) = E(XY)-EX\cdot EY$\\
	$=\int xyf^{(X,Y)}(x,y)d(x,y)-(\int xf^{(X)}(x)dx)(\int yf^{(Y)}(y)dy) \stackrel{X,Y,\ st.u.}{=}\int xyf^{(X,Y)}(x,y)d(x,y)-\int xyf^{(X,Y)}(x,y)d(x,y) =0 $\\
	Die Kovarianz ist linear $Kov(X+Y, Z) = Kov(X,Z)+Kov(Y,Z)$ (wegen symmetrie in beiden Argumenten linear)\\
	Sei X eine n-dimensionale Standardnormalverteilung mit $Y=AX+a$, daraus folgt\\
	\[Y\sim N(a,AA^T)\]
	$Y=AX+a$ liefert über Transformationssatz
	\[Y=g(X)=AX+A\]
	\[J_g(X)=A, det(J_g)=det(A)\]
	\[f^Y(y) =\frac{1}{\sqrt{|K|}}(\frac{1}{\sqrt{2\pi}})^nexp(-\underbrace{\frac{1}{2}(y-a)^T(K^{-1})(y-a)}_{quadratische\ Form})\]
	Wir wissen, dass wenn $Y\sim N(a,K)$ verteilt ist, gilt \textbf{$X=b+BY$, dann ist $X\sim N(b+Ba,BKB^T)$-verteilt.}\\
	Ziel ist $b+Ca =0$ und $CKC^T =E_n$ einheitsmatrix\\
	$K=AA^T$ gilt, $CAA^TC^T = E_n$, wenn $C=A^{-1}$\\
	$\sum ^n_{i=1} Y_i$ ist $N(\sum^n_{i=1} a_i, \sum^n_{i=1} K_{i,j})$-verteilt


	
\end{document}
