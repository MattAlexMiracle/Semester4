\documentclass{article}
\usepackage{listings}
\usepackage{mathrsfs}
\usepackage[utf8]{inputenc}
\usepackage{amssymb}
\usepackage{lipsum}
\usepackage{amsmath}
\usepackage{fancyhdr}
\usepackage{geometry}
\usepackage{scrextend}
\usepackage[english,german]{babel}
\usepackage{titling}
\setlength{\droptitle}{-3cm}
\usepackage{tikz}
\usepackage{algorithm,algpseudocode}
\usepackage[doublespacing]{setspace}
\usetikzlibrary{datavisualization}
\usetikzlibrary{datavisualization.formats.functions}
\usepackage{polynom}
\usepackage{amsmath}
\usepackage{gauss}
\usepackage{tkz-euclide}
\usetikzlibrary{datavisualization}
\usetikzlibrary{datavisualization.formats.functions}
\author{
Alexander Mattick Kennung: qi69dube\\
Kapitel 1
}
\usepackage{import}
\date{\today}
\geometry{a4paper, margin=2cm}
\usepackage{stackengine}
\parskip 1em
\newcommand\stackequal[2]{%
  \mathrel{\stackunder[2pt]{\stackon[4pt]{=}{$\scriptscriptstyle#1$}}{%
  $\scriptscriptstyle#2$}}
 }
\makeatletter
\renewcommand*\env@matrix[1][*\c@MaxMatrixCols c]{%
  \hskip -\arraycolsep
  \let\@ifnextchar\new@ifnextchar
  \array{#1}}
\makeatother
\lstset{
  language=haskell,
}
\lstnewenvironment{code}{\lstset{language=Haskell,basicstyle=\small}}{}
\usepackage{enumitem}
\setlist{nosep}
\usepackage{titlesec}

\titlespacing*{\subsection}{0pt}{2pt}{3pt}
\titlespacing*{\section}{0pt}{0pt}{5pt}
\titlespacing*{\subsubsection}{0pt}{1pt}{2pt}
\title{Vorlesung 4}


\begin{document}
	\maketitle
	Entscheidende bei $\sigma$-Algebras
	Dies für mengenalgebras algemein
	$\Omega\in\mathscr{A}$
	$A\in\mathscr{A}\implies C_\Omega(A)\in\mathscr{A}$
	$A_i\in\mathscr{A}\implies \bigcup\limits^n_{i=0} A_i$
	speziell für sigma Mengenalgebras\\
	$A_i\in\mathscr{A}\implies \bigcup\limits^\infty_{i=0} A_i$\\
	$\Omega = \{1,2,3\}, \Omega'=\{1,2,3,4,5,6\}$\\
	Welche menge ist mit $X(\omega)=2\omega$ beschreibbar\\
	$\{x\in A'\}:= \{\omega\in\Omega: X(\omega)\in A'\}$\\
	z.B. $X\in\{2,4\} = \{1,2\}\subset \Omega$ weil 2*1=2, 2*2=4\\
	und $X\in\{6\} = \{3\}\subset \Omega$ weil 3*2=6\\
	aber nicht $X\in\{5\}=\emptyset$ weil $\nexists \omega\in\Omega: 2\omega=5$.\\
	\section{Maß und Messräume}
	Das paar $(\Omega,\mathscr{A})$ heißt Messraum.\\
	Maßraum ist $(\Omega, \mathcal{A},P(x))$\\
	Messraum $(\Omega, \mathcal{A})$\\
	empirisches Gesetz de großen Zahlen:\\
	$h_n(x) = \frac{|\{x\in \mathscr{A}\}|}{n}$\\
	für $n\to\infty$ gegen Grenzwert.\\
	Annahme (``Heile Welt'')\\
	$A\cap B=\emptyset$ mit $A,B\in\mathscr{A}$
	$h_n(A+B) = h_n(A)+h_n(B)$ (Das plus ist die Vereinigung zweier \textbf{DISJUNKTER} mengen)\\
	$h_n(A\cup B)\leq h_n(A)+h_n(B)$ (sonst)\\
	$0\leq h_n(A)\leq 1$\\
	$h_n(\emptyset) =0$\\
	$h_n(\Omega) =1$\\
	Ein maß auf $\mathscr{A}$ ist  eine Abb $\mu:\mathscr{A}\to \mathbb{R}\cup\{\infty\}$
	\begin{itemize}
		\item $\mu(A)\geq 0$
		\item $\mu(\emptyset)=0$
		\item $\mu(A_1+A_2+\dots)=\mu(A_1)+\mu(A_2)+\dots$ (also disjunkte endliche Vereinigung)
	\end{itemize}
	$P:\mathscr{A}\to\mathbb{R}$ heißt wahrscheinlichkeitsmaß (W-maß)
	\begin{itemize}
		\item $P(A)\geq0\forall a\in\mathscr{A}$ (nichtnegativität)
		\item $P(\Omega)=1$ (Normiertheit)
		\item $P(\sum\limits^\infty_{i=1}A_i)=\sum\limits^\infty_{i=1}P(A_i)$ ($\sigma$-Additiv)
	\end{itemize}
	Einpunktverteilung $\begin{cases} 1& wenn\ x=A\\0&sonst\end{cases}$
	\begin{itemize}
		\item $P(A^C)=1-P(A)$
		\item $P(A\setminus B) = P(A)-P(A\cap B)$
		\item $P(A\cup B) = P(A)+P(B)-P(A\cap B)$
		\item $P(A\cup B)\leq P(A)+P(B)$
		\item $A\subset B\implies P(A)\leq P(B)$
	\end{itemize}
	Bedingte Wahrscheinlichkeit:\\
	\[P(A|B) = \frac{P(A\cap B)}{P(B)}\]
	oder $P(A\cap B) = P(B)\cdot P(A|B)$ (ähnlich der definition der Relativen häufigkeit)\\
	Totale Wahrscheinlichkeit:\\
	\[P(A)=\sum_{i\in I} P(A\cap B_i)=\sum_{i\in I}P(A|B_i)P(B_i)\]
	Also alle ``Branches'' des Baumes zusammenaddieren!\\
	\[P(R)=P(R|W)\cdot P(W)+P(R|M)P(M) = P(R|W)\cdot P(w)+P(R|M)(1-P(W))\]\\
	Umstellen.\\
	\section{REGEL VON BAYES}
	\[P(B_k|A)=\frac{P(A|B_k)P(B_k)}{\sum_{i\in I}P(A|B_i)P(B_i)} = \frac{P(A|B_k)P(B_k)}{P(A)}=\frac{P(A\cap B_k)}{P(A)}\]
	Stochastische Unabhängigkeit prior ist egal:\\
	$P(A|B)=P(A)$.\\
	Zwei Ereignisse A und B heißen stochastisch unabhängig bzgl P, wenn gilt:
	\[P(A\cap B)=P(A)P(B)\]
	oder
	\[P(A\cap B)=P(A|B)P(B)\]
	Weiterführende Aufgaben:\\
	Beispiel für einen Raum, in dem die Forderung $P(\sum\limits^\infty_{i=1}A_i)=\sum\limits^\infty_{i=1}P(A_i)$ notwendig ist.\\
	$(\Omega,\mathbb{B}, P)$-Maßraum.\\
	z.B. Wahrscheinlichkeit eine prim zahl aus den natürlichen Zahlen zu ziehen.\\
	(Summe ist vereinigung disjunkter Mengen, also $A\cap B=\emptyset \implies A\cup B=A+B$)\\
	Wahrscheinlichkeitsraum für Einpunktverteilung anhand Beispiel.\\
	Ein Wahrscheinlichkeitsraum ist ein Maßraum $(\Omega,\mathcal{A},P)$ dessen maß P ein wahrscheinlichkeitsmaß ist.\\
	$\Omega$ ist eine nichtleere Ergebnismenge, Elemente heißen Elementarereignisse\\
	$\mathcal{A}$ ist ein $\sigma$-Algebra über $\Omega$, elemente heißten Ereignisse\\
	$P:\mathcal{A}\to [0,1]$ ist ein Wahrscheinlichkeitsmaß.\\
	Ein beispiel für eine Einpunktverteilung wäre ein Würfel.\\
	$A=\{1,2,3,4,5,6\}\to P(A)=\begin{cases} 1&a\in A\\0&sonst\end{cases}$ hier $P(A)$ ``almost surely''\\
	Aus einem Deck mit 52 Pik-Assen ein Pik-Ass ziehen.\\
	Oder die Wahrscheinlichkeit die Exakte mitte einer Zielscheibe nicht zu treffen.\\
	(Das Ziel ist (0,0))\\
	Machen Sie sich die Rechenregeln 1-8 für Wahrscheinlichkeitsmaße
	anhand eines Würfels deutlich.

\end{document}